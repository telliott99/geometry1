\documentclass[11pt, oneside]{article} 
\usepackage{geometry}
\geometry{letterpaper} 
\usepackage{graphicx}
	
\usepackage{amssymb}
\usepackage{amsmath}
\usepackage{parskip}
\usepackage{color}
\usepackage{hyperref}

\graphicspath{{/Users/telliott/Dropbox/Github-Math/figures/}}
% \begin{center} \includegraphics [scale=0.4] {gauss3.png} \end{center}

\title{Angular measure}
\date{}

\begin{document}
\maketitle
\Large

%[my-super-duper-separator]

\label{sec:angular_measure}

\begin{center} \includegraphics [scale=0.4] {lines_angles_7.png} \end{center}

We can see that obviously

\[ \phi > \theta \]

Writing the statement $\phi > \theta$ is easy, but for this to make sense (what if they are almost equal?), we must have some way of taking the measure of an angle.  We can't just rely on a picture.

Our answer is to construct a circle around the central point.  

$\circ$  The measure of the angle is the distance along the circumference between the points where the lines cross the circle.

If that distance along the edge is larger for $\phi$ than for $\theta$, then $\phi > \theta$.  In the left panel, the arc between $Q$ and $R$ (call it arc $QR$) is larger than arc $PQ$.

\begin{center} \includegraphics [scale=0.4] {lines_angles_00.png} \end{center}

If we lay off the arcs starting from the same point $P$, the arc $PS$ is longer than the arc $PQ$ (right panel).

We don't need to actually measure the arc itself to do this.  Measuring curved lengths is a bit tricky to do.

Instead we can use a standard compass to lay off the linear distance from $P$ to $S$ and compare that with the distance from $P$ to $Q$.  Since the distance from $P$ to $S$ is more than from $P$ to $Q$, $\phi > \theta$.

This doesn't work very well as $S$ approaches the diameter through $O$ and $P$, but we could lay off the distance in two (or more than two) parts and that would be fine.  For example, take a unit circle, draw a diameter and the perpendicular bisector to form four right angles.  Then bisect each right angle and bisect again.  That would divide the circle into 16 arcs of equal length.

No matter how the measurement is to be performed, we \emph{define} the measure of an angle $s$ to be equal to the arc $a$ it sweeps out, or is subtended by, in a circle of radius $1$, a \emph{unit} circle.  Angles are not lengths, but numerically, the measure of the angle is the measure of the arc.

\begin{center} \includegraphics [scale=0.3] {arcs11.png} \end{center}

\subsection*{right angles}

$\bullet$ \ the sum of all the angles on one side of a line (at a given point), is equal to two right angles.

\begin{center} \includegraphics [scale=0.35] {lines_angles_5.png} \end{center}

This is simply a matter of symmetry and subtraction.  If the sum of all the angles at a point is equal to four right angles, then the sum of the angles on each side of a line through that point is equal to two right angles.

The convention that there are $360^{\circ}$ total in a circle dates to the time of the Babylonians (c. 2400 B.C.).

In degrees, a right angle is $90^{\circ}$ and two supplementary angles measure $180^{\circ}$.  

There is nothing particularly special about using $90^{\circ}$ as the measure of a right angle, or $360^{\circ}$ for one whole turn.  Well, here is one thing:  there are \emph{approximately} 360 days in a year, which marks the sun's track across the sky.  Another idea is that $360$ is special because it has so many factors, which makes it possible to divide up a circle evenly in $2, 3, 4, 5, 6, 8, 9, 10, 12, 15, 18, 20 \dots 180, 360$ parts.

Euclid just talks about angle measurements in terms of right angles.  For example, that the sum of two supplementary angles is equal to two right angles.

In his book, \emph{Measurement}, Lockhart adopts the convention that a whole turn is equal to $1$.  

\subsection*{radians}

We'll just mention here that one whole turn can be defined using a different unit of measure as $2 \pi$ \emph{radians}, and that convention turns out to be quite important for calculus.

It is based on two ideas:  the first, from above, that angular measure is numerically equal to arc length, and second, that in a unit circle, the circumference or total of the arc length is equal to $2 \pi$.

In calculus, all angles will be in radians.  One big reason for that comes in working out the \emph{derivative} of sine and cosine.  There, it will be important to consider the following expression:

\[ \lim_{\theta \rightarrow 0} \frac{\sin \theta}{\theta} = 1 \]

The notation on the left asks us to consider what happens as $\theta$ gets close to the value $0$, and the rest of it states that the ratio $\sin \theta/\theta$ is equal to $1$.  As a result, the derivative of $\sin \theta$ is simply $\cos \theta$.

Well, if you're working in degrees \emph{that's not true}.  There's an awkward constant of proportionality.  So we work in radians.  

It also makes plots prettier, since the sine of $x$ goes from $0$ to $1$ as $x$ goes from $0$ to $\pi/2 \approx 1.57$.  The sine of $1$ degree is effectively zero, so that wouldn't look so nice.

\end{document}