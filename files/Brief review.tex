\documentclass[11pt, oneside]{article} 
\usepackage{geometry}
\geometry{letterpaper} 
\usepackage{graphicx}
	
\usepackage{amssymb}
\usepackage{amsmath}
\usepackage{parskip}
\usepackage{color}
\usepackage{hyperref}

\graphicspath{{/Users/telliott/Github-math/figures/}}
% \begin{center} \includegraphics [scale=0.4] {gauss3.png} \end{center}

\title{Introduction}
\date{}

\begin{document}
\maketitle
\Large

%[my-super-duper-separator]

Let us summarize what we know so far about the basic theorems of geometry.  

They are almost all that you should need to attack any of the other problems in the book, and they'll be used repeatedly.

The first two aren't technically theorems but fundamental assumptions that we make about the geometrical world.

$\circ$ \ \ \hyperref[sec:supplementary_angle_theorem]{\textbf{supplementary angles}}

$\circ$ \ \  \hyperref[sec:alternate_interior_angle_theorem]{\textbf{alternate interior angles (parallel postulate)}}

The next three easily follow from those initial ideas.

$\circ$ \ \ \hyperref[sec:vertical_angle_theorem]{\textbf{vertical angles}}

$\circ$ \ \  \hyperref[sec:triangle_sum_theorem]{\textbf{triangle sum of angles}}

$\circ$ \ \ \hyperref[sec:complementary_angle_theorem]{\textbf{complementary angles}}

We have two basic methods for proving that two triangles are congruent

$\circ$ \ \ \hyperref[sec:SAS]{\textbf{SAS for congruence}}

$\circ$ \ \ \hyperref[sec:ASA]{\textbf{ASA for congruence}}

and one specifically for right triangles

$\circ$ \ \ \hyperref[sec:SSA_in_right]{\textbf{hypotenuse leg in a right triangle (HL)}}

(We proved that SSS is equivalent to SAS, and AAS is equivalent to ASA).

$\circ$ \ \ \hyperref[sec:SSS_implies_SAS]{\textbf{SSS implies SAS}}

Next we have the powerful fundamental theorems of geometry:

$\circ$ \ \  \hyperref[sec:isosceles_triangle_theorem]{\textbf{isosceles triangle theorem}} (sides $\rightarrow$ angles) and \hyperref[sec:isosceles_converse]{\textbf{converse}}

$\circ$ \ \ \hyperref[sec:external_angle_theorem]{\textbf{external angle theorem}}

$\circ$ \ \ \hyperref[sec:Thales_theorem]{\textbf{Thales' circle theorem}} (right angle in a semi-circle)

$\circ$ \ \ \hyperref[sec:area_ratio_theorem]{\textbf{area ratio theorem}}

$\circ$ \ \  \hyperref[sec:similar_right_triangles]{\textbf{similar triangles}} (similar right $\triangle$s, same ratio of sides)

$\circ$ \ \ \hyperref[sec:similarity_and_ratios]{\textbf{general similarity theorem}}

$\circ$ \ \ \hyperref[sec:Pythagoras_similar_triangles]{\textbf{Pythagorean theorem}} (similar triangles)

Some people might add a few more.  But this is a reasonable number to start with.  You should have these instantly available (and it's nice to know how to prove them, as well).

Looking forward, probably the most important we have yet to do is

$\circ$ \ \ \hyperref[sec:peripheral_angle]{\textbf{inscribed angle theorem}} (on a circle is one-half central angle)

which has the consequence that inscribed angles on the same arc are all equal.


\end{document}