\documentclass[11pt, oneside]{article} 
\usepackage{geometry}
\geometry{letterpaper} 
\usepackage{graphicx}
	
\usepackage{amssymb}
\usepackage{amsmath}
\usepackage{parskip}
\usepackage{color}
\usepackage{hyperref}

\graphicspath{{/Users/telliott/Github-math/figures/}}
% \begin{center} \includegraphics [scale=0.4] {gauss3.png} \end{center}

\title{Ceva's theorem}
\date{}

\begin{document}
\maketitle
\Large

%[my-super-duper-separator]

\subsection*{Ceva's theorem}

\label{sec:Ceva_theorem}

\label{sec:ceva_by_menlaus}

We previously showed a proof of Menelaus' theorem.  Let us continue with Ceva's theorem (right panel, below).  We will show that if the \emph{three lines are concurrent} (they all cross at the same point), then

\[ \frac{a_1 \cdot b_1\cdot c_1}{a_2 \cdot b_2\cdot c_2} = 1 \]

\begin{center} \includegraphics [scale=0.5] {menelaus2.png} \end{center}

We ignore here the signed or directed line segments of the original, which give minus $1$ as the product above.  These would cancel at the next step.

\emph{Proof}.

The left panel is from the proof of Menelaus' theorem.  In the right panel, consider the left half-triangle with a side composed of $e + d$.  Apply Menelaus' theorem once.
\[ \frac{d}{e} \cdot \frac{b_1}{b_2} \cdot \frac{c}{c_2} = 1 \]

For the right half-triangle, apply Menelaus again but go clockwise from the middle:
\[ \frac{d}{e} \cdot \frac{a_2}{a_1} \cdot \frac{c}{c_1} = 1 \]

Combine the two results:
\[ \frac{b_1}{b_2} \cdot \frac{c}{c_2} = \cdot \frac{a_2}{a_1} \cdot \frac{c}{c_1} \]
\[ \frac{a_1}{a_2} \cdot \frac{b_1}{b_2} \cdot \frac{c_1}{c_2} = 1 \]

$\square$

\subsection*{proof based on area}

\label{sec:ceva_by_area}

Draw $\triangle ABC$ and let $P$ be a point somewhere inside.  Extend lines from each vertex through $P$ to reach the opposite side.

\begin{center} \includegraphics [scale=0.22] {ceva_thm1.png} \end{center}

Consider two of the smaller triangles, say, $\triangle APB$ and $\triangle APC$.

\begin{center} \includegraphics [scale=0.22] {ceva_thm2.png} \end{center}

If we think of $AP$ as the base for both triangles, then $BE$ is the altitude of the first one and $CF$ is the altitude of the second.  Since they have the same base, the ratio of areas is in the proportion:
\[ \frac{\mathcal{A}_{APB}}{\mathcal{A}_{APC}} = \frac{BE}{CF} \]

But now consider $\triangle BEX$ and $\triangle CFX$.  They are both right triangles with shared vertical angles.  Thus
\[ \triangle BEX \sim \triangle CFX \]
As similar triangles, the sides are in proportion
\[ \frac{BX}{CX} = \frac{BE}{CF} = \frac{\mathcal{A}_{APB}}{\mathcal{A}_{APC}} \]

Reversing the order of the triangles and using a simpler notation for the parts of side $a$:
\[ \frac{a}{a'} = \frac{\mathcal{A}_{APC}}{\mathcal{A}_{APB}} \]

\begin{center} \includegraphics [scale=0.22] {ceva_thm3.png} \end{center}

This result is completely general.  $P$ could be anywhere, and we can compare any two of the three triangles.
\[ \frac{b}{b'} = \frac{\mathcal{A}_{BPC}}{\mathcal{A}_{APC}}, \ \ \ \ \ \ \ \ \frac{c}{c'} = \frac{\mathcal{A}_{APB}}{\mathcal{A}_{BPC}} \]

 \begin{center} \includegraphics [scale=0.22] {ceva_thm4.png} \end{center}
 Multiplying:
 \[ \frac{\mathcal{A}_{APC}}{\mathcal{A}_{APB}} \cdot \frac{\mathcal{A}_{BPC}}{\mathcal{A}_{APC}} \cdot  \frac{\mathcal{A}_{APB}}{\mathcal{A}_{BPC}} =  \frac{a}{a'} \cdot  \frac{b}{b'} \cdot  \frac{c}{c'}  \]
 
 Since the left-hand side cancels, we obtain:
 \[ \frac{a}{a'} \cdot  \frac{b}{b'} \cdot  \frac{c}{c'} = 1 \]

$\square$

This is Ceva's theorem.

The proof also works in reverse,  We change notation slightly, since we will use the prime symbol for another purpose.

\[ \frac{a_1}{a_2} \cdot \frac{b_1}{b_2} \cdot  \frac{c_1}{c_2}  = 1 \iff \text{3 lines cross at point P} \]

\emph{Proof}.

Everything is the same as before, except we suppose that the ratio of the parts of $a$ has to be slightly different in order to obtain $1$ when we multiply everything together.

Of course, this would mean that the line from $A$ would no longer go through $P$.  Suppose the new line gives $a_1'$ and $a_2'$ as the components of the side $a$.

\[ a_1 \ne a_1', \ \ \ \ \ \  a_2 \ne a_2' \]

But, by the forward theorem we have that the line which \emph{does} go through $P$ divides the side into $a_1$ and $a_2$ such that
\[ \frac{a_1}{a_2} \cdot \frac{b_1}{b_2} \cdot  \frac{c_1}{c_2}  = 1 \]

We conclude that
\[ \frac{a_1}{a_2} = \frac{a_1'}{a_2'} \]

Since they are parts of the same length, the components are individually equal.  That is, if the whole side is $a$ then $a_2 = a - a_1$ and $a_2' = a - a_1'$ so
\[ \frac{a_1}{a - a_1} = \frac{a_1'}{a - a_1'} \]
\[ a_1 a - a_1 a_1' = a_1' a - a_1 a_1' \]
\[ a_1 = a_1' \]

$\square$

\subsection*{centroid}

In the general case, the crossing lines are called cevians.  In the case of medians, they divide the sides opposite in half, and then the central point is called the centroid.  Recall that we had

\[ \frac{a}{a'} = \frac{\mathcal{A}_{APC}}{\mathcal{A}_{APB}} \]
\[ \frac{b}{b'} = \frac{\mathcal{A}_{BPC}}{\mathcal{A}_{APC}} \]

But if $a = a'$ and $b = b'$, then $\mathcal{A}_{APC} = \mathcal{A}_{APB} = \mathcal{A}_{BPC}$.  

For a physical object, the point $P$ would be the \emph{center of mass}.

\subsection*{Centroid}

\label{sec:centroid_one_third}

Consider the $\triangle ABC$. Draw the medians $BE$ and $CF$ (bisectors of the sides).  Extend $AG$ through the intersection of the two medians at $G$ to $D$ and then finally to $H$, where $CH$ is drawn parallel to $BGE$.  We profess not to know anything about $BH$ yet.

\begin{center} \includegraphics [scale=0.3] {centroid_pgram1.png} \end{center}

Since $CH \parallel BGE$ and $AE = EC$, we have that $\triangle AGE \sim \triangle AHC$ with ratio $2$ by the midpoint theorem (or what we've called SAS similarity).  Thus, $AG = GH$.

But we also have $AF = FB$, which means that $\triangle AFG \sim \triangle ABH$ with ratio $2$.  This gives $BH \parallel FGC$.

Therefore, $BGCH$ has two pairs of opposing sides parallel so it is a parallelogram.  The diagonals cross at their midpoints, which means that $AD$ is a median ($BC$ is bisected at $D$).  

In addition, $GD = DH$.  Thus, $GD$ is one-half of $GH$, and $GH$ is equal to $AG$, so $GD$ is one-quarter of $AH$ and one-third of $AD$.  $G$ lies on all three medians, and is called the \emph{centroid} of the triangle.  For a physical triangle it would be the center of mass.

\subsection*{Centroid alternate proof}
\begin{center} \includegraphics [scale=0.16] {centroid4.png} \end{center}

The medians divide any triangle into six equal small triangles.

\emph{Proof}.

$(\triangle GBX) = (\triangle GCX)$ by the \hyperref[sec:area_ratio_theorem]{\textbf{area-ratio theorem}}.  Hence the two small triangles on any one side have equal area:  $x = x'$, $y = y'$, $z = z'$.

Also by the same theorem, $(\triangle ABX) = (\triangle ACX)$.  So by subtraction $(\triangle AGB) = (\triangle AGC)$:  $y + y' = z + z'$.  Thus $y = z$.  But this is true for any side.  Hence all six triangles are equal in area.

$\square$

Consider $\triangle GBX$ and $\triangle ABX$ on base $XGA$.  They have the same altitude, from the base to the vertex at $B$.  The ratio of areas is $1:3$.  So that must also be the ratio $XG:XA$.

The centroid lies one-third of the way up the median from the side.

\subsection*{orthocenter}

\label{sec:orthocenter_proof}

Consider this triangle in which we have drawn the altitudes to each side.  We claim that they cross at a single point, called the orthocenter.

\begin{center} \includegraphics [scale=0.4] {ceva5.png} \end{center}

Let the angles be $A, B, C$ as labeled, and the sides opposite be $a, b, c$, subdivided as shown.

Then $\angle A$ is part of two right triangles, and by similar triangles we have that
\[ \frac{c_1}{b} = \frac{b_2}{c} \ \ \ \rightarrow \ \ \ \frac{b_2}{c_1} = \frac{c}{b} \]

Similarly for $\angle B$
\[ \frac{a_1}{c} = \frac{c_2}{a} \ \ \ \rightarrow \ \ \ \frac{c_2}{a_1} = \frac{a}{c} \]

And $\angle C$
\[ \frac{b_1}{a} = \frac{a_2}{b} \ \ \ \rightarrow \ \ \ \frac{a_2}{b_1} = \frac{b}{a} \]

The product of the three right-hand sides above is $c/b \cdot a/c \cdot b/a = 1$.  Therefore the product of the left-hand sides is also $1$:
\[ \frac{b_2}{c_1} \cdot \frac{c_2}{a_1} \cdot \frac{a_2}{b_1} = 1  \]

Invert and re-order the terms
\[ \frac{a_1}{a_2} \cdot \frac{b_1}{b_2} \cdot \frac{c_1}{c_2} = 1  \]

$\square$

\begin{center} \includegraphics [scale=0.25] {ceva4.png} \end{center}

Since we have satisfied Ceva's condition, the 3 altitudes all cross at a single point.  That point is the orthocenter, and this is a proof that it exists.

\subsection*{another approach to the orthocenter}

\begin{center} \includegraphics [scale=0.4] {ceva5b.png} \end{center}

There is a different set of similar triangles one can use for the orthocenter.  The triangle with sides $a_2$, $d$ and $k$ is similar to the triangle with sides $c_1$, $h$, and $e$.  For one pair, we obtain

\[ \frac{e}{k} = \frac{c_1}{a_2} \]

You should be able to use the other two pairs to construct Ceva's ratio equal to 
\[ \frac{e}{k} \cdot \frac{k}{g} \cdot \frac{g}{e} \]

which is, of course, equal to $1$.

Another easily derived relationship is that
\[ \frac{e}{k} = \frac{h}{d} \]
so
\[ de = hk \]

Going around the triangle we will get
\[ de = fg = hk  \]

This occurs because the position of $P$ with respect to each altitude is the same fraction of the whole.  

\subsection*{another proof of Ceva's theorem}

\label{sec:ceva_alternate_proof}

\begin{center} \includegraphics [scale=0.2] {ceva10.png} \end{center}
I found an alternative approach in a geometry textbook by Jurgensen \emph{et al}.  It's a bit weird because this is the only mention of the theorem in the book.  I like the proof, which is only hinted at, because we enhance the original diagram, and it provides exercise using similar triangles.

Draw $AX$ and $BY$ parallel to $CPZ$.

I found the algebra to be a little cleaner with different notation, using lowercase letters for the lengths.
\begin{center} \includegraphics [scale=0.2] {ceva11.png} \end{center}

\emph{Proof}.

Parallel lines give similar triangles with the ratios:
\[ \frac{a}{a'} = \frac{q}{e'}, \ \ \ \ \ \ \ \ \frac{b'}{b} = \frac{p}{e'} \]
which combine to give
\[ \frac{q}{p} = \frac{a}{a'} \cdot \frac{b}{b'} \]

We also have
\[ \frac{c + c'}{c} = \frac{q}{e}, \ \ \ \ \ \ \ \ \frac{c + c'}{c'} = \frac{p}{e} \]
combined
\[ \frac{c'}{c} = \frac{q}{p} \]

Equating the two results
\[ \frac{c'}{c} = \frac{a}{a'} \cdot \frac{b}{b'} \]
which rearranges to give
\[ \frac{a}{a'} \cdot \frac{b}{b'} \cdot \frac{c}{c'} = 1 \]

$\square$

\subsection*{Viviani}

Here's a problem from Acheson that looks challenging, but yields easily to the right perspective.

\begin{center} \includegraphics [scale=0.4] {Viviani.png}  \end{center}

Rather than letting $P$ be a special point, it can be anywhere inside the triangle.  $P$ is the opposite of a special point, it is completely general.  However, we are also given that the triangle is equilateral.

Let $\triangle ABC$ be equilateral, and let $P$ be an arbitrary point internal.
\begin{center} \includegraphics [scale=0.20] {Viviani3.png}  \end{center}

Draw the lines perpendicular to each of the sides from $P$.  The sum of the three lengths is the same no matter where $P$ is chosen inside $\triangle ABC$.

\emph{Proof}.

Draw the lines connecting the three vertices with $P$.
\begin{center} \includegraphics [scale=0.20] {Viviani4.png}  \end{center}

$\mathcal{A}_{ABC}$ is the sum of the areas of the three small triangles:
\[ 2\mathcal{A}_{ABC} = af + bg + ch \]
Let $a=b=c=s$
\[ 2\mathcal{A}_{ABC} = s(f + g + h) \]

We also know that the altitude of an equilateral triangle is in the ratio to the side as $\sqrt{3}/2$ so twice the area is
\[ 2\mathcal{A}_{ABC} = \frac{\sqrt{3}}{2} \ s \cdot s \]

It follows that 
\[ f + g + h =  \frac{\sqrt{3}}{2} \ s  \]

which is a constant for any given triangle, independent of the position of $P$ inside this equilateral triangle.

\subsection*{problem (Posamentier 1.18)}

In $\triangle ABC$, draw $DE \parallel AC$.

Let $BM$ be the median to side $b$.

Draw $AE$ and $CD$.  

Then, the two lines $AE$ and $CD$ are concurrent with $BM$ at a point, $P$.

\begin{center} \includegraphics [scale=0.2] {Ceva_prob.png}  \end{center}

\emph{Proof}.

By the converse of Ceva's theorem, the lines will be concurrent if
\[ \frac{AM}{MC} \cdot \frac{CE}{EB} \cdot \frac{BD}{DA} = 1 \]

Since $AM = MC$, the first term is just $1$.  Then, it remains to show that
\[ \frac{CE}{EB} \cdot \frac{BD}{DA} = 1 \]

We have $\triangle ABC \sim \triangle DBE$ so
\[ \frac{BD}{AD} = \frac{BE}{CE} \]

Substituting, we satisfy the condition for concurrence.
\[ \frac{CE}{BE} \cdot \frac{BE}{CE} = 1 \]

$\square$

\end{document}