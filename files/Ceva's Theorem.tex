\documentclass[11pt, oneside]{article} 
\usepackage{geometry}
\geometry{letterpaper} 
\usepackage{graphicx}
	
\usepackage{amssymb}
\usepackage{amsmath}
\usepackage{parskip}
\usepackage{color}
\usepackage{hyperref}

\graphicspath{{/Users/telliott/Github-math/figures/}}
% \begin{center} \includegraphics [scale=0.4] {gauss3.png} \end{center}

\title{Ceva's theorem}
\date{}

\begin{document}
\maketitle
\Large

%[my-super-duper-separator]

\subsection*{Ceva's theorem}

\label{sec:Ceva_theorem}

\label{sec:ceva_by_menlaus}

We previously showed a proof of Menelaus' theorem.  Let us continue with Ceva's theorem (right panel, below).  We will show that if the \emph{three lines are concurrent} (they all cross at the same point), then

\[ \frac{a_1 \cdot b_1\cdot c_1}{a_2 \cdot b_2\cdot c_2} = 1 \]

\begin{center} \includegraphics [scale=0.5] {menelaus2.png} \end{center}

We ignore here the signed or directed line segments of the original, which give minus $1$ as the product above.  These would cancel at the next step.

\emph{Proof}.

The left panel is from the proof of Menelaus' theorem.  In the right panel, consider the left half-triangle with a side composed of $e + d$.  Apply Menelaus' theorem once.
\[ \frac{d}{e} \cdot \frac{b_1}{b_2} \cdot \frac{c}{c_2} = 1 \]

For the right half-triangle, apply Menelaus again but go clockwise from the middle:
\[ \frac{d}{e} \cdot \frac{a_2}{a_1} \cdot \frac{c}{c_1} = 1 \]

Combine the two results:
\[ \frac{b_1}{b_2} \cdot \frac{c}{c_2} = \cdot \frac{a_2}{a_1} \cdot \frac{c}{c_1} \]
\[ \frac{a_1}{a_2} \cdot \frac{b_1}{b_2} \cdot \frac{c_1}{c_2} = 1 \]

$\square$

\subsection*{proof based on area}

\label{sec:ceva_by_area}

We will look at two more, very classic proofs of Ceva's theorem.  These again show the close relationship between similarity in triangles, and area.

The second one depends on \hyperref[sec:area_ratio_theorem]{\textbf{area-ratio theorem}} we developed previously.

\begin{center} \includegraphics [scale=0.5] {area11.png} \end{center}

This is simple algebra based on the area formula:  one-half the base times the height.  Since the two different shaded triangles have the same altitude, the ratio of their areas is the same as the ratio of the length of the bases.

In the triangle shown below, draw a line from the vertex $A$ to the opposite side $BC$, dividing it into lengths $a_1$ and $a_2$.

\begin{center} \includegraphics [scale=0.5] {ceva_new1.png} \end{center}

Drop the altitudes from the other vertices to the line.  We have vertical angles as well as the corresponding right angles, hence the two triangles are similar with ratios:
\[ \frac{a_1}{a_2} = \frac{h_1}{h_2} \]

\begin{center} \includegraphics [scale=0.5] {ceva_new2.png} \end{center}

Hence the altitudes are in the same ratio as the parts of the side $a_1/a_2$.

Now, pick a point \emph{anywhere} on that line.

\begin{center} \includegraphics [scale=0.5] {ceva_new3.png} \end{center}

We'll focus on $P$, but $Q$ would give the same result.  The triangle $\triangle APB$ can be visualized as having base $AP$ and altitude $h_1$, while $\triangle APC$ has the same base $AP$ and altitude $h_2$.

Since they have the identical base, the areas are in same the ratio as the altitudes, but these are, in turn, in the same ratio as the parts of the side.
\[ \frac{\Delta_{APB}}{\Delta_{APC}} = \frac{h_1}{h_2} = \frac{a_1}{a_2} \]

Let us abbreviate those areas as $L = \Delta_{APC}$ and $M = \Delta_{APB}$.  We have

\begin{center} \includegraphics [scale=0.5] {ceva_new4.png} \end{center}

\[ \frac{M}{L} = \frac{a_1}{a_2} \]

Now we just appeal to symmetry.

\begin{center} \includegraphics [scale=0.5] {ceva_new5.png} \end{center}

\[ \frac{K}{M} = \frac{b_1}{b_2} \]
\[ \frac{L}{K} = \frac{c_1}{c_2} \]

Multiply together to obtain a big cancelation:

\[ \frac{M}{L} \cdot  \frac{K}{M} \cdot  \frac{L}{K} = \frac{a_1}{a_2} \cdot \frac{b_1}{b_2} \cdot  \frac{c_1}{c_2} \]
\[ 1 = \frac{a_1}{a_2} \cdot \frac{b_1}{b_2} \cdot  \frac{c_1}{c_2} \]

This is the forward version of Ceva's theorem.  If we draw the three lines to be concurrent, then the parts of the sides are in this ratio, which we can also write
\[ \frac{a_1}{a_2} \cdot \frac{b_1}{b_2} =  \frac{c_2}{c_1} \]

$\square$

The proof also works in reverse,
\[ \frac{a_1}{a_2} \cdot \frac{b_1}{b_2} \cdot  \frac{c_1}{c_2}  = 1 \iff \text{3 lines cross at point P} \]

\emph{Proof}.

Everything is the same as before, except we suppose that the ratio of the parts of $a$ has to be slightly different in order to obtain $1$ when we multiply everything together.

Of course, this would mean that the line from $A$ would no longer go through $P$.  Suppose the new line gives $a_1'$ and $a_2'$ as the components of the side $a$.

\[ a_1 \ne a_1', \ \ \ \ \ \  a_2 \ne a_2' \]

But, by the forward theorem we have that the line which \emph{does} go through $P$ divides the side into $a_1$ and $a_2$ such that
\[ \frac{a_1}{a_2} \cdot \frac{b_1}{b_2} \cdot  \frac{c_1}{c_2}  = 1 \]

We conclude that
\[ \frac{a_1}{a_2} = \frac{a_1'}{a_2'} \]

Since they are parts of the same length, the components are individually equal.  That is, if the whole side is $a$ then $a_2 = a - a_1$ and $a_2' = a - a_1'$ so
\[ \frac{a_1}{a - a_1} = \frac{a_1'}{a - a_1'} \]
\[ a_1 a - a_1 a_1' = a_1' a - a_1 a_1' \]
\[ a_1 = a_1' \]

$\square$

\subsection*{centroid}

In the general case, the crossing lines are called cevians.  In the case of medians, they divide the sides opposite in half, and then the central point is called the centroid.  Recall that we had

\[ \frac{M}{L} = \frac{a_1}{a_2} \]
\[ \frac{K}{M} = \frac{b_1}{b_2} \]

But if $a_1 = a_2$ and $b_1 = b_2$, then $K = L = M$.  

Furthermore, the two component triangles of region $K$ have equal area, because they share the vertex at $P$ and their bases are equal.  (Which also implies the previous result).

Therefore, in the case of medians, all six small triangles are equal in area.

\begin{center} \includegraphics [scale=0.5] {ceva_new4.png} \end{center}

For a physical object, the point $P$ would be the \emph{center of mass}.

\subsection*{Centroid}

\label{sec:centroid_one_third}

Consider the $\triangle ABC$. Draw the medians $BE$ and $CF$ (bisectors of the sides).  Extend $AG$ through the intersection of the two medians at $G$ to $D$ and then finally to $H$, where $CH$ is drawn parallel to $BGE$.  We profess not to know anything about $BH$ yet.

\begin{center} \includegraphics [scale=0.3] {centroid_pgram1.png} \end{center}

Since $CH \parallel BGE$ and $AE = EC$, we have that $\triangle AGE \sim \triangle AHC$ with ratio $2$ by the midpoint theorem (or what we've called SAS similarity).  Thus, $AG = GH$.

But we also have $AF = FB$, which means that $\triangle AFG \sim \triangle ABH$ with ratio $2$.  This gives $BH \parallel FGC$.

Therefore, $BGCH$ has two pairs of opposing sides parallel so it is a parallelogram.  The diagonals cross at their midpoints, which means that $AD$ is a median ($BC$ is bisected at $D$).  

In addition, $GD = DH$.  Thus, $GD$ is one-half of $GH$, and $GH$ is equal to $AG$, so $GD$ is one-quarter of $AH$ and one-third of $AD$.  $G$ lies on all three medians, and is called the \emph{centroid} of the triangle.  For a physical triangle it would be the center of mass.

\subsection*{Centroid alternate proof}
\begin{center} \includegraphics [scale=0.20] {centroid4.png} \end{center}

The medians divide any triangle into six equal small triangles.

\emph{Proof}.

$(\triangle GBX) = (\triangle GCX)$ by the \hyperref[sec:area_ratio_theorem]{\textbf{area-ratio theorem}}.  Hence the two small triangles on any one side have equal area:  $x = x'$, $y = y'$, $z = z'$.

Also by the same theorem, $(\triangle ABX) = (\triangle ACX)$.  So by subtraction $(\triangle AGB) = (\triangle AGC)$:  $y + y' = z + z'$.  Thus $y = z$.  But this is true for any side.  Hence all six triangles are equal in area.

$\square$

Consider $\triangle GBX$ and $\triangle ABX$ on base $XGA$.  They have the same altitude, from the base to the vertex at $B$.  The ratio of areas is $1:3$.  So that must also be the ratio $XG:XA$.

The centroid lies one-third of the way up the median from the side.

\subsection*{orthocenter}

\label{sec:orthocenter_proof}

Consider this triangle in which we have drawn the altitudes to each side.  We claim that they cross at a single point, called the orthocenter.

\begin{center} \includegraphics [scale=0.4] {ceva5.png} \end{center}

Let the angles be $A, B, C$ as labeled, and the sides opposite be $a, b, c$, subdivided as shown.

Then $\angle A$ is part of two right triangles, and by similar triangles we have that
\[ \frac{c_1}{b} = \frac{b_2}{c} \ \ \ \rightarrow \ \ \ \frac{b_2}{c_1} = \frac{c}{b} \]

Similarly for $\angle B$
\[ \frac{a_1}{c} = \frac{c_2}{a} \ \ \ \rightarrow \ \ \ \frac{c_2}{a_1} = \frac{a}{c} \]

And $\angle C$
\[ \frac{b_1}{a} = \frac{a_2}{b} \ \ \ \rightarrow \ \ \ \frac{a_2}{b_1} = \frac{b}{a} \]

The product of the three right-hand sides above is $c/b \cdot a/c \cdot b/a = 1$.  Therefore the product of the left-hand sides is also $1$:
\[ \frac{b_2}{c_1} \cdot \frac{c_2}{a_1} \cdot \frac{a_2}{b_1} = 1  \]

Invert and re-order the terms
\[ \frac{a_1}{a_2} \cdot \frac{b_1}{b_2} \cdot \frac{c_1}{c_2} = 1  \]

$\square$

\begin{center} \includegraphics [scale=0.25] {ceva4.png} \end{center}

Since we have satisfied Ceva's condition, the 3 altitudes all cross at a single point.  That point is the orthocenter, and this is a proof that it exists.

\subsection*{another approach to the orthocenter}

\begin{center} \includegraphics [scale=0.4] {ceva5b.png} \end{center}

There is a different set of similar triangles one can use for the orthocenter.  The triangle with sides $a_2$, $d$ and $k$ is similar to the triangle with sides $c_1$, $h$, and $e$.  For one pair, we obtain

\[ \frac{e}{k} = \frac{c_1}{a_2} \]

You should be able to use the other two pairs to construct Ceva's ratio equal to 
\[ \frac{e}{k} \cdot \frac{k}{g} \cdot \frac{g}{e} \]

which is, of course, equal to $1$.

Another easily derived relationship is that
\[ \frac{e}{k} = \frac{h}{d} \]
so
\[ de = hk \]

Going around the triangle we will get
\[ de = fg = hk  \]

This occurs because the position of $P$ with respect to each altitude is the same fraction of the whole.  

\subsection*{another proof of Ceva's theorem}

\label{sec:ceva_alternate_proof}

I found an alternative approach to proving Ceva's theorem as an exercise in a geometry textbook by Jurgensen \emph{et al}.  It's a bit weird because this is the only mention of the theorem in the book.  I like the proof, which is only hinted at, because we enhance the original diagram, and it provides exercise using similar triangles.

The problem says to extend line segments parallel to $CX$ from $A$ and $B$ and form triangles with the extensions of $BZ$ and $AY$.

\begin{center} \includegraphics [scale=0.35] {ceva6.png} \end{center}

The hint given is to consider similar triangles.  There are a number of them, due to the three parallel lines.  I found the algebra to be a little cleaner with different notation, using lowercase letters for most of the sides.

\begin{center} \includegraphics [scale=0.35] {ceva8.png} \end{center}

We have that $\triangle CYP \sim \triangle MBY$ because $CX \parallel MB$.  So
\[ \frac{x}{CP} = \frac{a}{a'} \]

With similar logic, we find that
\[ \frac{y}{CP} = \frac{b'}{b} \]

Hence 
\[ \frac{x}{y} = \frac{a}{a'} \cdot \frac{b}{b'} \]

We also have similar triangles that involve the base $AB$, such as the one with vertex $N$ compared to vertex $P$.  $\triangle ABN \sim \triangle XBP$ so that
\[ \frac{y}{XP} = \frac{c + c'}{c'} \]

With similar logic, we find that
\[ \frac{x}{XP} = \frac{c + c'}{c} \]

So that
\[ \frac{x}{y} = \frac{c'}{c} \]

Combined with the previous result:
\[ \frac{x}{y} = \frac{a}{a'} \cdot \frac{b}{b'} = \frac{c'}{c} \]

Thus

\[ \frac{a}{a'} \cdot \frac{b}{b'} \cdot  \frac{c}{c'} = 1 \]

\subsection*{Viviani}

Here's a problem from Acheson that looks challenging, but yields easily to the right perspective.

\begin{center} \includegraphics [scale=0.4] {Viviani.png}  \end{center}

Rather than letting $P$ be a special point, it can be anywhere inside the triangle.  $P$ is the opposite of a special point, it is completely general.

However, we are also given that the triangle is equilateral.

\begin{center} \includegraphics [scale=0.4] {Viviani2.png}  \end{center}

Note that $AP$ is not the continuation of the vertical down to side $a$.  If you look closely, there is a kink, a change of direction, at $P$.  None of the dotted lines is parallel to any other line.

The area of $\triangle APB = hc/2$, and so on for the sub-triangles, so the area of the whole triangle is
\[ 2 \Delta = hc + fa + gb \]

But since the triangle is equilateral, $a = b = c = s$.
\[ 2 \Delta = s(f + g + h) \]
\[ \frac{2 \Delta}{s} = f + g + h \]

From the properties of equilateral triangles, we also know that for a triangle with sides of unit length, the altitude has length $\sqrt{3}/2$, so twice the area $2 \Delta = \sqrt{3}/2$, 

and then finally
\[ \frac{\sqrt{3}}{2} = f + g + h \]

and that is true no matter where $P$ is placed inside this equilateral triangle.

\subsection*{problem}

Draw $PQ$ parallel to $BC$, then draw $CP$ and $BQ$.  If $AM$ is the median of $\triangle ABC$, these lines are concurrent.

\begin{center} \includegraphics [scale=0.4] {Posamentier1_18.png}  \end{center}

By (parts of) similar triangles

\[ \frac{CQ}{QA} = \frac{PB}{AP} \]
\[ \frac{CQ}{QA} \cdot \frac{AP}{PB} = 1 \]

since $BM = MC$
\[ \frac{CQ}{QA} \cdot \frac{AP}{PB} \cdot \frac{BM}{MC} = 1 \]

By the converse of Ceva's theorem, these lines are concurrent.

\end{document}