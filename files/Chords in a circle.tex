\documentclass[11pt, oneside]{article} 
\usepackage{geometry}
\geometry{letterpaper} 
\usepackage{graphicx}
	
\usepackage{amssymb}
\usepackage{amsmath}
\usepackage{parskip}
\usepackage{color}
\usepackage{hyperref}

\graphicspath{{/Users/telliott/Github-math/figures/}}
% \begin{center} \includegraphics [scale=0.4] {gauss3.png} \end{center}

\title{Rectangles in a circle}
\date{}

\begin{document}
\maketitle
\Large

%[my-super-duper-separator]

\label{sec:equal_arcs_equal_chords}

Chords are straight lines inside a circle, with endpoints which lie on the circle.  If a chord is extended (produced, in Euclid's terminology), then the extended chord becomes a secant, with part of the line lying outside the circle.

$\bullet$ \ If two arcs are equal, the corresponding chords are also equal, and vice-versa.

\emph{Proof}.  (forward)

Draw the radii flanking both arcs.  By definition, we have equal angles for the equal arcs, and therefore congruent triangles by SAS.  The equal chords are opposing sides to the central angle in the congruent triangles.

$\square$

\emph{Proof}.  (converse)

Draw the radii flanking both chords.  We have congruent triangles by SSS, so the central angles are equal.  By Euclid III.26, we have equal arcs.

$\square$

We may occasionally not even point out the equivalence, but just assume that if two arcs are equal, the corresponding chords are also equal, and vice-versa.

\subsection*{inequality}

Suppose we have one angle greater than another, or one chord greater than another?

\emph{Proof}.  (Sketch).  Start with two chords in a circle.  Draw the radii for both.  Then, by the \hyperref[sec:hinge_theorem]{\textbf{hinge theorem}}, Euclid I.24, the greater angle lies opposite the greater third side.  Thus, the greater chord lies in the greater circumference because they are connected through the central angle.  $\square$

\subsection*{crossed chords}

\begin{center} \includegraphics [scale=0.3] {arcs6.png} \end{center}
Previously, we showed that when two arcs cross in a circle, each of the two equal vertical angles is equal to the \emph{average} of the two arcs they cut out.  It is also apparent that the two triangles formed are similar.

In two similar triangles, the sides opposite equal angles are in the same ratio, so:
\[ \frac{a}{c} = \frac{d}{b} \]
It follows that
\[ ab = cd \]
The products of the two parts of each chord are equal.

\subsection*{bisected chord theorem}

\label{sec:perpendicular_bisector_of_a_chord}

\label{sec:chords_and_bisectors}

Given any chord of a circle and a line perpendicular to it, if the perpendicular is a radius of the circle then the chord is bisected.  

Conversely, if the perpendicular is a bisector, then it is also a radius.  

\begin{center} \includegraphics [scale=0.3] {perp_chords6.png} \end{center}

\emph{Proof}.

For the forward theorem:  we have that $OQ$ is a radius (it lies on the diameter), and it is perpendicular to chord $PR$.  

We have $OP = OR$, as radii of the circle.  Also, side $OS$ is shared, and there are right angles at $S$.  So $\triangle OPS \cong \triangle ORS$ by hypotenuse-leg in a right triangle (HL).  

We conclude that $PS = SR$.  

$\square$

In addition, since $\triangle OPS \cong \triangle ORS$, $\angle POS = \angle ROS$, the arcs $PQ$ and $QR$ are cut out by equal central angles, therefore they are equal.

The converse theorem was established when we derived the method to construct a circle given any three points on it.  

Briefly, all points that are equidistant from the two ends of a chord must lie on its perpendicular bisector.  Since the center of the circle containing the chord forms radii when joined to the two ends, it must also lie on the perpendicular bisector.  

Therefore the perpendicular bisector is a diameter of the circle.

$\square$

One consequence is that if a line is the perpendicular bisector of a chord of the circle, it is also the perpendicular bisector of any other chord parallel to the first.

Furthermore, if we have a tangent and any chord that is parallel to it, then the radius drawn to the point of tangency bisects the chord.

This follows simply from the definition of the tangent as perpendicular to the radius at the point of tangency.

\begin{center} \includegraphics [scale=0.3] {perp_chords7.png} \end{center}

By our previous theorem, $OT$ bisects $PR$ so that $PS = SR$.

$\circ$  \ Given a point of tangency on a radius perpendicular to any chord, not only the chord but the arc lengths are evenly divided by the radius.  $PT = RT$.

We can also note that if the distance from $S$ to the periphery of the circle is $d$ and the height of the half chord is $h$, by the crossed chord theorem we have
\[ (2r - d)(d) = h^2 \]
which can be solved as a quadratic in $d$.
\[ d^2 - 2rd + h^2 = 0 \]
\[ d = \frac{2r \pm \sqrt{4r^2 - 4h^2}}{2} \]
\[ = r \pm \sqrt{r^2 - h^2} \]
which is already fairly obvious since $OP = r$ and, it's the hypotenuse in a right triangle with $PS = h$, so $OS$ is exactly the square root, while $r$ minus that is just $d$.

\subsection*{Euclid II.14}

\label{sec:Euclid_II_14}

As an aside, suppose it is desired to construct the square equal in area to a given rectangle $APQR$.

Extend $AP$ to $B$ such that PB = PQ.  

Bisect AB to find center M for a circle with radius $AM = BM$.
\begin{center} \includegraphics [scale=0.20] {EII_14.png} \end{center}

By II.5
\[ AP.PB = MB^2 - MP^2 \]
\[ AP \cdot PQ = MC^2 - MP^2 = PC^2 \]

Thus $PC$ is the side of the desired square.

This can also be done by similar triangles, since $ \triangle CAP \sim \triangle BCP$.  Thus $PB/PC = PC/AP$ so $AP \cdot PB = PC^2$.

$\square$

Since we can get from Pythagoras to the crossed chord theorem, it may not be so surprising that we can go backward.

\subsection*{Pythagorean theorem by crossed chords}

\label{sec:PProof_chords}

\emph{Proof}.

A right triangle with sides $a,b$ and $c$ is inscribed in a circle.  By Thales' theorem, the hypotenuse $c$ is a diagonal.  Draw two more diagonals, parallel to the sides $a$ and $b$.  We can use similar right triangles to show that these bisect sides $a$ and $b$.
\begin{center} \includegraphics [scale=0.45] {pyth20.png} \end{center}

So then, we will use the crossed chords theorem, multiplying the two halves of $a$ together.  The question is, what are the two parts of the dotted diameter that crosses $a$ at right angles?

The long part is $c/2 + b/2$.  Can you see why?  

The very short part is $c - (c/2 + b/2) = c/2 - b/2$.  The reason is that when added to the long part, the result is just
\[ c/2 + b/2 + c/2 - b/2 = c \]

We have
\[ \frac{a}{2} \cdot \frac{a}{2} = (\frac{c}{2} + \frac{b}{2})  (\frac{c}{2} - \frac{b}{2}) \]
\[ a \cdot a = (c + b)(c - b)  = c^2 - b^2 \]
\[ a^2 + b^2 = c^2 \]

$\square$

$\bullet$ \ The crossed chords formed for the vertices of any cyclic quadrilateral form two pairs of similar triangles.  The sides that are in the same ratio are on different chords.  The fundamental ratio $a/A = b/B$ leads directly to the crossed chord theorem.

The basic rule is that corresponding sides lie opposite corresponding (equal) angles.  In the figure $a$ and $A$ lie opposite to the angle marked with a red dot.
\begin{center} \includegraphics [scale=0.18] {crossed_chords3.png} \end{center}

In the left panel (circle) corresponding sides lie on different chords, while in the right panel they lie on the same chord.  With crossed chords, the corresponding angles and sides are in clockwise order, while for the other case the order is counter-clockwise.

If in doubt, make sure that the sides you think are corresponding sides lie opposite to equal angles.

\subsection*{another view}
Another way to look at this is shown in the following diagram:
\begin{center} \includegraphics [scale=0.25] {annulus.png} \end{center}

Given a larger circle of radius $R$ and a chord of it with length $2a$, draw the smaller circle of radius $r$ that just touches the chord.  Then, by the Pythagorean theorem we have that 
\[ R^2 - r^2 = a^2 \]
But $R^2 - r^2$ (times $\pi$) is the area of the annulus or ring, the difference between the area of the outer circle and that of the inner one.  This is equal to $\pi \ a^2$.

So then, there is family of circles of different sizes that can be drawn, each with a chord of length $2a$ (as long as the radius is greater than $a$).  For each of those circles, one can draw the annulus as we did above, and find that the area is exactly the same.

\begin{center} \includegraphics [scale=0.75] {annulus2.png} \end{center}

\url{https://puzzles.nigelcoldwell.co.uk}

\subsection*{problem}

Given that the line touching the circle at $T$ is a tangent and that $BC$ is parallel to it.  Show that the angles marked with black dots are equal.
\begin{center} \includegraphics [scale=0.4] {perp_chords8.png} \end{center}

I think the idea of the problem was to use similar triangles somehow.  But given our previous work, this is a trivial consequence of the fact that the arcs $CT$ and $TB$ are equal.

So, as peripheral angles corresponding to equal arcs, $\angle CAT = \angle TAB$.

\subsection*{diameters form a rectangle}

\label{sec:diameters_form_rectangle}

$\circ$  When any two diameters of a circle are drawn and consecutive chords on the circle joined together, the result is a rectangle.

\begin{center} \includegraphics [scale=0.14] {rect_in_circle.png} \end{center}

\emph{Proof}.

A simple proof relies on Thales' circle theorem.  All four vertices are right angles.  

It follows that both pairs of opposing sides are parallel.  For example 
\[ AD \perp AB \perp BC \Rightarrow AD \parallel BC \]

So $ABCD$ is a parallelogram with right-angled vertices, a rectangle.

$\square$

One can, of course, also use the fact that certain angles are subtended by equal arcs, so they are equal (for example $\angle DAC = \angle DBC$). Then use congruent triangles.

\subsection*{problem}

Construct a square circumscribed by a circle, i.e. all four vertices lie on the circle.

\emph{Solution}.  (Sketch).

The diagonals originate at vertices, by definition.  We also know that the diagonals of a square are perpendicular.  Hence, erecting the perpendicular bisector of any diagonal of the circle locates four points on the circle that are vertices of a circumscribed square.

$\square$

\subsection*{parallel chords joined by equal arcs and chords}

\begin{center} \includegraphics [scale=0.14] {parallel_chords2.png} \end{center}

Given any two parallel chords in a circle, $AB$ and $CD$.  Draw $AD$.

The inscribed angles formed, $\angle ADC$ and $\angle BAD$, are equal by alternate interior angles,

Thus arc $AC = $ arc $BD$.

Equal arcs have equal chords so $AC = BD$.

$\square$

It is a simple matter to extend this result to triangles such as $\triangle ABC \cong \triangle BAD$.

\emph{Proof}.

Since the arcs are equal $\angle ABC = \angle BAD$.

\begin{center} \includegraphics [scale=0.14] {parallel_chords2.png} \end{center}

But $\angle BDA$ and $\angle ACB$ both cut the same arc, $AB$.

So $\triangle ABD$ and $\triangle BAC$ have three angles equal, and they share the same side $AB$ opposite one pair of equal angles.

$\triangle ABD \cong \triangle BAC$ by ASA.

$\square$

\subsection*{rectangle side on a circle}

\label{sec:rectangle_side_on_a_circle}

We derive a useful theorem about any rectangle in a circle, and its converse.

\begin{center} \includegraphics [scale=0.16] {rect_in_circle2.png} \end{center}

Let $ABCD$ be a rectangle such that $A$ and $B$ lie on the circle, but $C$ and $D$ do not.

Then the extensions of $DC$ are equal, namely $DE = CF$.

\emph{Proof}.

Draw $AF$.  As before, $\angle AFE = \angle BAF$.

Since the inscribed angles are equal, so are the arcs:  arc $AE =$ arc $BF$.

Equal arcs have equal chords, hence $AE = BF$.

So $\triangle ADE \cong \triangle BCF$ by HL.

It follows that $DE = CF$.

$\square$

\emph{Proof}.  (Alternate).

Suppose that $ACGE$ is a rectangle.  Let $BF$ be the perpendicular bisector of $AC$.

\begin{center} \includegraphics [scale=0.5] {perp_chords9.png} \end{center}

Then, the extension of $BF$ passes through the center of the circle (it is part of a radius), by the converse of the theorem discussed earlier.  Every point that is equidistant from $A$ and $C$, including the center of the circle, lies on the extension of $BF$.

$OB$ is also the perpendicular bisector of $DH$, since $ACGH$ is a rectangle.  So
\[ DF = FH \]

but $AB = BC = EF = FG$.  Subtracting equals from equals:
\[ DF - EF = FH - FG \]
\[ DE = GH \]

$\square$

For any rectangle with two vertices lying on a circle and two inside the circle, when the side which connects the inside points is extended both ways to reach the circle, the extensions will be equal.

\subsection*{converse}

Above, we started with a rectangle in a circle.  Now we start simply with any two parallel chords.  Then one will be longer than the other (and may even be a diameter).

Suppose we have $DH \parallel AC$.
\begin{center} \includegraphics [scale=0.5] {perp_chords9.png} \end{center}

Drop $AE \perp DH$ and $CG \perp DH$.  By alternate interior angles, $\angle A$ and $\angle C$ are right angles, as well as $\angle AEF$ and $\angle CGF$.  Hence $ACGE$ is a rectangle.

$\square$

Then all the consequences of the forward theorem follow.  In particular $DE = GH$.  

And of course, the perpendicular bisector of $AC$ is also the perpendicular bisector of $DH$.  (\emph{Proof}.  Draw $\triangle OAB$ and $\triangle OCB$ as well as $\triangle ODF$ and $\triangle OHF$ to show that all the angles at $B$ and $F$ are right angles.  $\square$)

\subsection*{Extraordinary property} 

\label{sec:extraordinary_property}

According to Acheson, this theorem comes from a book by Malton where it is described as an ``extraordinary property of the circle".

\begin{center} \includegraphics [scale=0.6] {Acheson_G110.png} \end{center}

Let two chords of a circle meet at right angles, and let the arms of the chords be $a + b$ and $c + d$.

By crossed chords we have $ab = cd$.

We will show that he squares of the four components add up to a constant, and that constant is equal to $(2R)^2$, twice the radius of the circle, squared.

\[ a^2 + b^2 + c^2 + d^2 = (2R)^2 = 4R^2 \]

The key to the proof is to form a rectangle.

Pick one of the chords, say $a + b$.  Draw the diameter of the circle that terminates on one end of each of the chords.  The other diameter joins the other ends of chords $a + b$ but hasn't been drawn.

\begin{center} \includegraphics [scale=0.17] {perp_chords3b.png} \end{center}

If the four points at the ends of the two diameters are joined to form a rectangle then the short extension at the top is also equal to $d$, so the height of the rectangle is $c - d$.

We proved this fact about rectangles in a circle \hyperref[sec:rectangle_side_on_a_circle]{\textbf{here}}.  

So then:

\emph{Proof.}

This is trivial now that we have $c - d$.  By the Pythagorean theorem
\[ (a + b)^2 + (c - d)^2 = (2R)^2 \]
\[ = a^2 + b^2 + c^2 + d^2 + 2ab - 2cd \]

But $ab = cd$ so
\[ 4R^2 = a^2 + b^2 + c^2 + d^2 \]

$\square$


\end{document}