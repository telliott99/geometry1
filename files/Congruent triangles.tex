\documentclass[11pt, oneside]{article} 
\usepackage{geometry}
\geometry{letterpaper} 
\usepackage{graphicx}
	
\usepackage{amssymb}
\usepackage{amsmath}
\usepackage{parskip}
\usepackage{color}
\usepackage{hyperref}

\graphicspath{{/Users/telliott/Github-Math/figures/}}

\title{Congruence of triangles}
\date{}

\begin{document}
\maketitle
\Large

%[my-super-duper-separator]

\subsection*{meaning of congruence}

Probably the most fundamental idea concerning triangles is how to decide that two triangles are \emph{congruent}.  If so, they have all 3 sides the same length, and all 3 angles the same measure.

However this is logically the converse (backwards) of what we really want to know.  We would like to be able to say that that two triangles are congruent \emph{if} they have [ some properties ].

We will find tests that, if passed, imply congruence.

\begin{center} \includegraphics [scale=0.4] {congruent.png} \end{center}

We allow one triangle to be rotated at any angle with respect to the other.  An example of this is the two triangles at the top in the figure above.

We will also allow the term congruent to apply to the case of a triangle and its mirror image.  All of the triangles shown are congruent, even though two are flipped --- they are mirror images.

\subsection*{SSS test for congruence}

Perhaps a practical definition would be that if you used a pair of scissors to cut out one triangle and then lay it on top of the second one so that it superimposes exactly, they would be congruent.

In fact this is how Euclid handles the issue in the first theorem of \emph{Elements}, I.4.

A little thought may convince you that if all three corresponding side lengths are equal, two triangles are congruent.  Draw one side of a triangle, and from its endpoints draw circles with radii the length of the other two sides.

\begin{center} \includegraphics [scale=0.2] {SSS_2.png} \end{center}

The two radii map out all the points that are the same distance from the centers.  They include only two possible arrangements for three given side lengths which result in triangles.

We see that the radii cross at two and only two different points, which have mirror image symmetry above and below the original base.  Three given side lengths can only be drawn together to give two resulting triangles, and these two shapes are mirror images.

It is certainly possible to come up with side lengths that \emph{cannot} form a triangle.  Consider (left panel, above) what would happen if the two shorter sides were exactly equal to the longest one.  Then they would meet at a single point on the longest side, and there would be no triangle.

If the sum were shorter, they would not meet at all.  This is called the triangle inequality and we'll prove it more formally later.

Thus, we arrive at a fundamental theorem about congruency for triangles:

$\bullet$  Two triangles are \emph{congruent} if they have the same three side lengths. 

This test is abbreviated SSS (side-side-side).

\subsection*{SAS}

\label{sec:SAS}

In addition to SSS (side-side-side), there are three other conditions that always lead to congruence of two triangles when they are satisfied, namely

$\circ$  SAS (side-angle-side)

$\circ$  ASA (angle-side-angle)

$\circ$  AAS (angle-angle-side)

When we say SAS, we mean that the equal angle we know lies between the pair of equal sides.  Similarly, ASA means that we know one side of a triangle and the two angles at either end of that side are the angles in question.

As we did above for SSS, a good way to think about the congruence condition SAS is to imagine trying to construct a triangle from the given information, and ask whether it is uniquely determined.  

\begin{center} \includegraphics [scale=0.4] {SAS2.png} \end{center}

Two sides and the angle between them are given.  So draw that part of the triangle.  Notice that the second and third vertices are also determined, because they just lie at the ends of the two sides we're given.  All that remains is to draw the line segment that joins them.

Of course we might have drawn the shorter side as the horizontal.  If we did that, it would generate the mirror image, and still be congruent by our rule.

\subsection*{ASA}

\label{sec:ASA}

The next one is ASA.  Since we know two angles, we know the third.  Here is a diagram of the situation:

\begin{center} \includegraphics [scale=0.4] {ASA1.png} \end{center}
 
Draw the known side, then using the known angles, start two other sides from the ends of that side.  They must cross at a unique point.

But...actually, if we start the two lines from opposite ends of the horizontal

\begin{center} \includegraphics [scale=0.4] {ASA4.png} \end{center}

there is another solution, the mirror image.  These two triangles are congruent to the one above.
 
I'm tempted to draw the constructions below the starting line.  But this doesn't give anything new.  These are merely rotated versions of the ones above.  Try it and see.  Congruent triangles include a pair of mirror images and that's it.

Now, if we know two angles we also know the third, by the angle sum theorem.  For this reason, ASA and AAS mean that we have exactly the same information, because we know all three angles and we know one side.  

Crucially, we know \emph{which} two angles flank the known side.  Equivalently, it is enough to know which angle is opposite to the known side.
 
\subsection*{marks for equality}

It is often useful to marking sides and (particularly) angles to show they are equal.  Here is 
  
\begin{center} \includegraphics [scale=0.4] {SAS.png} \end{center}

In this diagram, sides of equal length are indicated by one or more short lines called hash marks.  

Equal angles are usually indicated by dots in this book. Dots are easier to place on the figures, and lend themselves to color-coding;  the common method for pencil and paper is to draw an arc with a hash across it, or just use a filled and open circle.

\begin{center} \includegraphics [scale=0.4] {ASA3.png} \end{center}

\begin{center} \includegraphics [scale=0.4] {AAS.png} \end{center}

\subsection*{but not SSA}

There is one set of three that doesn't work in the general case, and that is SSA (side-side-angle).

Suppose we know the lengths of sides $a$ and $b$ and the angle at vertex $B$, adjacent to $a$ and opposite side $b$.  

\begin{center} \includegraphics [scale=0.2] {ambig.png} \end{center}

Since the length of the third side isn't known we make it a dashed line.  Let's see if we can construct a triangle from this information.

We do not know the angle at vertex $C$ between $a$ and $b$ so we imagine $b$ swinging on a hinge there.  If $b$ is too short, there can't be a triangle.  If the length of $b$ is exactly right, we'll have a right angle at $A''$.  

If $b$ is longer than this but still $b < a$, there are two points $A$ and $A'$ where $b$ can intersect with the side projecting from vertex $B$. 

This is the \emph{ambiguous case}:  we have two possibilities.  If two different triangles match by the SSA criterion, we cannot say whether they are congruent or not without more information.

The right angle case is not ambiguous, but we'll save that for the chapter on right triangles.

If you compare this chapter with most others in the book you'll notice that we have not formally proved that any of these methods are correct.  Even Euclid encounters some difficulty with this point.  He "proves" SAS by a method that is arguably not really a proof.

It won't hurt my feelings if you think of them as axioms.  The famous mathematician David Hilbert does this.  We will revisit the issue when we look at Euclid's proofs.

\subsection*{similar triangles have equal angles}

\label{sec:two_angles_similar}

Sometimes two triangles are not congruent, but have all three angles the same.  We call such triangles \emph{similar}.

\begin{center} \includegraphics [scale=0.4] {similar.png} \end{center}

Similarity means that all three angles are the same but the triangles are of different overall sizes.

Our basic criterion for similarity is AAA (angle-angle-angle).  

However, because of the sum of angles theorem, if any two angles of a pair of triangles are known to be equal, then the third one must be equal as well.  We can say that:

$\bullet$  Two triangles are similar if they have at least two angles equal.

\subsection*{scaling}

For similar triangles, the three corresponding pairs of sides are in the same proportions, but re-scaled by a constant factor.

\begin{center} \includegraphics [scale=0.4] {similar2.png} \end{center}

From the above diagram of two similar triangles, similarity implies that (for example)
\[ \frac{A}{a} = \frac{B}{b} \]

which can be rearranged to give:

\[ \frac{a}{b} = \frac{A}{B} \]

For any pair of similar triangles, there is a constant $k$ such that

\[ k = \frac{A}{a} = \frac{B}{b} = \frac{C}{c} \]

We will come back to proofs about similar triangles in a later chapter.  In particular, we will show that AAA implies equal ratios of sides as given here, as well as the converse.

\subsection*{sides and opposing angles}

When two triangles are congruent, each of the three angles and all three sides are the same.  You might wonder whether the sides could be placed in a different order.  We usually draw side of length $a$ opposite $\angle A$, side $b$ opposite $\angle B$ and side $c$ opposite $\angle C$.  Can we switch the sides so that say, the length $c$ is opposite $\angle B$?

It turns out not to be possible.  Later we will have a theorem which say that in any triangle, the longest side is opposite the largest angle, and the smallest side opposite the smallest angle.  If switching two sides were possible, the theorem would be false.


\end{document}