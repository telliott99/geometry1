\documentclass[11pt, oneside]{article} 
\usepackage{geometry}
\geometry{letterpaper} 
\usepackage{graphicx}
	
\usepackage{amssymb}
\usepackage{amsmath}
\usepackage{parskip}
\usepackage{color}
\usepackage{hyperref}

\graphicspath{{/Users/telliott/Github-Math/figures/}}
% \begin{center} \includegraphics [scale=0.4] {gauss3.png} \end{center}

\title{Arcs of a circle}
\date{}

\begin{document}
\maketitle
\Large

%[my-super-duper-separator]

\label{sec:quadrilateral_supplementary}

There is a wonderfully simple theorem about quadrilaterals (Euclid III.22).  A cyclic quadrilateral is a four-sided polygon whose four vertices all lie on one circle.

\begin{center} \includegraphics [scale=0.12] {EIII_22.png} \end{center}

$\bullet$ \ For \emph{any} cyclic quadrilateral, the opposing angles are supplementary (they sum to two right angles).

\emph{Proof}.

Together, opposing angles in a cyclic quadrilateral exactly correspond to the whole arc of the circle.

Since the central angle for that arc is four right angles, the sum of opposing inscribed angles is just one-half that or two right angles.

$\square$

Euclid's proof uses sum of angles:

\begin{center} \includegraphics [scale=0.12] {EIII_22.png} \end{center}

\emph{Proof}.

$\angle ADC = \angle ADB + \angle BDC$, subtended by arcs $AB$ and $BC$.

As angles on equal arcs (III.21) the latter two angles are equal to $\angle ACB$ and $\angle BAC$.

But by I.32, the same two angles plus the total angle at vertex $B$ are equal to two right angles.

$\square$

\subsection*{converse of cyclic quadrilateral theorem}

This one uses Euclid III.22 rather than III.21.  

\label{sec:inscribed_angles_converse2}

Let $\triangle ABC$ lie on a circle.  

Let point $D$ be such that $\angle ADC$ is supplementary to $\angle ABC$.

Then $D$ lies on the same circle.
\begin{center} \includegraphics [scale=0.22] {cyclic_quad_converse_b} \end{center}

\emph{Proof}.

Aiming for a contradiction, suppose $D$ does not lie on the circle.

Let $D$ be external and $D'$ lie on the point where $BD$ cuts the circle (right panel).  

By the forward theorem, $\angle AD'C$ is supplementary to $\angle ABC$ and so equal to $\angle ADC$.

But by Euclid I.21,  $\angle ADC < \angle AD'C$.  

This is a contradiction.  Therefore $D$ is not external.  

A similar argument will show that $D$ is not internal.  

Therefore $D$ lies on the circle and $D$ and $D'$ are the same point.

$\square$


\subsection*{Brahmagupta's theorem}

This is a theorem credited to Brahmagupta.

\url{https://en.wikipedia.org/wiki/Brahmagupta_theorem}

\begin{center} \includegraphics [scale=0.15] {bg4.png} \end{center}

Given $ABCD$ is a cyclic quadrilateral with diagonals that cross at right angles.

Given $EF \perp DC$.

Show that $AF = BF$.

\emph{Proof}.

We redraw the figure from Wikipedia slightly

\begin{center} \includegraphics [scale=0.15] {bg5.png} \end{center}

We are given that $\triangle AMB$ is right.

If $AF = BF$, then $MF = AF = BF$ by the right triangle midpoint theorem.

This suggests we try to show that $\triangle AFM$ and $\triangle MFB$ are isosceles.

Start with $\triangle AFM$.

We see that $\angle MAF$ is subtended by chord $BC$, but so is $\angle MDE$, so they are equal.

We're given the $\triangle MDC$ is right and that $ME$ is the altitude.

It follows that $\angle CME = \angle MDC$ by similar triangles.

Then $\angle AMF = \angle CME$ by vertical angles.

We have then, $\angle AMF = \angle MAF$.

$\triangle AFM$ is isosceles and so $AF = MF$.

\begin{center} \includegraphics [scale=0.15] {bg5.png} \end{center}

For the other one, we need $\angle FMB = \angle FBM$.

$\angle MBF = \angle MCD$, as inscribed angles on the same arc.

And $\angle MCD = \angle DME$ by similar triangles.

$\angle DME = \angle FMB$ by vertical angles.

It follows that $\angle FMB = \angle MBF$.

$\triangle MFB$ is isosceles and so $MF = BF$ by I.6.

Equating equals:  $AF = BF$.

$\square$

\subsection*{problem}

Arcs of a circle often simplify problems.  Draw the altitudes and orthocenter of a triangle and draw its circumcircle.

\begin{center} \includegraphics [scale=0.16] {Posamentier_a.png} \end{center}

Posamentier gives this relationship.  We claim that $HE = ZE$.

\emph{Proof}.

It seems clear that this must be a consequence of a congruence:  it looks like $\triangle AZC \cong \triangle AHC$.  How to prove that?

By looking at arcs subtended, we can get relationships for the new angles in $\triangle AZC$.
\begin{center} \includegraphics [scale=0.16] {Posamentier_b.png} \end{center}

In particular $\angle ACZ$ is complementary to $\angle BZC$, and $\angle ACF$ is complementary to $\angle BAC$, while $\angle BZC$ and $\angle BAC$ are subtended by the same arc, $BC$.  Hence the red dots show they are equal.

Since the angles at $E$ are right and $CE$ is shared, it follows that $\triangle HCE \cong \triangle ZCE$.  It follows that $HE = ZE$.

A similar argument will show that $\triangle HAE \cong \triangle ZAE$, hence $\triangle AHC \cong \triangle AZC$ by SSS.

$\square$

\subsection*{Van Schooten's theorem}

\begin{center} \includegraphics [scale=0.16] {Van_Schooten1.png} \end{center}

This is given as a problem by Surowski (1.3.6).

Given an equilateral triangle $ABC$ draw its circumcircle.

Draw an arbitrary line segment from vertex $A$ through side $BC$ to meet the circle at $M$.  Prove that $AM = BM + MC$.

This is easy to prove as a special case of Ptolemy's theorem.

Nevertheless, we do it as suggested.

\emph{Proof}.

Draw lines from $M$ to each vertex and extend one to give $AM = MD$.

By inscribed angles both angles at $M$ are equal to two-thirds of a right angle.

$\triangle AMD$ is isosceles with the vertex equal to two-thirds of a right angle.

Thus $\triangle AMD$ is equilateral.

Subtract the central $\angle MAC$ from two equal angles to yield $\angle BAM = \angle DAC$.

\begin{center} \includegraphics [scale=0.16] {Van_Schooten2.png} \end{center}

$\triangle BAM \sim \triangle CAD$.

But $BA = CA$ so $\triangle BAM \cong \triangle CAD$ by ASA.

It follows that $BM = CD$.

Thus $MC + CD = MC + BM = MD = AM$.

$\square$

I saw this question on the internet:  ``is a parallelogram a cyclic quadrilateral?''

In a parallelogram, opposing angles are equal.  In a cyclic quadrilateral, opposing angles are supplementary.  The only supplementary, equal angles are two right angles.  So a rectangle is the only parallelogram that is a cyclic quadrilateral.



\end{document}