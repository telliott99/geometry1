\documentclass[11pt, oneside]{article} 
\usepackage{geometry}
\geometry{letterpaper} 
\usepackage{graphicx}
	
\usepackage{amssymb}
\usepackage{amsmath}
\usepackage{parskip}
\usepackage{color}
\usepackage{hyperref}

\graphicspath{{/Users/telliott/Github-Math/figures/}}
% \begin{center} \includegraphics [scale=0.4] {gauss3.png} \end{center}

\title{Euclid's Elements}
\date{}

\begin{document}
\maketitle
\Large

%[my-super-duper-separator]

In this chapter we will study some more \emph{Propositions} from the first volume of Euclid's \emph{Elements}.  They are short, sweet and powerful.  

\subsection*{Euclid I.16}

\label{sec:Euclid_I_16}

$\bullet$  \ In any triangle, if one of the sides is produced (extended), then the exterior angle is greater than either of the interior and opposite angles.

\begin{center} \includegraphics [scale=0.4] {Euclid_I_16.png} \end{center}

In $\triangle ABC$ extend side $BC$ to $D$.  We are concerned with the exterior $\angle ACD$.

The claim is that $\angle ACD$ is greater (larger) than either of the interior angles $\angle ABC$ or $\angle BAC$.

\emph{Proof}.

Find the midpoint of side $AC$ at $E$ so $AE = EC$ and then draw $BEF$ so that $BE = EF$.  Note that $\angle ECD$ is the same angle as $\angle ACD$.

We have equal angles at $E$ by the vertical angle theorem.  So by SAS the two smaller triangles $\triangle BEA$ and $\triangle CEF$ to the left and right are congruent.

Thus $\angle BAE = \angle ECF$.  Since the whole is greater than the part, $\angle ECD > \angle ECF = \angle BAE$.

We can make a similar construction and proof for $\angle ABC$.

The exterior angle is greater than either of the interior and opposite angles.

$\square$

\begin{center} \includegraphics [scale=0.4] {Euclid_I_16.png} \end{center}

One might ask why Euclid doesn't use supplementary angles to obtain a stronger proof.  The main reason is that he has not yet proved the triangle sum theorem.  That is Euclid I.32.

However, we did previously show a proof of the triangle sum theorem, that the sum of angles in any triangle is equal to two right angles.  

And you were asked then to find the relationship between an external angle and the angles of the corresponding triangle.  Here is our version of that proof.

\subsection*{external angle theorem}

\label{sec:external_angle_theorem}

$\bullet$ \ the external angle is equal to the sum of two internal angles

\emph{Proof}.

\begin{center} \includegraphics [scale=0.4] {PI_16d.png} \end{center}

As supplementary angles, $u + u' =$ two right angles.  As the three angles of a triangle, $s + t + u =$ two right angles as well.

But things equal to the same thing are equal to each other:
\[ u + u' = s + t + u \]

Subtracting equals from equals:
\[ u' = s + t \]

$\square$

This relationship is fundamental.

The next theorem is also extremely useful, and it follows from Euclid I.16.

\subsection*{Euclid I.18}

\label{sec:Euclid_I_18}

$\bullet$  \ Comparing two sides in any triangle, if one side is longer than the other, then the angle opposite is larger.

\begin{center} \includegraphics [scale=0.4] {PI_18a.png} \end{center}

\emph{Proof}.

Given that side $b > a$, mark off $a$ on $b$.

\begin{center} \includegraphics [scale=0.4] {PI_18b.png} \end{center}

By the external angle theorem (I.16)
\[ v > t \]



But $v = s'$ (by Euclid I.5) so 
\[ s' > t \]
And since $s > s' $
\begin{center} \includegraphics [scale=0.4] {PI_18a.png} \end{center}
\[ s >  s' > t \]

$\square$

We get the converse almost for free.

\subsection*{Euclid I.19}

\label{sec:Euclid_I_19}

$\bullet$  \ Comparing two angles in any triangle, if one angle is larger than the other, then the side opposite is longer.

\begin{center} \includegraphics [scale=0.4] {PI_18a.png} \end{center}

\emph{Proof}.

We are given $s > t$ and claim that $b > a$.

We proceed by eliminating the other two possibilities.  Rewrite the order of terms as $t < s$, with the claim $a < b$.

It cannot be that $a = b$ because then $s = t$ by isosceles $\triangle$ (Euclid I.5), but we are given $s > t$.

So then suppose $a > b$.  By the previous proposition (Euclid I.18), we would have that $t > s$.  But this is again contrary to what we were given.  

Since the other two possibilities are eliminated, we must have $a < b$.

$\square$

We have made use of what's called a trichotomy.  There are only three possibilities:
\[ a < b, \ \ \ \ \ \ a = b, \ \ \ \ \ \ a > b \]

This applies to line segments and angles as well as many other things.

\subsection*{Euclid I.20.  Triangle inequality}

\label{sec:triangle_inequality}

$\bullet$  \ In any triangle, the longest side is smaller than the sum of the two shorter sides.  

Clearly, if a given side is not the longest, so it is shorter than one of the others, then it must also be shorter than the sum of that other one plus the third.  For this reason, we consider only the longest side.

\begin{center} \includegraphics [scale=0.15] {EI_20.png} \end{center}

Given that side $AC$ is the longest in $\triangle ABC$.

Extend side $BC$ so that $BD = AB$.  By Euclid I.5, $\triangle ABD$ is isosceles, so the angles with blue dots are equal.

Then $\angle D$ is smaller than $\angle DAC$ and therefore, by Euclid I.19 just above, $AC$ is less than $DC$.  

But $DC$ is equal to the sum of the two smaller sides of $\triangle ABC$.  Hence

\[ AC < AB + BC \]

$\square$

An equivalent statement is the famous ``a straight line is the shortest distance between two points''.

The triangle inequality has corresponding statements and proofs in other parts of mathematics including real and complex analysis.

\subsection*{Euclid I.24.  SAS inequality theorem}

\label{sec:hinge_theorem}

\begin{center} \includegraphics [scale=0.16] {Euclid_I_24b.png} \end{center}

The next theorem is called the SAS inequality theorem, or informally, the ``hinge'' theorem.

$\bullet$  \ If two triangles have two pairs of sides which are the same length, the triangle with the larger included angle also has the larger third side.

We have $\triangle ABC$ with sides $a$ and $c$ flanking $\angle \theta$ and $\triangle ZBC$ with sides $a$ and $c$ flanking $\angle \phi$, and $\theta < \phi$.  We claim that side $b$ in $\triangle ABC$ is smaller than side $d$ in $\triangle ZBC$.

Draw the two triangles nestled inside one another.  Notice that $\triangle AZB$ is isosceles, so $\angle AZB = \angle ZAB$.

\emph{Proof}.

\[ \angle AZC < \angle AZB \ \ \ \ \text{and} \ \ \angle ZAB < \angle ZAC  \]
From above
\[ \angle AZB = \angle ZAB \]
So
\[ \angle AZC < \angle ZAC \]

\begin{center} \includegraphics [scale=0.16] {Euclid_I_24b.png} \end{center}
By Euclid I.19, it follows that side $d$ opposite the larger angle $\angle ZAC$ is larger than side $b$ opposite the smaller angle $\angle AZC$ in $\triangle AZC$.

At the same time, in the two triangles $\triangle ABC$ and $\triangle ZBC$ the smaller angle $\theta$ is opposite the smaller side $b$, while the larger angle is $\phi$, opposite the longer side, $d$.

$\square$

\url{https://proofwiki.org/wiki/Hinge_Theorem}

Note that Euclid I.24 says that where the included angle is larger, then the side opposite is larger.  The converse is Euclid I.25.

\emph{Proof}.  Aiming for a contradiction, given side $d$ is smaller than side $b$, suppose that $\angle ABC < \angle ZBC$.  The forward theorem gives $d > b$.  This is a contradiction. $\square$

Another subtlety is that it cannot be that $b = d$, by Euclid I.7, and also then the two triangles would be congruent by SSS, but the angles differ, so this is a contradiction.

One last point:  this proof suffers from the same issue that Euclid I.7 does, namely that a different drawing can be made that makes the proof non-sensical.  The solution is the same as well.

\emph{Proof}.
\begin{center} \includegraphics [scale=0.16] {Euclid_I_24c.png} \end{center}
Since $\triangle ABZ$ is isosceles, the base angles are also equal, namely $\angle AZE = \angle ZAF$.  So then 
\[ \angle AZF < \angle AZE = \angle ZAF < \angle ZAC \]
It follows that since the greater side is opposite the greater angle in $\triangle ACZ$, $d > b$.

$\square$

This is enough of the \emph{Elements} to give us a good taste of the basics of Greek geometry of lines and triangles, and methods of proof.

\subsection*{problem}
\begin{center} \includegraphics [scale=0.16] {isosceles13.png} \end{center}
Let $\triangle ABC$ be isosceles, and $D$ lie on $BC$.  Using Euclid I.19, prove that $DC < DA$.  Separately, use I.20 for the same proof.

Here is another idea.  \emph{Proof}.  (Sketch).  Let the perpendicular bisector of $AC$ be $BE$.  Drop another perpendicular from $D$ to $AC$ at $F$.  So $E$ is the midpoint and $F$ lies directly below $D$.

Clearly, $AF > AE > FC$.  So
\[ FC^2 < AF^2 \]
\[ FC^2 + DF^2 < AF^2 + DF^2 \]
But by the Pythagorean theorem, the left-hand side is $DC^2$, while the right-hand side is $DA^2$.  Taking side lengths as the positive square root, we have $DC < DA$.  $\square$

In fact, \emph{every} point which lies on the same side of the perpendicular bisector of $AC$ as $C$, is closer to $C$ than to $A$.

\subsection*{Triangle inequality by circles}

Here is a modern proof of Euclid I.20.

\emph{Proof}.

Consider $\triangle ABC$.  Draw circle $\mathcal{C}_1$ on center $B$ with radius $AB$ and also circle $\mathcal{C}_2$ on center $C$ with radius $AC$.  The third vertex $A$ lies on the intersection of the two circles.  
\begin{center} \includegraphics [scale=0.25] {tri_inequalityb.png} \end{center}

If $BC$ is longer (greater) than the sum of the two radii, then the circles will not cross one another. 

If $BC$ is equal to this sum, then they only touch, on the line $BC$.  Thus, in order for there to be a triangle, $BC$ must be less than the sum of the diameters:  $BC < AB + AC$.  

$\square$

\url{https://arxiv.org/pdf/1803.01317}

\subsection*{angle of reflection}

Suppose we shine a light at a mirror, or just look at the reflection of someone else, or even our own outstretched hand.  The question arises, what determines the path of the light or the image as it travels to the eye?  Heron of Alexandria discovered a proof of the answer in about 100 A.D.

The diagram shows two \emph{possible} paths light might take, but there is only one path that it \emph{does} take, shown in the right panel.  Light takes the path where it reaches our eyes the fastest.

\begin{center} \includegraphics [scale=0.5] {Acheson_G46.png} \end{center}

What is this path?  What is the angle the ray of light makes with the mirror?  This angle is called the angle of reflection.

Draw $\triangle POB'$, imagined to be on the other side of the mirror, with $B$ the same distance away from the mirror as $B'$, but on the other side.  The two triangles $\triangle BPO$ and $\triangle B'PO$ are congruent by SAS.

Clearly, the shortest distance from $A$ to $B'$ is a straight line, by the triangle inequality.

So the result is that $\angle APQ$ equals $\angle B'PO$, which (by congruent triangles) equals $\angle BPO$.  This is usually stated as "the angle of incidence is equal to the angle of reflection."

\subsection*{smallest perimeter}

\label{sec:smallest_perimeter}

Show that for two triangles with the same area, an isosceles triangle has the \emph{smallest} perimeter. 

\emph{Proof}.

We suppose that the base of the isosceles triangle $\triangle ABC$ is equal to one of the sides of the other triangle $\triangle ABD$.  If we would need to re-scale one side to have equality, we could then make a corresponding change in the altitude to that side to maintain equal area.

The equal area constraint means that points $C$ and $D$ lie along a horizontal line parallel to the common base $AB$.

\begin{center} \includegraphics [scale=0.5] {least_perimeter1.png} \end{center}

Draw a vertical from $B$ to meet the extension of $AC$ at $K$.  Extend $CD$ to meet $KB$ at $H$ and also draw $DK$.

We're given that $AC = BC$ and so $\angle CAB = \angle CBA$ (magenta dots).

\begin{center} \includegraphics [scale=0.5] {least_perimeter2.png} \end{center}

It follows that other angles are also equal to those two (by alternate interior angles).

\begin{center} \includegraphics [scale=0.5] {least_perimeter3.png} \end{center}

$\angle BCH = \angle ABC$ and $\angle KCH = \angle CAB$.  

Since $\angle CAB = \angle ABC$, it follows that $\angle BCH = \angle KCH$.

The angles at $H$ are right angles, since $BK$ is perpendicular to $AB$ and $CH$ is parallel to $AB$.

Therefore $\triangle CHK \cong \triangle BCH$ by ASA, so $BC = CK$.  

Similar reasoning will give that $BD = DK$.

But now by the triangle inequality:
\[ AC + CK < AD + DK \]
Substituting from above
\[ AC + BC < AD + BD \]
Add $AB$ to both sides
\[ AC + BC + AB < AD + BD + AB \]

The perimeter of $\triangle ABC$ is less than that of $\triangle ABD$.

$\square$



\end{document}