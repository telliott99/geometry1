\documentclass[11pt, oneside]{article} 
\usepackage{geometry}
\geometry{letterpaper} 
\usepackage{graphicx}
	
\usepackage{amssymb}
\usepackage{amsmath}
\usepackage{parskip}
\usepackage{color}
\usepackage{hyperref}

\graphicspath{{/Users/telliott/Github-Math/figures/}}
% \begin{center} \includegraphics [scale=0.4] {gauss3.png} \end{center}

\title{Equilateral triangles}
\date{}

\begin{document}
\maketitle
\Large

%[my-super-duper-separator]

\label{sec:equilateral_triangles}

\subsection*{basic triangle}

An equilateral triangle has all three sides the same length.  By symmetry, as there is no reason to favor one vertex, the three vertices have the same angular measure, namely, $\pi/3$ (since the total is two right angles).

\begin{center} \includegraphics [scale=0.3] {equilateral.png} \end{center}

\emph{Proof}.  

The  \hyperref[sec:Euclid_I_5]{\textbf{isosceles triangle theorem}} (Euclid I.5:  equal sides $\rightarrow$ angles), which  can be applied twice to give the result.

We also have \hyperref[sec:Euclid_I_19]{\textbf{Euclid I.19}}, which says that if one angle is larger than another in a triangle, then the corresponding side is also longer.  So, assume the angles are not the same, then that implies the sides are not the same length either.  Hence the angles are the same.  

$\square$

An equilateral triangle is also isosceles (three times over), and as a result the altitude dropped from a vertex to the opposing base bisects both that base and the angle at the vertex.

\begin{center} \includegraphics [scale=0.3] {iso13.png} \end{center}

\subsection*{area}

To calculate the area of the equilateral triangle, we could write a general formula, but it is usually easier to go back to basics and derive one.

\begin{center} \includegraphics [scale=0.4] {equi1.png} \end{center}

In the left panel, the result of bisection is that, for a side of length $2$, half the base is $1$, and then Pythagoras tells us that the altitude is $\sqrt{3}$.  Using the standard formula, the area of this triangle is also $\sqrt{3}$.

To re-scale the triangle, write whichever measurement is given.  For example in the middle we're given a side length of $1$ so that means that all lengths are multiplied by $1/2$.  

On the right, we're given that the altitude is $1$ so then we must multiply by $1/\sqrt{3}$.  The logic is that to erase that $\sqrt{3}$ in the altitude, we must multiply by its inverse.

If you really insist on a general formula...

Suppose the length of the side is $s$, then the ratio of the altitude to the side is
\[ \frac{h}{s} = \frac{\sqrt{3}}{2} \]

Therefore twice the area of the triangle is
\[ 2A = \frac{\sqrt{3}}{2} \cdot s^2 \]

We can check that against a general trigonometric formula for the area of any triangle.  Twice the area is the product of two adjacent sides times the sine of the angle between.  Here the angle is $\pi/3$ and its sine is $\sqrt{3}/2$ so we have $2A = \sqrt{3}/2 \cdot s^2$.

It can be quicker to rely on the fact that the area goes like the square of the side.  For side length $2$ we had $A = \sqrt{3}$.  If we shrink the triangle to side length $1$, the area goes down by a factor of four, to $A = \sqrt{3}/4$.

\subsection*{circumscribed equilateral triangle}

Here is a fun construction based on an equilateral triangle.  

\begin{center} \includegraphics [scale=0.2] {equi_tri.png} \end{center}

Any triangle fits into a unique circle (see the discussion of circles and bisectors of the chords).  Draw the circumcircle for equilateral $\triangle ABC$.

Draw a radius to a vertex of the triangle, and extend it as the diameter $BD$, which crosses the opposing  side at $E$.

We can prove a number of properties.  A good place to start is that $BD$ bisects $\angle B$.

\emph{Proof}.

Connect $AO$ and $CO$ (only one is shown).  

\begin{center} \includegraphics [scale=0.2] {equi_tri2.png} \end{center}

$\triangle AOB \cong \triangle COB$ by SSS.

It follows that $\angle B$ is bisected.

Since $AB = CB$, $\triangle ABE \cong \triangle CBE$ by SAS.

So the angles at $E$ are all equal, and thus right angles.

$\square$

The next property is that $\triangle OAD$ is also equilateral.

\emph{Proof}.

$\angle AOD$ is the central angle for the half-angle at $\angle B$, hence $\angle AOD = \angle B$.

By inscribed angles, $\angle BDA = \angle C$, so $\triangle OAD$ is equi-angular, by sum of angles.  It is therefore equilateral.

$\square$

The three congruence classes of right triangles are all similar. 

\begin{center} \includegraphics [scale=0.2] {equi_tri2.png} \end{center}

\emph{Proof}.

$\triangle BAD$ is a right triangle by Thales' theorem.  

From above, $\triangle BEA$ and $\triangle AED$ are both right as well.

The first two share $\beta$ (the half-angle at $B$), while the third has $\angle ADB$ shared with $\triangle BAD$.

$\square$

$AD = AO = CO = CD$.

\emph{Proof}.

From above, $\triangle OAD$ is equilateral.  Similarly, $\triangle OCD$ is equilateral.

It follows that $AOCD$ is a rhombus.

This would also follow by the converse of the diagonals theorem for parallelograms, since $AE = CE$.

$\square$

Finally, $DE = OE = $ one-half of $OB$ and one-third of $BE$.

\emph{Proof}.

$\triangle BEA \sim \triangle BAD \sim \triangle AED$.

Then $AD$ is one-half $BD$, but $DE$ is one-half $AD$, hence $1/4$ of $BD$.  

Since $OD$ is one-half of $BD$, $DE$ is one-third of $BE$.

$\square$

\emph{Proof}.  (Alternate).

Let the side length be $2$ and then $AE = CE = 1$.  By crossed chords, $AE \cdot CE = 1 = BE \cdot DE$.

But by the Pythagorean theorem $BE = \sqrt{3}$ and then $DE = 1/\sqrt{3}$ and their ratio is
\[ \frac{1}{\sqrt{3} \cdot \sqrt{3}} = \frac{1}{3} \]

$\square$

\emph{Proof}.  (Alternate).

\begin{center} \includegraphics [scale=0.2] {equi_tri2.png} \end{center}

Let $AD = CD = d$ and $r$ be the radius of the circle.

By Ptolemy's theorem:
\[ sd + sd = 2 rs \]
\[ d = r \]

It follows that $AOCD$ is a parallelogram, the diagonals bisect one another, etc.

$\square$

Thus, the altitude of the equilateral triangle is $3/4$ of the diameter of the circle that just encloses it.  And the point where the altitudes meet in an equilateral triangle is $1/3$ of the way up from the base, since $r = 2a$.  

\subsection*{problem}

Here is a problem from Dr. Paul Yiu.  Given that $ABC$ is equilateral and that the points $D$ and $E$ are midpoints of the sides, find $DF/DE$.
\begin{center} \includegraphics [scale=0.2] {equi2b.png} \end{center}

We \emph{could} calculate the length of the perpendicular from the center of the circle ($O$, not labeled) to $DE$ and then use the Pythagorean theorem.

A simpler approach is to set the side length of the triangle to $2$, so each half is $1$, and since $BDE$ is also equilateral, $DE = 1$ (or use the midline theorem).  

Let $DF$ be $x$.  Then, by the theorem of crossed chords we have that
\[ 1 \cdot 1 = (x-1) \cdot x \]
\[ x^2 - x - 1 = 0 \]
Solve using the quadratic equation:
\[ x = \frac{1 + \sqrt{5}}{2} = \phi \]
This is the famous ``golden'' ratio.

Let's do it the hard way as well.  If we keep the side length of the large triangle as $2$, then the small triangle has side length $1$, and the altitude is $\sqrt{3}/2$.

We need the radius.  The altitude of the large triangle is $\sqrt{3}$ and the diameter is $4/3$ of that, so the radius is $2/3$ which makes $r = 2/\sqrt{3}$.  

The distance from $O$ to $DE$ is the difference, and we will need the square of that:
\[ \delta^2 = (\frac{\sqrt{3}}{2} - \frac{2}{\sqrt{3}})^2 \]
\[ = \frac{3}{4} + \frac{4}{3} - 2 \cdot \frac{\sqrt{3}}{2} \cdot \frac{2}{\sqrt{3}} \]
\[ = \frac{3}{4} + \frac{4}{3} - 2 \]

The distance from the midpoint of $DE$ to $F$, squared, is:
\[ \frac{4}{3} - \delta^2 = \frac{4}{3} - (\frac{3}{4} + \frac{4}{3} - 2) = \frac{5}{4} \]
so the square root is $\sqrt{5}/2$ and then we must add $1/2$ to obtain $DF$.  That matches what we had before.

\subsection*{problem}
Here is a second problem from Dr. Yiu.

Crossing lines are drawn from the vertices in an equilateral triangle, such that the larger pair of angles at $P$ are each $120^{\circ}$.
\begin{center} \includegraphics [scale=0.2] {equi3b.png} \end{center}
We have that, since $\angle ZPY + \angle A = $ two right angles, the angles inside the quadrilateral at $Z$ and $Y$ are supplementary, so the opposing ones are equal.

A consequence of that is $\triangle AYB$ and $\triangle BZC$ have all three angles the same, but also share a side, hence they are congruent.  Therefore $AY = BZ$ and $AZ = YC$.

By the same reasoning, $\angle AZC = \angle BYC$, which means that $\triangle AZC$ and $\triangle YCB$ share all three angles plus two sides, so they are congruent.  Therefore $BY = ZC$.

There are a ridiculous number of similarities and equalities here.  The acute angles at $P$ are $60^{\circ}$, which gives more similar triangles such as $\triangle PZB \sim \triangle AYB$ and $\triangle PYC \sim \triangle AZC$.  I write the vertices flanking the short and medium sides in order to make it easy to write the ratios.

We are almost too wealthy.  We are asked to show that
\[ \frac{AY}{AZ} = \frac{PB}{PC} \]

Now, $\triangle BPC$ is not similar to anything else.  But from the first similarity above ($\triangle PZB \sim \triangle AYB$) we get
\[ PZ/ZB/BP = AY/YB/BA \]
so we can write
\[ PB = AB \cdot ZB/YB \]
From the second similarity ($\triangle PYC \sim \triangle AZC$) we get
\[ PY/YC/CP = AZ/ZC/AC \]
so we can write
\[ PC = AC \cdot YC/ZC \]
Then we form the ratio (canceling $AB = AC$)
\[ \frac{PB}{PC}  = \frac{ZB}{YB} \cdot \frac{ZC}{YC} \]
\begin{center} \includegraphics [scale=0.2] {equi3b.png} \end{center}
Using the equalities $ZC = BY$, $AY = ZB$, $AZ = YC$:
\[ \frac{PB}{PC}  = \frac{ZB}{YC} = \frac{AY}{AZ} \]

I don't know if there is an easier solution, just happy to have found one.

\subsection*{Bertrand's paradox}

Grinstead and Snell's wonderful \emph{Introduction to Probability} has this problem (example 2.6).  It's called Bertrand's paradox.  We are asked to draw a chord of a unit circle randomly.

\begin{center} \includegraphics [scale=0.6] {Bertrand1.png} \end{center}

Here we might say, let's choose each of three angles $\alpha$, $\beta$ and $\theta$ randomly (uniform density) from $[0, 2 \pi]$.  

But there is no reason why the radius to $B$ cannot lie along the $x$-axis, so there are really only two choices.  

The question is posed:  what is the probability that the length of this random chord is $> \sqrt{3}$.

However, there are several different approaches to parametrize the problem, and randomizing the different parameters leads to different results.

\subsection*{equilateral triangles}

We briefly review some properties of equilateral triangles which we looked at earlier, with a slightly different take.

Drop an altitude and observe the ratio of side lengths.  It is convenient to start with a side length of 2 for the original triangle, then in the bisected copies the sides are in the ratio 1-2-$\sqrt{3}$ (the $1$ by bisection, and $\sqrt{3}$ by the Pythagorean theorem).

The angle at each vertex of the original equilateral triangle is $\pi/3$, so the new triangles have angles of $\pi/6$, both by bisection or because the altitude forms an angle of $\pi/2$ at the base, so the sum of angles theorem gives us the last angle.

In the right panel, the equilateral triangle is inscribed in a unit circle, so $OR = OP = OQ = 1$.  We claim that the line segment $OM$ has a length of $1/2$.

\begin{center} \includegraphics [scale=0.6] {Bertrand2.png} \end{center}

\emph{Proof}.

$\angle PQR$ is a right angle, by Thales' circle theorem, and $\angle MRQ$ is shared, so $\triangle PQR$ is similar to $\triangle RMQ$.  Therefore, $\angle RPQ = \angle MQR = \pi/3$.

Therefore the sides of $\triangle PQR$ are also in the ratio 1-2-$\sqrt{3}$, with $PQ/PR = 1/2$ and so $PQ = OP = OQ$.  Thus, $\triangle OPQ$ is equilateral.

$QM \perp OP$ so $MQ$ is the bisector of both $\angle PQO$ and the base $OP$.  

Therefore, $OM$ is one-half of $OP$ and has a length of $1/2$.

$\square$

\subsection*{first parametrization}

We have just shown that the altitude of the inscribed equilateral triangle in a unit circle has length $3/2$.  This means that the ratio of the inscribed triangle to the standard one is $\sqrt{3}/2$.

And that means the side length of the inscribed equilateral triangle is $\sqrt{3}$.

That explains the length chosen for the chord in this problem.  We see that if $M$ is chosen at random anywhere along $OP$, one-half of the time the chord formed will be larger than $\sqrt{3}$.

\begin{center} \includegraphics [scale=0.5] {Bertrand2.png} \end{center}

So the probability we are asked to give is just $1/2$.

\subsection*{second parametrization}

\begin{center} \includegraphics [scale=0.5] {Bertrand3.png} \end{center}

The second parametrization has the same triangle we just saw, $\triangle PQR$.  The angle at vertex $R$ is $\theta$.

$\theta$ can lie in the interval $[0, \pi/2]$, and in the event that $\theta < \pi/6$, the chord length $RQ > \sqrt{3}$.

The probability that the chord is greater than $\sqrt{3}$ in length is $1/3$, since $\pi/6$ is one-third of $\pi/2$.

\subsection*{third parametrization}

Finally, we imagine picking two coordinates $(x,y)$ at random from the interior of the circle.  We place the midpoint of the chord $M$ at $(x,y)$.

If $M$ is such that $r = \sqrt{x^2 + y^2} < 1/2$, then $M$ will be closer to the center of the circle than $1/2$ and so the chord length will be $> \sqrt{3}$.

The number of points that have this property is proportional to the relative areas of the inside small circle, and the larger circle around is.

\[ \frac{\pi (1/2)^2}{\pi} = \frac{1}{4} \]

We see that, depending on which parameter is randomized, we obtain a probability of $1/2$, $1/3$ or $1/4$.

In Jaynes's words, the problem is not well-formed.

\subsection*{problem}

\begin{center} \includegraphics [scale=0.4] {area_prob2.png} \end{center}
We have a square with side $a$ and an equilateral triangle, also with side $a$.  Their bases are colinear and a red line is drawn as shown.  What is the area of the blue triangle?

The angles marked $\theta$ are equal because the base and the top of the square are parallel, but the angle is \emph{not} $30$.  The red line intersects the side of the square at half the height, so $a/2$.

We compute \emph{twice} the area of the combined blue and part of the equilateral triangle as $a^2/2$ and that of the blue triangle as $xa/2$.  But what is $x$?

We need a preliminary result.  The Pythagorean theorem allows us to calculate that the altitude of an equilateral triangle is $\sqrt{3}$ times one-half the base.
\begin{center} \includegraphics [scale=0.2] {equi.png} \end{center}

Back to our problem.  The triangle with $x$ as its base does have an angle of $30$ so its height $h$ is $\sqrt{3} \cdot x$.  We compute twice the area of the part of the equilateral triangle as $\sqrt{3} xa$.
\begin{center} \includegraphics [scale=0.4] {area_prob2.png} \end{center}

Put the whole thing together as
\[ 2\mathcal{A} = \frac{a^2}{2} = \frac{xa}{2} + \sqrt{3} xa \]
\[ a = x + 2 \sqrt{3} x \]
\[ x = \frac{a}{1 + 2 \sqrt{3}} \]

The final result is $1/2$ times $a/2$ times what we have above.

\subsection*{problem}

Here is a problem from the entrance exam for Brown, 1901.

\begin{center} \includegraphics [scale=0.4] {Brown1901_4.png} \end{center}

A regular triangle is what we would call an equilateral triangle.  Applying the Pythagorean theorem to one-half of such a triangle yields 1-2-$\sqrt{3}$ for the sides, where $2$ is the hypotenuse.

\begin{center} \includegraphics [scale=0.5] {Brown1901_4b.png} \end{center}

From this we deduce that the altitude is $3r$ and the side length is $2 \sqrt{3} r$, so the triangle's area is 
\[ A = \frac{1}{2} \cdot 3 \cdot 2 \sqrt{3} \cdot r^2 \]

The red part is the difference between this and the circle's area.

\[ \text{red} = (3 \sqrt{3} - \pi) r^2 \]

We're given a radius of $7$ but that's arithmetic.  We don't need that now.

\subsection*{problem}

Here is a problem from the entrance exam for Harvard, 1901.

\begin{center} \includegraphics [scale=0.4] {harvard1901_6.png} \end{center}

The first thing to notice is that the circle is tangent to the sides.  The tangent touches the circle at a single point, so the missing part of the diagram looks like this:

\begin{center} \includegraphics [scale=0.6] {harvard1901_6a.png} \end{center}

The whole angle at the top is $\pi/3$ so the half-angle is $\pi/6$.  This right triangle has sides in the ratio 1-2-$\sqrt{3}$.  The radius is given as $1$ so the red base side of the triangle is $\sqrt{3}$.  

The missing perimeter of the triangle is twice this or $2 \sqrt{3}$.  But the added perimeter from the circle is $2/3 \cdot 2 \pi r = 4/3 \pi$.

The area of the triangle is $1/2 \cdot \sqrt{3} \cdot 1 = \sqrt{3}/2$ and there are two copies so the total missing triangular area is $\sqrt{3}$.  The added circular area is $2/3 \pi$. 

The original perimeter of the triangle was $7 \cdot 3 = 21$ and the area was $\sqrt{3}/4 \cdot s^2$.

We'll just set up the two calculations:

\[ A = \frac{\sqrt{3}}{4} \cdot 7^2 - \sqrt{3} + 2/3 \cdot \pi \]
\[ P = 7 \cdot 3 - 2 \sqrt{3} + 4/3 \cdot \pi \]

It seems that the most efficient way to calculate this is to do $ \sqrt{3} + 2/3 \cdot \pi$, then double it, then do the rest.


\end{document}