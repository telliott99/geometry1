\documentclass[11pt, oneside]{article} 
\usepackage{geometry}
\geometry{letterpaper} 
\usepackage{graphicx}
	
\usepackage{amssymb}
\usepackage{amsmath}
\usepackage{parskip}
\usepackage{color}
\usepackage{hyperref}

\graphicspath{{/Users/telliott/Github-Math/figures/}}
% \begin{center} \includegraphics [scale=0.4] {gauss3.png} \end{center}

\title{Pythagorean Theorem}
\date{}

\begin{document}
\maketitle
\Large

%[my-super-duper-separator]

\label{sec:Euclid_I_47}

My favorite proof of the Pythagorean theorem relies on the construction below, sometimes called the ``bridal chair" or the ``windmill", where the central triangle is a right triangle, and the other figures are squares (Euclid I.47).  

\begin{center} \includegraphics [scale=0.3] {pythagoras2.png} \end{center}

We have the squares on three sides of a right triangle.

What we will show is that the rectangular area which is part of the large square, in red, is equal in area to the entire small square, in maroon.

\subsection*{preliminary}

We begin by restating a fundamental idea about the area of triangles, which is that, if two triangles have the same base and the same height, they have the same area.  So if we imagine sliding the top vertex of a triangle along a line parallel to the base, the area will not change.

\begin{center} \includegraphics [scale=0.5] {pyth11.png} \end{center}

\emph{Proof}.

The next figure shows this principle as it comes up in the proof.  The gray triangle with the black base is one-half the area of the small square.

\begin{center} \includegraphics [scale=0.3] {pyth12.png} \end{center}

Slide the vertex down to the right and the resulting triangle will still have the same area.

We can do the same thing with a triangle in the large square, below.  

The gray triangle has half the area of the part of the large square that is to the left of the dotted line, because its base is equal to the side of the square and the height extends to the right to the dotted line.  

\begin{center} \includegraphics [scale=0.3] {pyth13.png} \end{center}

Now slide the vertex up.  The area is unchanged.

And now, finally, we observe that the two triangles have exactly the same shape.  Just rotate one to obtain the other.

\begin{center} \includegraphics [scale=0.4] {pyth14.png} \end{center}

This completes our informal proof.  

$\square$

The formal approach follows, based on triangle congruence.

\subsection*{main}

We label some points as shown:
\begin{center} \includegraphics [scale=0.45] {pythagoras3.png} \end{center}
   
First, drop a vertical line $EHG$, constructing the rectangle $AFGH$.
   
Finally, sketch dotted lines for the long sides of two triangles:
\begin{center} \includegraphics [scale=0.4] {pythagoras4.png} \end{center}

\emph{Proof}.

The crucial point is this:  we will show that triangle $\Delta ABC$ is congruent to triangle $\Delta AEF$.  

Use side-angle-side (SAS).  The two sets of sides are evidently equal 
\[ AB = AE \]
\[ AF = AC \]

because, for example $AB$ and $AE$ are given as the sides of one square, and $AC$ and $AF$ as sides of another.

What about the included angle?  The angles $\angle BAC$ and $\angle EAF$ each contain a right angle plus the shared angle $\angle EAC$.

\[ \angle BAC = \text{ right angle } + \angle EAC \]
\[ \angle EAF = \angle EAC + \text{ right angle } \]
\[ \angle BAC = \angle EAF \]


So they are themselves equal, and thus we have proved SAS and thus the congruence relationship:
\[ \triangle ABC = \triangle AEF \]

\begin{center} \includegraphics [scale=0.4] {pyth15.png} \end{center}

The next part of the proof is to move the vertex of triangle $\Delta ABC$ to the left and see that it has base $AB$ and altitude $AE$ so its area is one-half that of the small square $ABDE$.  On the other hand triangle $\Delta AEF$ has base $AF$ and altitude $AH$ so its area is one-half that of the rectangle $AFGH$.

Hence we have proved that the two colored areas in this figure are equal:

\begin{center} \includegraphics [scale=0.35] {pythagoras2.png} \end{center}
Finally, we could proceed to do the same thing on the right side of the figure, but we just appeal to symmetry.  All the equivalent relationships will hold.  

$\square$

\subsection*{Quorra's corollary}

\label{sec:Quorra}

\begin{center} \includegraphics [scale=0.5] {pyth_corollary2.png} \end{center}

Let $\triangle ABC$ be \emph{any} triangle (here it is obtuse).  Draw $CM$ and $CN$ so that the new angles $\angle CMB$ and $\angle CNA$ (labeled $\phi$), are equal to $\angle C$.  The corresponding triangles are similar to the original, because they share $\phi$ plus one other from the original triangle.

We use single letters for the sides to make the algebra simpler.  $a$ is opposite both $\angle A$ and $\angle CMB$, while $b$ is opposite both $\angle B$ and $\angle CNA$, and $c$ is opposite $\angle C$.

The shortest side in $\triangle CMB$ is $x$ and the longest is $a$, while in $\triangle ABC$ the corresponding sides are $a$ and $c$.  So by equal ratios of sides in similar triangles we have that $x/a = a/c$.  The middle side in $\triangle CNA$ is $y$ and by similar logic we have that $y/b = b/c$.

\[ a^2 = cx, \ \ \ \ \ \ b^2 = cy \]
\[ a^2 + b^2 = c(x + y) \]

The sum of the squares of the two short sides of a triangle is equal to the product of the third side with the the sum of the two components $x + y$, when they are drawn with the angle $\phi$ as specified.
\begin{center} \includegraphics [scale=0.5] {pyth_corollary2.png} \end{center}
This is actually a generalization of our original algebraic proof of the Pythagorean theorem.

In the case where the angle at vertex $C$ is a right angle, then $M$ coincides with $N$, because there is only one vertical to a line from a given point.  So then and $x + y = c$, and this reduces to the Pythagorean theorem.

There are a large number of proofs of the Pythagorean theorem.  Many of them are collected or linked here:

\url{https://www.cut-the-knot.org/pythagoras/}

\end{document}