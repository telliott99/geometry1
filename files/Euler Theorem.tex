\documentclass[11pt, oneside]{article} 
\usepackage{geometry}
\geometry{letterpaper} 
\usepackage{graphicx}
	
\usepackage{amssymb}
\usepackage{amsmath}
\usepackage{parskip}
\usepackage{color}
\usepackage{hyperref}

\graphicspath{{/Users/telliott/Github-math/figures/}}
% \begin{center} \includegraphics [scale=0.4] {gauss3.png} \end{center}

\title{Introduction}
\date{}

\begin{document}
\maketitle
\Large

%[my-super-duper-separator]

In this chapter we talk about the famous Euler Theorem (the one in geometry, there are others).

Consider $\triangle ABC$ with its circumcircle on center $O$ with radius $R$, and its incircle on center $I$ with radius $r$.

Let the distance between $O$ and $I$ be $d$. 

Then the distance from I to the outer circle is $R+d$ in one direction and $R-d$ in the other direction.  The product of parts of the chord is
\[ (R+d)(R-d) = R^2 - d^2 \]

\begin{center} \includegraphics [scale=0.16] {EulerThm.png} \end{center}

We consider another chord drawn through $I$, namely the bisector of $\angle B$, extended to meet the circumcircle at $P$.

Let the bisected angles be $2 \beta = \angle B$ and $2 \gamma = \angle C$, as usual.

Draw the diameter $POQ$.  By Thales' theorem, $\triangle PQC$ is right.

Let $IZ \perp BZA$, so $\triangle IBZ$ is right.

By inscribed angles, $\angle PQC = \angle PBC = \beta$.

Thus, the two right triangles are similar, since they both contain $\beta$.

Form the ratios:
\[ \frac{IZ}{IB} = \frac{PC}{PQ} \]
Substituting
\[ \frac{r}{IB} = \frac{PC}{2R} \]
\[ 2rR = PC \cdot IB \]

If we can show that $PC = PI$, we will have that the right-hand side of the previous equation is equal to $PI \cdot IB = R^2 - d^2$, by crossed chords.

We consider $\triangle PCI$.

\begin{center} \includegraphics [scale=0.16] {EulerThm.png} \end{center}

$\angle PCI$ is equal to $\gamma + \beta$ by inscribed angles.

$\angle CPI$ is equal to $\angle A$.

By sum of angles, $\angle PIC$ is also equal to $\gamma + \beta$.

With equal base angles, by I.6 $\triangle PCI$ is isosceles.

Thus $PI = PC$, and the main result follows:
\[ 2rR = R^2 - d^2 \]
\[ = (R+d)(R-d) \]

\[ \frac{1}{r} = \frac{2R}{(R+d)(R-d)} \]
\[ = \frac{1}{R+d} + \frac{1}{R-d} \]



\end{document}