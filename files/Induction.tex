\documentclass[11pt, oneside]{article} 
\usepackage{geometry}
\geometry{letterpaper} 
\usepackage{graphicx}
	
\usepackage{amssymb}
\usepackage{amsmath}
\usepackage{parskip}
\usepackage{color}
\usepackage{hyperref}

\graphicspath{{/Users/telliott/Github-Math/figures/}}

\title{Induction}
\date{}

\begin{document}
\maketitle
\Large

%[my-super-duper-separator]

\subsection*{induction in geometry}

\label{sec:induction_chapter}

In the figure below is a polygon---an irregular heptagon.  Actually, there are three polygons altogether, there is the heptagon with $n+1$ sides, the hexagon with only $n$ sides that would result from cutting along the dotted line, and the triangle that is cut off.

We want to find a formula for the sum of the internal angles that depends only on the number of sides or vertices.

\begin{center} \includegraphics [scale=0.4] {polygon.png} \end{center}

The first part of the answer is to guess. 

\begin{center} \includegraphics [scale=0.35] {polygon2.png} \end{center}

We know that for a triangle ($n = 3$), the sum of the angles is $180^\circ$, and the sum does not depend on whether the triangle is acute, right or obtuse.  

Continuing with the square ($n = 4$), we can draw the diagonal and observe that the sum of all the angles is twice  $180^\circ$ or  $360^\circ$.  The partition into two triangles can be carried out with any quadrilateral, it does not require any sides being equal.

From this we guess that the formula may be:
\[ S_n = (n - 2) \cdot 180 \]

And indeed, in going from $n=4$ to $n=5$ sides we can think of the pentagon as being a quadrilateral with an extra triangle.  

And in the first figure, you can see that by adding the extra vertex to go to the $n+1$-gon, we added a triangle, or perhaps you'd rather say than in going from $n+1$ to $n$ we lost a triangle.  

In all cases, the difference between $n$ and $n+1$ is $180^\circ$.

The formula \emph{seems} to work.

We can use induction to \emph{prove} that it is correct.

The proof has two parts.  We must verify the formula for a base case like the triangle, which we've done.  You may wish to check that it works for the square as well, but that's not strictly necessary.

\emph{Proof}.

The second part of the proof is to verify that in going from $n$ to $n+1$, we add another $180^\circ$.  
The formula for $n$ sides is $(n-2)180^\circ$, adding another triangle gives:
\[ (n-2)180^\circ + 180^\circ \]
That must be equal to what the formula gives for $n+1$ sides:
\[ ((n+1)-2)180^\circ \]

Substituting $x$ for $180^\circ$ and equating the two, we have
\[ (n-2)x + x = ((n+1)-2) x \]
\[ n - 2 + 1 = n + 1 - 2 \]
\[ n = n \]

which is certainly correct.

$\square$

\subsection*{squares}

\label{squares_formula}

``Consider the number 1 represented by a dot.  Round this place three other dots so that the four dots form a square ($1 + 3 = 2^2$)."

\begin{center} \includegraphics [scale=0.45] {squares.png} \end{center}

We see that the sum of the first $n$ odd numbers is equal to $n^2$.  

\[ 1 = 1^2 \]
\[ 1 + 3 = 2^2 \]
\[ 1 + 3 + 5 = 3^2 \]
\[ 1 + 3 + 5 + 7 = 4^2 \]

We would like to find a relationship between the last value on the left-hand side and the number that is squared on the right.  Listing a few more terms we have a correspondence between

\begin{verbatim}
 1  3  5  7  9 11 13 ... 
 1  2  3  4  5  6  7 ...
\end{verbatim}

It appears that the first number grows like twice the second, just one less.  So we guess this formula

\[ 1 + 3 + 5 + \dots + (2n - 1) = n^2 \]

and confirm it works for $n = 1, 2, 3 \dots$.  

\emph{Proof}.

Now, compute 
\[ (n + 1)^2 = n^2 + 2n + 1 \]

So if we add $2n + 1$ to both sides above we have
\[ 1 + 3 + 5 + \dots + (2n - 1) + (2n + 1) = n^2 + 2n + 1 = (n + 1)^2 \]

and then notice that
\[ 2n + 1 = 2(n+1) - 1 \]

We have proved that substituting $n+1$ for $n$ in the original formula maintains the equality.  In other words, assuming the formula is correct for $n$, we have shown that it works for $n+1$.

This is the inductive step, and together with our tests of the base case, it proves the formula is correct for all $n$.

$\square$

\subsection*{Towers of Hanoi}

This example has a tenuous connection to geometry but it is a very clear example of induction and why it works as a method of proof.

\begin{center} \includegraphics [scale=0.3] {towers.png} \end{center}

In this famous game the goal is to move a set of disks from one peg to another.  Let us choose the one on the right as the target.

\url{https://en.wikipedia.org/wiki/Tower_of_Hanoi}

The rules are:

$\bullet$ \ Only one disk may be moved at a time.

$\bullet$ \ Each move consists of taking the upper disk from one of the pegs and sliding it onto another peg, on top of the other disks that may already be present on that peg.

$\bullet$ \ No disk may be placed on top of a smaller disk.

Here is an intermediate stage of the game:
\begin{center} \includegraphics [scale=0.3] {towers2.png} \end{center}

The next move is to place the blue disk on the middle peg.  I think you can take it from there.

We can solve the puzzle for any number of disks $n$.

\emph{Proof}.

By induction.  

Start from the first position:
\begin{center} \includegraphics [scale=0.3] {towers.png} \end{center}

Suppose we know how to move $n-1$ disks from one peg to another.  Move them to the middle peg, then move the nth disk to the right peg, then place all the $n-1$ disks on top.  We have moved $n$ disks.

The base case is to move the single light blue disk.  That's trivial.  The only thing to watch is if the number of disks is even or odd.  If even, choose peg 2, otherwise peg 3.

$\square$

Which peg is to be moved at each stage is shown in this graphic:
\begin{center} \includegraphics [scale=0.5] {towers3.png} \end{center}

You can see the structure clearly.  The first 7 steps are required to move the 3rd (blue) disk and all those above to peg no. 2.  Then the green disk is placed in final position.  Finally the same 7 steps are used to move the blue disk and all those above to peg no. 3, completing the puzzle.

\begin{quote}The puzzle was invented by the French mathematician Édouard Lucas in 1883. There is a legend about a Vietnamese temple which contains a large room with three time-worn posts in it surrounded by 64 golden disks. The monks of Hanoi, acting out the command of an ancient prophecy, have been moving these disks, in accordance with the rules of the puzzle, since that time. The puzzle is therefore also known as the Tower of Brahma puzzle. According to the legend, when the last move of the puzzle is completed, the world will end.\end{quote}

\subsection*{summary}

We can visualize an inductive proof as a kind of chain.  We show that the base case is true, for some value of $n$.  Then we show that if the formula works for $n$ , it must work for $n+1$.

\begin{quote}Mathematical induction proves that we can climb as high as we like on a ladder, by proving that we can climb onto the bottom rung (the basis) and that from each rung we can climb up to the next one (the step).\end{quote}

- Graham, Knuth and Patashnik

Below is a graphic from Tom Apostol's famous calculus text.  Its meaning should be obvious at this point.

\begin{center} \includegraphics [scale=1.0] {Apostol_induction.png} \end{center}

\subsection*{Binet's theorem}
In the chapter on the regular \hyperref[sec:pentagons]{\textbf{pentagon}}, we had the following:
\[ F_n = \frac{1}{\sqrt{5}} \cdot (\phi^n - \psi^n) \]

This is called Binet's formula, which gives the nth Fibonacci number.

We can prove this formula by induction.  First, the  base case.  As we said $\phi - \psi = \sqrt{5}$ so $F_1 = 1$, which is certainly correct.  

Since we need both $F_{n-2}$ and $F_{n-1}$ to compute $F_n$ using the recurrence, we should do the next one also as part of the base case.  Namely

\[ \frac{1}{\sqrt{5}} \cdot (\phi^2 - \psi^2) \]
\[ = \frac{1}{\sqrt{5}} \cdot \ [ \ 1 + \phi - (1 + \psi) \] 
\[ =  \frac{1}{\sqrt{5}} \cdot (\phi - \psi) \]
\[ = 1 \]

So then we assume that the following two equations are correct
\[ F_{n-2} = \frac{1}{\sqrt{5}} \cdot (\phi^{n-2} - \psi^{n-2}) \]
\[ F_{n-1} = \frac{1}{\sqrt{5}} \cdot (\phi^{n-1} - \psi^{n-1}) \]
and now, compute $F_n$:

\[ F_{n-2} + F_{n-1} = \frac{1}{\sqrt{5}} \cdot (\phi^{n-2} + \phi^{n-1} -  \psi^{n-2} - \psi^{n-1}  ) \]

This will match the formula, provided that
\[ \phi^{n-2} + \phi^{n-1} = \phi^n \]
Factoring, we obtain
\[ \phi^n = \phi^2(\phi^{n-2}) \]
\[ = (1 + \phi)(\phi^{n-2}) \]
\[ = \phi^{n-2} + \phi^{n-1} \]

And the same applies to $\psi$.

This completes the proof.  $\square$

\subsection*{de Moivre}
\label{sec:de_Moivre_theorem}

We want to mention a famous theorem by deMoivre and its proof.  It is possibly too early to do this, but we mentioned this theorem in a few places and the proof is a pretty straightforward example of induction.  The theorem states
\[ (\cos x + i \sin x)^n = \cos nx + i \sin nx \]
for integer $n$.

Probably you've seen $i$ before as a symbol for $\sqrt{-1}$.  For example, Euler has a famous formula
\[ e^{ix} = \cos x + i \sin x \]

If you think of $i$ as just being a number that when squared gives $-1$, that will be enough.  We can't really say too much to justify it here.  

There is one more rule that's important, however.  If we have an equation where some terms have a cofactor of $i$ and some do not, those without are called \emph{real} and those with are called \emph{imaginary}.  The important (and powerful) rule is that both the real and the imaginary parts of an equation must be equal.

Here is an example.  Let us multiply
\[ e^{ix} \cdot e^{iy} = e^{i(x+y)} \]
But if we return to Euler's formula, the left-hand side is
\[ (\cos x + i \sin x)(\cos y + i \sin y) = \]
\[ =  \cos x \cos y + i(\sin x \cos y + \sin y \cos x) - \sin x \sin y \]
And if that starts to look familiar, it should.  The right-hand side is
\[ e^{i(x+y)} = \cos (x+y) + i \sin (x+y) \]

We recall the rule that the real and imaginary parts must be equal separately.  That is:
\[ \cos (x+y) = \cos x \cos y - \sin x \sin y \]
\[ i \sin(x+y) =  i(\sin x \cos y + \sin y \cos x) \]
These are the sum of angles formulas, delivered by some alchemy.

Now, consider exponentiation.  Clearly
\[ (e^{ix})^n = (\cos x + i \sin x)^n \]

But by the rules of exponents we should have
\[ e^{ixn} = \cos nx + i \sin nx \]
Putting the two parts together, we have de Moivre's theorem, as we said.
\[ (\cos x + i \sin x)^n = \cos nx + i \sin nx \]

An example that we have given elsewhere is the cube.  If we choose $n = 3$ and consider only the real part, we can find an expression for $\cos 3x$.  Expand the left-hand side:
\[ \cos^3 x + 3 \cos^2 x (i \sin x) + 3 \cos x (i \sin x)^2 + (i \sin x)^3 \]
Of course the first term is real, but the third term is also because $i^2 = -1$ so then
\[ \cos 3x  = \cos^3 x - 3 \cos x \sin^2 x \]
\[ = 4 \cos^3 x - 3 \cos x \]
which we obtained previously by using the sum of angles twice.  Now, to prove the theorem.

\emph{Proof}.  (by induction, for positive integer $n$).

Let $n = 1$ so then
\[ \cos x + i \sin x = \cos x + i \sin x \]
(Actually, $n = 0$ will work, and is legal, but no matter).

The inductive step.  We assume that
\[ (\cos x + i \sin x)^k = \cos kx + i \sin kx \]
So then
\[ (\cos x + i \sin x)^{k+1} = (\cos kx + i \sin kx)(\cos x + i \sin x) \]
Multiplying out the right-hand side
\[ = \cos kx \cos x + i \sin x \cos kx + i \sin kx \cos x - \sin kx \sin x \]
If we gather the real terms we have
\[ \cos kx \cos x - \sin kx \sin x = \cos (kx + x) = \cos (k+1)x \]
while
\[ i (\sin x \cos kx + \sin kx \cos x) = i \sin (kx + x) = i \sin (k+1) x \]
$\square$


\end{document}