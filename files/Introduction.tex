\documentclass[11pt, oneside]{article} 
\usepackage{geometry}
\geometry{letterpaper} 
\usepackage{graphicx}
	
\usepackage{amssymb}
\usepackage{amsmath}
\usepackage{parskip}
\usepackage{color}
\usepackage{hyperref}

\graphicspath{{/Users/telliott/Github-math/figures/}}
% \begin{center} \includegraphics [scale=0.4] {gauss3.png} \end{center}

\title{Introduction}
\date{}

\begin{document}
\maketitle
\Large

%[my-super-duper-separator]

The image below is a detail from a painting by Raphael entitled ``School of Athens", which was used as the front cover of a wonderful book annotating the Heath translation of Euclid's \emph{Elements}.

\begin{center} \includegraphics [scale=0.70] {school_of_athens.png} \end{center}

It took a genius to figure it out the first time, but it is within anyone's grasp to appreciate what they found.  I imagine myself looking over Archimedes' shoulder as he explains the steps of a proof to me.

Most scientists I've met loved geometry in school, as I did.  They like how visual it is, and they like clever simple proofs.  Geometry should be fun!

A central feature of this book is the relentless use of proof.  I emphasize the key insight for each, and have tried to make the proofs simple and as easy to follow as possible.  You will notice that we frequently provide multiple proofs (using different methods) for the same theorem.

I express my sincere thanks to the authors of my favorite books, which are listed in the references and mentioned at various places in the text.  Everything in here was appropriated from them in one way or another, and styled to my taste.  

I offer my profound thanks also to Eugene Colosimo, S.J.  He was among the best of a great group of teachers.

This book is the pdf in that repository linked below.  Most of the rest is the source files.  There are several other books there as well, if you go up one level in Github.

\url{https://github.com/telliott99/geometry}

Note:  the sites referred to by some urls in the text have disappeared from the web and the count of missing pages grows over time.  I have left those URLs in, as a kind of protest, and because it is impractical to police them, but also because they may still be useful in connection with the wayback machine.

\url{https://web.archive.org} 

\end{document}