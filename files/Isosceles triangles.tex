\documentclass[11pt, oneside]{article} 
\usepackage{geometry}
\geometry{letterpaper} 
\usepackage{graphicx}
	
\usepackage{amssymb}
\usepackage{amsmath}
\usepackage{parskip}
\usepackage{color}
\usepackage{hyperref}

\graphicspath{{/Users/telliott/Github-Math/figures/}}

\title{Isosceles triangles}
\date{}

\begin{document}
\maketitle
\Large

%[my-super-duper-separator]

Triangles are classified by the largest angle they contain:  acute, right, or obtuse.  
\begin{center} \includegraphics [scale=0.2] {tri_types2.png} \end{center}

The acute triangle (left) has all three angles smaller than a right angle.  

The right triangle, naturally, has one right angle.

An obtuse triangle has one angle larger than a right angle (right panel, above).

\subsection*{symmetry}

One can also talk about the situation where either two sides, or all three sides, have the same length.  An \emph{isosceles} triangle has two sides the same length, while an \emph{equilateral} triangle has all three sides the same.

The most important consequence of all three sides equal for an equilateral triangle is three-fold rotational symmetry.  Three turns of $120$ degrees, and we're back where we started.  Each of the two intermediates is identical.

\begin{center} \includegraphics [scale=0.4] {equilateral.png} \end{center}

The implication of rotational symmetry is that the three angles are also equal because there is no reason to choose one larger than any other.  

Therefore each angle of an equilateral triangle is $2/3$ of a right angle, or $60^{\circ}$, by the triangle sum theorem (\hyperref[sec:triangle_sum_theorem]{\textbf{ref}}).

It is also true that if all three angles are equal, then the triangle is equilateral (three sides equal).  We will show how to prove this later.

\subsection*{notation}

The Greeks, including Euclid, always label points with letters, and line segments are referred to by the endpoints.  Angles and triangles are denoted by the line segments from which they are composed, as in $\angle ABC$, and triangles by their vertices:  $\triangle ABC$.

\begin{center} \includegraphics [scale=0.4] {triangle7.png} \end{center}

The Greek notation (left panel) is problematic because it is more complicated than need be. I have found myself repeatedly tracing out angles from three points.

Quite often, we will label the side opposite a vertex with a lower case letter:  side $a$ lies opposite $\angle A$.  We may also use letters like $\theta$ and $\phi$ for angles, or even, more boldly, $s$ and $t$.

Sometimes, we will dispense with labels altogether and use colored dots for equal angles.

The image below is from the web, it uses the convention of an arc drawn across equal angles.  The curved arcs are common in books.

\subsection*{theorem from Thales}

$\bullet$ \ If a triangle is isosceles (two sides equal), then the base angles are also equal.

The converse is

$\bullet$  \ If two base angles are equal, then the triangle is isosceles.

My favorite proof of both theorems about isosceles triangles is from reflective or mirror image symmetry.  

\begin{center} \includegraphics [scale=0.4] {isosceles.png} \end{center}

\emph{Proof}.

Imagine that the triangle sticks straight up from the plane like one of the standing stones at Stonehenge.  Think about walking around the triangle.

Looking from behind, it would appear exactly the same.  We would say that the left side as viewed from the front is equal to the right side as also viewed from the front, because if we walk around behind the triangle the \emph{right} side becomes the \emph{left} and vice-versa.

Much later than Euclid, Pappus invokes SAS on the mirror image, rather than thinking about the plane being in 3D space.

The top vertex angle is shared.  So the triangle as we look from the front is equal by SAS to the one where we look from the back.  It follows that the base angles are equal.

$\square$

\subsection*{proofs based on triangle bisection}

\label{sec:isosceles_triangle_theorem}

The argument above is not Euclid's proof, namely, I.5, the second theorem in his book.  We will not give that one right now, but instead something simpler.

The forward theorem on isosceles triangles is:

$\bullet$ \ Two sides equal $\Rightarrow$ opposite angles equal.

There is a simple demonstration based on angle bisection, and another based on the ability to cut a line segment into two equal parts, also called a bisection.  

\begin{center} \includegraphics [scale=0.16] {I_5_SAS.png} \end{center}

\emph{Demonstration}.

In the left panel, we are given $BA = BC$.

Draw the bisector of $\angle B$ (middle panel).  This construction forms equal angles at the top, marked with red dots.  So $\angle ABD = \angle CBD$.

$BD$ is shared (3 hash marks, right panel).

The two smaller triangles $\triangle ABD$ and $\triangle CBD$ are congruent by SAS.

We write that as $\triangle ABD \cong \triangle CBD$.

$\square$

Therefore (as corresponding parts of congruent triangles) the base angles are equal..  Other corresponding angles and sides are equal as well.

The full set of equal angles and sides is:
\begin{center} \includegraphics [scale=0.16] {I_5_result.png} \end{center}

The base is bisected, there is a right angle where the bisector $BD$ cuts the base, plus of course the central line segment $BD$ is equal to itself.

\subsection*{converse}

\label{sec:isosceles_converse}

We can also prove the converse theorem by a similar approach:

$\bullet$ \ Two angles equal $\Rightarrow$ opposite sides equal.

\begin{center} \includegraphics [scale=0.16] {I_6_AAS.png} \end{center}

\emph{Proof}.

We are given that the angles marked with black dots are equal. 

We again draw the bisector of the angle $B$.  Then, by sum of angles, we have all three angles the same, and the side $AD$ is shared.

Therefore, $\triangle ABD \cong \triangle ADC$ by AAS.

It follows that $BA = BC$.  

Since ADC lie on one straight line (they are \emph{collinear}), there are right angles at the base.

$\square$

Euclid's proof of the isosceles triangle theorem is more complicated than what we have given above, and there is a good reason for this.  

Our demonstration depends on the existence of the angle bisector, but we haven't actually shown how to do that.  It will turn out that \hyperref[sec:Euclid_I_9]{\textbf{construction}} of the bisector \emph{depends on} the isosceles triangle theorem.  

That's a problem because the reasoning is circular, thus invalid.  We cannot use $p$ to prove $q$ if we have previously used $q$ to prove $p$.  That proves nothing.  We will fix this problem later.

\subsection*{other proofs}

We used the angle bisector at vertex $B$ for the proofs above.  But we might equally have constructed a right angle at $D$, or bisected the base.  We will just show diagrams for proofs adapted to these methods and sketch the idea.

\begin{center} \includegraphics [scale=0.16] {I_5_SSS.png} \end{center}

Above, we are given $BA = BC$ and $AD = CD$, i.e. the base is bisected.  Since $BD$ is shared we have that $\triangle ABD \cong \triangle CBD$ by SSS.

\begin{center} \includegraphics [scale=0.16] {I_6_AASb.png} \end{center}

Above, we are given equal angles at $A$ and $C$ and right angles at $D$.  We have three angles equal plus a shared side, so congruent triangles by AAS (or after application of the triangle sum theorem, by ASA).

\begin{center} \includegraphics [scale=0.16] {I_5_HL.png} \end{center}
 
Above, we are given $BA = BC$ and right angles at $D$ (and $BCD$ \emph{colinear}).  This might be called SSA.

However, since the triangles are right triangles, this is a special case called hypotenuse-leg in a right triangle (HL).  The two small triangles are congruent.  For a proof, look ahead to Pythagoras's theorem:  if in a right triangle we know two sides, we know the third.

In general, if the known angle were not a right angle, this is not enough to show congruence.  However, there is much more to be said about this, as we will see.


\end{document}