\documentclass[11pt, oneside]{article} 
\usepackage{geometry}
\geometry{letterpaper} 
\usepackage{graphicx}
	
\usepackage{amssymb}
\usepackage{amsmath}
\usepackage{parskip}
\usepackage{color}
\usepackage{hyperref}

\graphicspath{{/Users/telliott/Github-math/figures/}}
% \begin{center} \includegraphics [scale=0.4] {gauss3.png} \end{center}

\title{Incenter, Orthocenter, Circumcenter}
\date{}

\begin{document}
\maketitle
\Large

%[my-super-duper-separator]

\label{sec:nine_point_circle}

We've seen that the perpendicular bisectors of the sides and, separately, the angle bisectors of triangles, converge on (are concurrent with) points that are the centers of circles with interesting properties.  These circles contain either the vertices of the triangle (circumcircle) or have the sides as tangents to the circle (incircle).

Now we investigate the altitudes and their point of convergence, the orthocenter.  It turns out there is a special circle, but it does not have the orthocenter as its center.

Quite surprisingly, there are nine points on the circle.  Three of these are midpoints of the sides.  This means that all four categories of special points of a triangle are connected to circles of one kind or another.

The circle that goes through the midpoints of the sides also goes through the points where the altitudes of a triangle meet the sides, as well as the midpoints of that part of each altitude lying between the orthocenter and the corresponding vertex.

It's challenging to draw the figure.  Some of the measurements may look a little off, but the logic will show that the circle indeed contains the nine points cited.

\begin{center} \includegraphics [scale=0.16] {ninepoint3.png} \end{center}

\emph{Proof}.

In $\triangle ABC$ draw the altitudes $AD$, $BE$ and $CF$.  Bisect the sides at midpoints $L$, $M$ and $N$.  Mark the orthocenter at $H$ and bisect $AH$ at $P$, $BH$ at $Q$ and $CH$ at $R$.

Because of the six bisections, we can find a number of similar triangles to show that certain lines in the diagram are parallel, and then involving the altitudes, find certain right angles.

For example, in $\triangle AHC$ we have that $P$ and $M$ are midpoints of two sides, so it follows that $\triangle APM \sim \triangle AHC$, and then $PM \parallel FHC$.

Looking at $\triangle BHC$, by the same logic we can show that $QL \parallel FHC$.  But this implies $PM \parallel QL$.

In exactly the same way, we can show that $PQ \parallel AB$ and $ML \parallel AB$ so $PQ \parallel ML$.  Thus, $PMLQ$ is a parallelogram.  

Since $AB \perp CF$, it follows that $ML \perp PM$.  Thus, $\angle PML$ is right.  It follows that $PMLQ$ is a rectangle.

Next, draw the circle on diameter $PL$ (the center is not shown, but can be found by bisecting $PL$).  $\angle PML$ and $\angle PQL$ are both right, which means that points $M$ and $Q$ lie on this circle with diameter $PL$.  And since $\angle ADL$ is right, $D$ also lies on the circle.  That's five of the nine points.

\begin{center} \includegraphics [scale=0.16] {ninepoint3.png} \end{center}

Since $Q$ and $M$ lie on the circle and $\angle QPM$ and $\angle QLM$ are right, it follows that $QM$ is also a diameter of the same circle.  And since $QEM$ is right, $E$ also lies on this circle.

\begin{center} \includegraphics [scale=0.32] {ninepoint4.png} \end{center}

A similar series of steps will show that $PRLN$ is a rectangle, which is enough to establish that $N$ and $R$ are on the circle, and that $NR$ is actually a diameter of the circle.  Since $\angle NFR$ is right, then $F$ also lies on the circle.

\begin{center} \includegraphics [scale=0.16] {ninepoint5.png} \end{center}

 We do not need it, but there is a third rectangle, namely, $NMRQ$.
 
$\square$

\subsection*{radius}

It turns out that the radius of the nine-point circle is $1/2$ the radius of the circumcircle of the same triangle.

\emph{Proof}.

Draw $\triangle PQR$.  

\begin{center} \includegraphics [scale=0.30] {ninepoint6.png} \end{center}

The nine-point circle \emph{is} the circumcircle of $\triangle PQR$.  We know a formula that connects the area $\mathcal{A}$ of a triangle with its side lengths and the radius of the circumcircle $R$:

\[ R = \frac{abc}{4 \mathcal{A}} \]

Let $\triangle ABC$ have side lengths $a$, $b$ and $c$ and area $\mathcal{A}$.  Let $\triangle PQR$ have side lengths $a'$, $b'$ and $c'$ and area $\mathcal{A}'$.

We can use similar triangles to show that each side and altitude in $\triangle ABC$ is twice that of $\triangle PQR$.  

For example, we will show that $AB = 2 PQ$.

Label the point $X$ where $HF$ cuts $PQ$.  $PXQ \parallel AB$, so $PXQ \perp CHF$, and $\triangle AFH \sim \triangle PXH$ with a ratio of $2$.  It follows that $2 PX = AF$ and $2 XH = FH$.

The latter equality gives us $\triangle BFH \sim \triangle QXH$, again with a ratio of $2$.  Thus $2 XQ = FB$ and $2 QH = BH$.  The result follows by addition.  In a similar way we can show that $2 HZ = HE$ so by addition the altitude $BE$ is also scaled by a factor of $2$.  

This means that the area is scaled by a factor of $4$.  The ratio of radii is:

\[ \frac{abc}{a'b'c'} \cdot \frac{\mathcal{A}'}{\mathcal{A}} \]

The first term is 8 and the second term is $1/4$ so the result is just a factor of 2.

\subsection*{center of nine point circle}

I find from reading wikipedia

\url{https://en.wikipedia.org/wiki/Nine-point_circle}

that the center of the nine point circle bisects the Euler line.  Also, that the customary designation for this point is $N$, so we should (perhaps) redraw all the figures with $K$,$L$ and $M$ as the midpoints of the sides.

Recall that the centroid $G$ lies one-third of the way along the Euler line from the circumcenter $O$ to the orthocenter $H$.  Then $N$ bisects $OH$ and $G$ lies one-sixth of that length away from $N$, on the side toward $O$.

\begin{center} \includegraphics [scale=0.30] {ninepoint7.png} \end{center}

I found a proof in Coxeter.

\emph{Proof}.

\begin{center} \includegraphics [scale=0.30] {ninepoint8.png} \end{center}

The side lengths of $\triangle PQR$ are exactly half those of $\triangle ABC$.

But the sides of $\triangle KLM$ are also exactly half those of $\triangle ABC$.

So the two smaller triangles are congruent.  They share the same circumcircle on center $N$, since all the vertices lie on the circle, which is the nine point circle for the parent $\triangle ABC$.  

The two triangles are related by rotation through a half-turn on center $N$.

Find the circumcenter of $\triangle ABC$ by dropping perpendiculars from points $K$, $L$ and $M$ to meet at $O$.  The same lines are the altitudes of $\triangle KLM$, since for example, $ML \parallel QR \parallel BC$.

The circumcenter of $\triangle ABC$, point $O$, is the same point as the orthocenter of $\triangle KLM$.  Rotation about $N$ converts $H$ into $O$ and vice-versa.  It follows that the distances are the same:  $HN = ON$.

$\square$

\subsection*{Feuerbach}

In 1822 Feuerbach showed that the nine-point circle is tangent to the three excircles of a triangle, as well as the incircle.

\begin{center} \includegraphics [scale=0.2] {Feuerbach.png} \end{center}


\end{document}