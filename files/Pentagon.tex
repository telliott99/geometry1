\documentclass[11pt, oneside]{article} 
\usepackage{geometry}
\geometry{letterpaper} 
\usepackage{graphicx}
	
\usepackage{amssymb}
\usepackage{amsmath}
\usepackage{parskip}
\usepackage{color}
\usepackage{hyperref}

\graphicspath{{/Users/telliott/Dropbox/Github-Math/figures/}}

\title{Pentagon}
\date{}

\begin{document}
\maketitle
\Large

%[my-super-duper-separator]

\subsection*{pentagons}
\label{sec:pentagons}

In this chapter we explore some properties of a regular pentagon.  A pentagon has 5 sides, and a regular polygon has all sides equal.

The question arises of how to draw a regular pentagon.  It's fairly difficult to do this freehand with a ruler, just guessing.  Constructions exist, as we'll see below.

Euclid deals with the question by constructing an isosceles triangle with the base angles twice the angle at the third vertex in IV.10, and in IV.11 he shows that such a triangle can be used to find five equally spaced points on a circle.

We follow a different approach.  For now, assume that we have done this.

\begin{center} \includegraphics [scale=0.16] {pent6.png} \end{center}
Consider the regular pentagon $ABCDE$, with diagonals also drawn internally.

Since $AB = AC$, by I.5 we have $\angle ABC = \angle ACB$.

It is a challenge to relate these two angles to the two marked with black dots at vertex $A$, below left.  How to justify the equality?  

We rely on rotational symmetry.  Each vertex must have the same component angles.  This gives us the four equal angles (black dots, left panel).  We could mark all the corresponding angles, such as $\angle DBE$, equal at this point.
\begin{center} \includegraphics [scale=0.16] {pent7.png} \end{center}

Symmetry also gives us the equal central angles labeled with red dots in the right panel.  Then by sum of angles we have:
\[ 2B + 3R = 4B + 1R \]
which quickly leads to $B = R$.

The assumption that one can construct a circle which circumscribes the regular pentagon makes the proof of equal angles trivial, using the inscribed angle theorem.

\begin{center} \includegraphics [scale=0.16] {pent9.png} \end{center}

One set of equal inscribed angles is marked.  We then use the fact that the chords are all equal to fill in the rest of the equalities.

Thus, all the triangles have angles that consist of multiples of the same angle.  Each of these small angles is $1/5$ of two right angles or $36^{\circ}$.
\begin{center} \includegraphics [scale=0.16] {pent8.png} \end{center}

By I.6 we have that $AD = AE$.  Again, by symmetry, all the internal diagonals are equal.

$ACPB$ has adjacent angles add to two right angles and opposite angles equal, so it is a parallelogram.  Therefore, it has opposite sides equal, but we are also given adjacent sides equal, so it is a rhombus.  There are five congruent rhombi in the figure.

By sum of angles, the central $PQRST$ is a pentagon, and by symmetry, it is a regular pentagon.  One can repeat the process of drawing the diagonals and generate new pentagons inside pentagons, forever.

By alternate interior angles we have $\angle ASR = \angle ADE = \angle AED = \angle ARS$.  So $\triangle ASR$ is isosceles and $\triangle ASR \sim \triangle ADE$.

There are two types of similar isosceles triangle in the figure:  tall skinny ones (36-72-72), and short fat ones (108-36-36).  The tall skinny ones come in three size classes, so for example $\triangle ASR \cong \triangle BTS$ and so on.

Here are three examples of the tall skinny triangle:
\begin{center} \includegraphics [scale=0.4] {three_triangles.png} \end{center}

When we work through the similar triangles using relationships between lengths (like those from the rhombus sides equal to pentagon sides), we'll see something very interesting.  But I think it may be useful to stop here and preview the answer, so we can get the arithmetic straight.

\subsection*{the golden ratio}

\begin{center} \includegraphics [scale=0.2] {golden_ratio2.png} \end{center}

We draw a square and then extend two parallel sides to make a large rectangle and a small one at the same time.  We don't want just any rectangles, but require that they be similar, they should have the same ratio of the long side to the short side.

We can conveniently model this in two ways.  In the first, the square is scaled to have side length $1$ and the extension is $x$, while in the second, the square has side length $x$ and the extension is $1$.  For the first, similarity gives:
\[ \frac{x+1}{1} = \frac{1}{x} \]
\[ x^2 + x = 1 \]
\[ x^2 + x - 1 = 0 \]

For the second one:
\[ \frac{x + 1}{x} = \frac{x}{1} \]
\[ x + 1 = x^2 \]
\[ x^2 - x - 1 = 0 \]

The first version has solutions:
\[ x = \frac{-1 \pm \ \sqrt{5}}{2} \]

The second version has solutions:
\[ x = \frac{1 \pm \ \sqrt{5}}{2} \]

The solutions are closely related.  It is traditional to pick the one that has $x > 1$.  We let $x$ be the long side of the rectangle and $1$ be the short side.

This is called $\phi$, the famous golden mean or ratio.  
\[ \phi = \frac{1 + \sqrt{5}}{2} \]

It has a value of about
\[ \frac{1 + \sqrt{5}}{2} \approx 1.65 \]

To return to our problem, we could work with the colored triangles and re-scale the figure by a factor of $x$.
\begin{center} \includegraphics [scale=0.4] {three_triangles_2.png} \end{center}

But there is an even nicer way, which comes from connecting adjacent sides of the pentagon.  In this case, we retain the original side lengths at $1$.

\begin{center} \includegraphics [scale=0.2] {pent13.png} \end{center}

Extend $AB$ and $ED$ to meet at $Y$.

Since the external angles in $\triangle YBD$ are equal, by supplementary angles $\angle YBD = \angle YDB$, so by I.6 $\triangle YBD$ is isosceles.  

Since $\triangle ADE$ has the same base angles and the same base, $\triangle ADE \cong \triangle ADB$.

Let the ratio $AD:DE = YD:DE$ be $x$.  If the sides have unit length ($DE = 1$) then by similar triangles
\[ \frac{DY}{BD} = \frac{EY}{AE} \]
\[ \frac{x}{1} = \frac{x+1}{x} \]
\[ x^2 = x + 1 \]

So $x$ is $\phi$.
The same ratio holds for any of the tall skinny triangles.

This solution is as we said before, and the equation below is the one to remember, with $\phi$ substituted for $x$:
\[ \phi^2 = 1 + \phi \]

We can check that $\phi$ really does solve the equation:
\[ \phi^2 =  \frac{1 + \sqrt{5}}{2} \cdot  \frac{1 + \sqrt{5}}{2} \]
\[ = \frac{1}{4} (1 + 2 \sqrt{5} + 5) \]
\[ = 1 + \frac{2 + 2 \sqrt{5}}{4} = 1 + \phi \]

One other thing we need can be obtained from the re-scaled figure.  Let the sides be $x$ (or $\phi$)::
\begin{center} \includegraphics [scale=0.4] {cos72.png} \end{center}

The right triangle with hypotenuse $x$ (or $\phi$) and base $1/2$ at the upper left, has base angle $72^{\circ}$ because it contains two copies of the fundamental angle.

The cosine of this angle is 
\[ \cos 72^{\circ}  = \frac{1/2}{\phi}  = \frac{1}{2 \phi} \]

One can also see this in the other figure:
\begin{center} \includegraphics [scale=0.2] {pent14.png} \end{center}

$\angle YBD = 72^{\circ}$.

Its cosine is one-half of $BD$, that is, $1/2$, divided by $\phi$.  We will come back for this.

\subsection*{more}

There are occurrences of $\phi$ all over the regular pentagon (see the first reference below).

One can also get the golden ratio, $\phi$, from the short squat triangles.

\begin{center} \includegraphics [scale=0.25] {pent10.png} \end{center}
\[ \frac{x}{1 + x} = \frac{1}{x} \]
\[ x^2 = 1 + x \]
as before.

And another:

\begin{center} \includegraphics [scale=0.16] {pent11.png} \end{center}

Draw the perpendicular at any vertex, say $BZ$, so that $\angle ABZ$ is a right angle.

$\angle DBE$ is bisected, since the total angle at $B$ is $108^{\circ}$, so by subtraction $\angle DBZ = 18^{\circ}$, which is half of $\angle DBE = 36^{\circ}$.

We use the bisector theorem:
\[ \frac{EZ}{DZ} = \frac{BE}{BD} = \frac{1 + x}{x} = x = \phi \]
as shown immediately above.

We can extend adjacent sides in the previous figure.  Since $\angle ABZ$ is a right angle and $\angle DBZ$ is one-half of the bisected internal $\angle DBE$ (in $\triangle BDE$), it follows that $\angle DBY$ is the bisector of the external $\angle DBX$.\

\begin{center} \includegraphics [scale=0.30] {pent12.png} \end{center}

By the external bisector theorem:
\[ \frac{EY}{DY} = \frac{BE}{BD} = \phi \]

When two adjacent sides of a regular pentagon are extended, then the length of the side plus the extension is in proportion to the side length as $\phi$.

But this is also evident from the fact that $\triangle BYD$ is similar to $\triangle AYE$.

\subsection*{construction of circumscribed pentagon}

Joyce and Bogolmony both give this construction, due to Richmond.

\url{https://mathcs.clarku.edu/~djoyce/java/elements/bookIV/propIV11.html}

\url{https://www.cut-the-knot.org/pythagoras/RichmondPentagon.shtml}

Wikipedia gives a rearranged version of the same thing.

\begin{center} \includegraphics [scale=0.2] {Richmond2.png} \end{center}

In a circle on center $O$, draw the radius $OA$ and make $\angle AOP$ a right angle, with the second radius bisected at $P$.

Draw $AP$.  Now bisect the angle $APO$ to find $Q$ on $OA$.

Finally, draw the perpendicular to $OA$ at $Q$ and find where it cuts the circle at $B$.  $AB$ is one side of the pentagon circumscribed by this circle.

\emph{Proof}.
We first show that $OA/OQ = \sqrt{5}$.

Let the radius equal $2$.  Then $OP = 1$ and the Pythagorean theorem gives $AP = \sqrt{5}$.

From the bisector theorem
\[ \frac{OQ}{AQ} = \frac{OP}{AP} = \frac{1}{\sqrt{5}} \]
\[ AQ = \sqrt{5} \cdot OQ \]
The result follows easily.

\begin{center} \includegraphics [scale=0.2] {Richmond2.png} \end{center}

We also know that $OQ + AQ = 2$ so
\[ OQ + \sqrt{5} \cdot OQ = 2 \]
\[ (1 + \sqrt{5}) \cdot OQ = 2 \]
\[ OQ = \frac{2}{1 + \sqrt{5}} \]
\[ OQ = \frac{1}{\phi} \]

In other words, the ratio $OP:OQ = \phi$.

$\angle AOB$ has adjacent side $OQ = 1/\phi$ and the hypotenuse is $2$ so the cosine is $1/2 \phi$.

But this is the cosine of $72^\circ$, which means that $\angle AOB = 72^{\circ}$, and corresponds to one-fifth of the complete circle.  It is the central angle of a pentagon.

Here is a figure (redrawn from Bogolmony) which shows how Richmond's approach extends to finding all of the vertices of the regular pentagon (Bogolmony cites Conway and Guy).
\begin{center} \includegraphics [scale=0.2] {Richmond4.png} \end{center}

$APO$ is bisected by $PQ$.  $PR$ is drawn as the bisector of the external angle to $\angle APO$ (i.e. supplementary).  Two adjacent bisectors of supplementary angles together form a right angle.

Thus $PR$ is perpendicular to $PQ$ at $P$ and forms $\triangle RPQ$ as a right triangle.  $\triangle RPQ \sim \triangle ROP \sim \triangle POQ$.

By similar triangles we have that 
\[ \frac{OP}{OR} = \frac{OQ}{OP} \]
\[ OP^2 = OR \cdot OQ \]
\[ 1 = OR \cdot \frac{1}{\phi} \]
\[ OR = \phi \]

Bogolmony suggests that we recall the external angle bisector theorem.  The two sides $AP:OP$ are in the same ratio as $AR:OR = \sqrt{5}$.

So $AR = \sqrt{5} \cdot OR$.  Then
\[ AR = 2 + OR = \sqrt{5} \cdot OR \]
\[ 2 = (\sqrt{5} - 1) OR \]
\[ OR = \frac{2}{\sqrt{5} - 1} \]

Let's see:
\[ \phi = \frac{\sqrt{5} + 1}{2} \]
\[ \phi - 1 = \frac{\sqrt{5} - 1}{2} \]
\[ \frac{1}{\phi - 1} = \frac{2}{\sqrt{5} - 1} \]

So
\[ OR = \frac{1}{\phi - 1} \]

But
\[ \phi^2 = \phi + 1 \]
\[ 1 = \phi^2 - \phi  \]
\[ \frac{1}{\phi} = \phi - 1 \]

So $OR = \phi$.

But $OR$ is the adjacent side in the right triangle $COR$, with hypotenuse equal to $2$.  So the cosine of $\angle COR$ is $\phi/2$.

The angle whose cosine is $\phi/2$ is $36^{\circ}$.  Hence $\angle COD$ is twice that or $72^{\circ}$, which is the correct angle for the central angle of one section of the regular pentagon.

Finally, we need to show that $\cos 36^{\circ} = \phi/2$.

Here is one way:
\begin{center} \includegraphics [scale=0.4] {cos36b.png} \end{center}

In the figure above, the right triangle with hypotenuse $1$ and base $x/2$ at the upper left, has base angle $36^{\circ}$ because it contains a single copy of the fundamental angle.

The cosine of this angle is 
\[ \cos 36^{\circ}  = \frac{\phi}{2} \]

Algebraically, we may recall the half angle formula:
\[ \cos^2 A = \frac{1 + \cos 2A}{2} \]

We had $\cos 72^{\circ} = 1/(2 \phi)$
\[ \cos^2 36^{\circ} = \frac{1 + 1/(2 \phi)}{2} = \frac{2 \phi + 1}{4 \phi} \]

Now
\[ \phi^2 = \phi + 1 \]
\[ \phi^3 = \phi^2 + \phi = 2 \phi + 1 \]
Hence the numerator above is $\phi^3$ which gives
\[ \cos^2 36^{\circ} = \frac{\phi^2}{4} \]

And the result follows (I won't say easily).  The first method was a lot easier.

$\square$

\subsection*{Euclid's construction}

Euclid uses several steps to construct a pentagon circumscribed by a circle.  The critical chain of dependencies is II.11 $\Rightarrow$ IV.10 $\Rightarrow$ IV.11.  The latter two are pretty involved and we have seen a nice construction already.  We will just look at II.11.

\subsection*{II.11}

\begin{quote}To cut the line $AB$ at $H$ such that the rectangle contained by the whole and one of the segments ($HB$) is equal to the square on the other segment.\end{quote}

\begin{center} \includegraphics [scale=0.25] {Euclid_II_11.png} \end{center}

To prove:  $HD = FH$.

\emph{Geometric Proof}.

Draw the square on $AB$ so $AB = BD$.

Bisect $AC$ at $E$.  Draw $BE$.

Extend $EA$ so that $EF = BE$.  

Draw the square on $AF$.  $H$ is the desired point.

I claim $HB \cdot AB = AH^2$.
\[ \frac{AH}{HB} = \frac{AB}{AH} \]

\begin{center} \includegraphics [scale=0.25] {Euclid_II_11.png} \end{center}

In other words, the golden ratio or mean.

By II.6
\[ CF \cdot FA = EF^2 - AE^2 \]

Since $EF = BE$ and by I.47

\[ CF \cdot FA = BE^2 - AE^2 = AB^2 \]

The left-hand side is $FK$, and the right-hand side is $AD$.

Subtract $AK$ from each.  We obtain:
\[ FH = AH^2 = HD \]

as required.

$\square$

\begin{center} \includegraphics [scale=0.25] {Euclid_II_11.png} \end{center}

Algebraically, let $AB = x, AH = y, HB = x - y$, and then
\[ (x-y) \cdot x = y^2 \]
\[ x^2 - xy = y^2 \]
\[ x^2 = xy + y^2 \]

If $y = 1 = AH$ then
\[ x^2 = x + 1 \]
As advertised, $x = \phi = AB$, the golden ratio.

\subsection*{$\phi$ in the square}

Let's finish the chapter (and this volume) with one last example I found on Bogolmony's site.

\url{https://www.cut-the-knot.org/do_you_know/GoldenRatioInSquare.shtml}

It was ``contributed by Ercole Suppa (Italy) at the Peru Geometrico facebook group.''

\begin{center} \includegraphics [scale=0.2] {square_phi.png} \end{center}

One side of a square is bisected at $M$ and then the two incircles are drawn as shown.  Remarkably, the radii are in the ratio $R/r = \phi$.

\emph{Proof}.

Let the square $AC$ have sides of length $2$, so $MB = 1$ and $MC = \sqrt{5}$.  We find the semi-perimeters of the two triangles:
\[ s_{\triangle MBC}  = \frac{1}{2} \cdot (3+\sqrt{5}) \]
\[ s_{\triangle MDC} = \frac{1}{2} \cdot (2+2 \sqrt{5}) = 1+ \sqrt{5}\]

Then we find the areas as
\[ \mathcal{A}_{\triangle MBC} = rs_{\triangle MBC} = \frac{1}{2} \ r \ (3+\sqrt{5}) \]
\[ \mathcal{A}_{\triangle MDC} = Rs_{\triangle MDC} = R\ (1+\sqrt{5}) \]

\begin{center} \includegraphics [scale=0.2] {square_phi.png} \end{center}

Since the base of one triangle is twice the other, we have
\[ \mathcal{A}_{\triangle MDC} = 2A_{\triangle MBC} = r(3+\sqrt{5}) \]
Then
\[ R(1+\sqrt{5}) = r(3+\sqrt{5}) \]
\[ \frac{R}{r} = \frac{3+\sqrt{5}}{1+\sqrt{5}} \]
which certainly doesn't look like $\phi$, although it does have $\sqrt{5}$.

Let's just play around:
\[ \phi = \frac{1 + \sqrt{5}}{2} \]
\[ 2 \phi = 1 + \sqrt{5} \]
\[ 2 \phi + 2 = 3 + \sqrt{5} \]

Hence we have that 
\[  \frac{R}{r} = \frac{ 2 \phi + 2}{2 \phi} \]
\[ = 1 + \frac{1}{\phi} \]

And since
\[ \phi^2 = \phi + 1 \]
\[ \phi = 1 + \frac{1}{\phi} \]
It follows that 
\[ \frac{R}{r} = \phi \]

$\square$

That's amazing!

\end{document}