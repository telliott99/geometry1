\documentclass[11pt, oneside]{article} 
\usepackage{geometry}
\geometry{letterpaper} 
\usepackage{graphicx}
	
\usepackage{amssymb}
\usepackage{amsmath}
\usepackage{parskip}
\usepackage{color}
\usepackage{hyperref}

\graphicspath{{/Users/telliott/Github-math/figures/}}
% \begin{center} \includegraphics [scale=0.4] {gauss3.png} \end{center}

\title{Power of a point}
\date{}

\begin{document}
\maketitle
\Large

%[my-super-duper-separator]

We're going to look at how Euclid was able to establish the 

$\hyperref[sec:chord_segments]{\textbf{crossed chord theorem}}$ 

and $\hyperref[sec:secant_tangent_theorem]{\textbf{secant tangent theorem}}$.

The main purpose is to appreciate how much he was able to do (up to Book III of \emph{Elements}) without a theory of similar figures, using only the Pythagorean theorem.

But we start with two propositions from Book II, which deal with what might be termed geometric algebra.

\subsection*{Euclid II.5}

\label{sec:Euclid_II_5}

\begin{center} \includegraphics [scale=0.25] {gnomon1.png} \end{center}

In this figure, the line $AB$ is bisected at $M$ and then the point $P$ placed somewhere within $MB$.  The first proposition, II.5, says that
\[ AP \cdot PB + MP^2 = MB^2 \]

where $AP \cdot PB$ is the area of rectangle $AD$, $PE$ is the square on $PB$, with $PB = BE$, and $CF$ is the square on $CD$, with $CD = MP = EG$.

The construction contains two sets of equal rectangles.  The first is $AC = ME = BF$.  And the second is $MD = DG$.  This is by I.43.

In the figure below, the two white parallelograms are equal.
\begin{center} \includegraphics [scale=0.15] {EI_43.png} \end{center}

We saw an example very early for rectangles as the \hyperref[sec:area_ratio_theorem]{\textbf{area-atio theorem}}.
\begin{center} \includegraphics [scale=0.6] {Acheson_G42.png} \end{center}
\[ ad = bc \]

Returning to the present theorem
\begin{center} \includegraphics [scale=0.25] {gnomon1.png} \end{center}

As  noted, the construction contains two sets of equal rectangles.  The first is $AC = ME = PG$, while the second is $MD = DG$.

Here's the neat idea: $AC + MD + PE = AE$.

But this is equal to $MD + PE + DG$, since $AC = DG$.  The latter area is an L-shaped piece called a \emph{gnomon}.  

So then if we add $CF$ (equal to $MP^2$) we have $MG$ (equal to $MB^2$).

\[ AP \cdot PB + MP^2 = MB^2 \]

I find this more difficult to remember than I would like.  But it is easy to check using algebra.  Let $x = AM$ and $y = MP$ so $AP = x + y$ and $PB = x - y$ and then
\[ AP \cdot PB = x^2 - y^2 = MB^2 - MP^2 \]

\subsection*{Euclid II.6}

\label{sec:Euclid_II_6}

The second theorem is very similar but has the point $P$ located on an extension of $AB$:
\begin{center} \includegraphics [scale=0.20] {gnomon2.png} \end{center}

Again we have a gnomon and a difference of squares.  The gnomon is equal to the whole rectangle $AP \cdot PB$ and when added to $MB^2$ we get the whole large square $MP^2$
\[ AP \cdot PB + MB^2 = MP^2 \]

In the final formula, we have just switched $MB$ and $MP$, although $AP$ is not what it used to be.

In algebraic terms we let $AM = x$ and now $PB = y$ so
\[ (2x + y) y + x^2 = 2xy + y^2 + x^2 = (x + y)^2 \]
\[ AP \cdot PB + MB^2 = MP^2 \]

\subsection*{application to chords}
We have a chord of a circle on center $O$, $AB$.  $AB$ is bisected at $M$ and $P$ is placed somewhere within $MB$.
\begin{center} \includegraphics [scale=0.16] {EIII_35b.png} \end{center}

By II.5
\[ AP \cdot PB + MP^2 = MB^2 \]
\[ AP \cdot PB = MB^2 - MP^2 \]

We notice both terms on the right are part of right triangles.
\[ AP \cdot PB = (MB^2 + OM^2) - (MP^2 + OM^2) \]
\[ = r^2 - OP^2 \]

Let $d$ be the distance from $P$ to the center, and this becomes
\[ AP \cdot PB = r^2 - d^2 \]
This result is \emph{independent} of the particulars of $AB$ and depends only on the placement of $P$ (and the requirement that $AB$ pass through $P$).  So any other chord that also passes through $P$, say $CD$, has the same result.

\begin{center} \includegraphics [scale=0.15] {EIII_35c.png} \end{center}
\[ AP \cdot PB = CP \cdot PD \]
This, III.35, is just the crossed chord theorem in disguise.

\subsection*{tangent}

The next theorem concerns the tangent.
\begin{center} \includegraphics [scale=0.18] {EIII_36.png} \end{center}

Here we have $AB$ bisected at $M$, the center of the circle, and then $P$ placed on the extension of $AB$.

By II.6
\[ AP \cdot PB + MB^2 = MP^2 \]

$MB$ and $MT$ are radii so
\[ AP \cdot PB = MP^2  - MT^2 = PT^2 \]

The length of the tangent from point $P$ is equal $AP \cdot PB$.

The two results can be unified by considering $PA$ and $PB$ as directed line segments.  In the crossed-chords example, they point in opposite directions so their product is negative, while for tangent and secant they point in the same direction.

We can view the result for chords $r^2 - d^2$ as the same as the result for the tangent, $d^2 - r^2$, provided we take account of the sign of $AP \cdot PB$.

\subsection*{secant}

\begin{center} \includegraphics [scale=0.20] {EIII_36b.png} \end{center}

As with the tangent, by II.6 we have
\[ AP \cdot PB + MB^2 = MP^2 \]
and again, we have right triangles so
\[ AP \cdot PB + (MB^2 + OM^2) = (MP^2 + OM^2) \]
\[ AP \cdot PB + OB^2 = OP^2 \]

so again
\[ AP \cdot PB = d^2 - r^2 \]

By using the previous two results together, we have that for any secant drawn from $P$, $PA \cdot PB$ is equal to the square of the tangent from the same point $PT^2$.

This is the $\hyperref[sec:secant_tangent_theorem]{\textbf{secant tangent theorem}}$.



\end{document}