\documentclass[11pt, oneside]{article} 
\usepackage{geometry}
\geometry{letterpaper} 
\usepackage{graphicx}
	
\usepackage{amssymb}
\usepackage{amsmath}
\usepackage{parskip}
\usepackage{color}
\usepackage{hyperref}

\graphicspath{{/Users/telliott/Dropbox/Github-Math/figures/}}
% \begin{center} \includegraphics [scale=0.4] {gauss3.png} \end{center}

\title{Ptolemy's theorem}
\date{}

\begin{document}
\maketitle
\Large

%[my-super-duper-separator]

In a previous \hyperref[sec:Ptolemy]{\textbf{chapter}} we introduced Ptolemy's theorem.
\begin{center} 
\includegraphics [scale=0.12] {pt1b.png} 
\includegraphics [scale=0.13] {pt1c.png} 
\end{center}

In the left panel
\[ AB \cdot CD + BC \cdot AD = AC \cdot BD \]
or somewhat more compactly in the right panel:  $ac + bd = ef$.

Here we provide two more proofs of this theorem, as examples of wonderful proofs, and then explore some consequences.
\begin{center} \includegraphics [scale=0.4] {Ptolemy3.png} \end{center}

Above is a graphic from wikipedia that shows where we're going in the first proof.  We will form two sets of similar triangles and use our knowledge about corresponding ratios.

\url{https://en.wikipedia.org/wiki/Ptolemy%27s_theorem}

\subsection*{Ptolemy's theorem from similar triangles}

\label{sec:Ptolemy_similar_triangles}

\emph{Proof}.

\begin{center} \includegraphics [scale=0.14] {Ptolemy12b.png} \end{center}

Find $P$ on $AC$ such that $\angle ADP = \angle CDB$ (red dots).

Since $ABCD$ is a cyclic quadrilateral, we can find other equal angles (black and magenta dots).

We write the vertices in the same order as the equal angles.

$\triangle ADP \sim \triangle BDC$.

So
\[ \frac{AD}{BD} = \frac{DP}{CD} = \frac{AP}{BC} \]
\[ AD \cdot BC = AP \cdot BD \]

Since $\angle PDB$ is shared, $\angle ADB = \angle PDC$.

$\triangle PDC \sim \triangle ADB$.

So
\[ \frac{PD}{AD} = \frac{CD}{BD} = \frac{PC}{AB} \]
\[ AB \cdot CD = PC \cdot BD \]

\begin{center} \includegraphics [scale=0.14] {Ptolemy12b.png} \end{center}

Adding
\[ AB \cdot CD + AD \cdot BC = PC \cdot BD + AP \cdot BD \]
\[ = AC \cdot BD \]

$\square$

This is Ptolemy's theorem.

Yiu also proves the converse theorem.

The dots in this proof make it clear which two pairs of triangles are similar, and I've taken care to list the vertices of the triangles in the same order as the sides in each pair of similar triangles, from smallest to largest.

For an alternate notation see \hyperref[sec:Ptolemy_alt]{\textbf{here}}.

\subsection*{proof by switching sides}

\label{sec:Ptolemy_switch_sides}

(adapted from wikipedia).

\url{https://en.wikipedia.org/wiki/Ptolemy%27s_theorem}

In the proof below, we denote the area of polygons such as $ABCD$ as $(ABCD)$.

\emph{Proof}.

\begin{center} \includegraphics [scale=0.16] {pt2a.png} \end{center}
We add up (twice) the areas of the component triangles:

\[ 2(AMB) = qr \cdot \sin AMB \]
\[ 2(BMC) = qs \cdot \sin BMC \]
\[ 2(CMD) = ps \cdot \sin AMB \]
\[ 2(AMD) = pr \cdot \sin BMC \]

But $\sin AMB = \sin BMC$.  Hence twice the area of $(ABCD)$ is
\[ 2(ABCD) = (qr + qs + pr  + ps) \cdot \sin AMB \]
\[    = (q + p)(r + s) \cdot \sin AMB \]
\[    = AC \cdot BD \cdot \sin AMB \]
We're on to something.  Now, the great idea.  

Move the point $D$ to $D'$ such that $AD' = CD$ and $AD = CD'$.
\begin{center} \includegraphics [scale=0.16] {pt2b.png} \end{center}

When we do this, note two things.  First, $\angle CAD = \angle ACD'$ (black dots), because the arcs they intercept are equal, $AD' = CD$.

Second, $\triangle ADC \cong \triangle ACD'$ by SSS.  Therefore the area hasn't changed:  $(ABCD) = (ABCD')$.

Now compute (twice) the area of two component triangles.
\begin{center} \includegraphics [scale=0.16] {pt2c.png} \end{center}

\[ 2(ABD') = AB \cdot AD'  \cdot  \sin \angle BAD' \]
\[  = AB \cdot AD'  \cdot \sin \angle BCD' \]

\[ 2(CBD') = BC \cdot CD' \cdot \sin \angle BCD' \]
since the two angles are supplementary.

But $\angle BCD' = \angle ADC + \angle CAD$.  (Check out the dots).

So $\angle BCD' = \angle AMB$.  Thus, their sines are equal.  We have

\begin{center} \includegraphics [scale=0.16] {pt2c.png} \end{center}

\[ 2(ABD') + 2(CBD') = 2(ABCD) \]
\[ [ \ AB \cdot AD' + BC \cdot CD'  \ ] \ \sin \angle BCD' = AC \cdot BD \cdot \sin AMB \]
\[ AB \cdot AD' + BC \cdot CD' = AC \cdot BD \]

Finally, since $AD = CD'$ and $CD = AD'$:
\[ AB \cdot CD + BC \cdot AD = AC \cdot BD \]

$\square$

This is Ptolemy's theorem.

\textbf{corollaries}

Here are just a few of the results that follow from this remarkable theorem.

\subsection*{equilateral triangle}

\begin{center} \includegraphics [scale=0.2] {equi4.png} \end{center}

Inscribe an equilateral triangle in a circle and pick any point on the circle.

\[ ps = qs + rs \]
\[ p = q + r \]

We proved this earlier, without using Ptolemy's theorem, as Van Schooten's theorem.

Here's a different problem from basically the same diagram (Coxeter).

\begin{center} \includegraphics [scale=0.18] {equi5b.png} \end{center}
\[ \frac{1}{PA} + \frac{1}{PC} = \frac{1}{PQ} \]

Let us re-label with $PA = t$, $PC = u$ and $PQ = w$.  We will prove that:
\[  \frac{1}{t} + \frac{1}{u} = \frac{1}{w} \]

\begin{center} \includegraphics [scale=0.18] {equi5c.png} \end{center}

\emph{Proof}.

We have similar triangles:  
\[ \triangle AQB \sim \triangle PQC, \ \ \ \ \ \ \ \  \triangle AQP \sim \triangle BQC \]

Which gives the ratios
\[ \frac{v}{y} = \frac{s}{u} = \frac{x}{w}, \ \ \ \ \ \ \ \  \frac{w}{y} = \frac{t}{s} = \frac{x}{v} \]

Thus:
\[ \frac{s}{u} = \frac{x}{w}, \ \ \ \ \ \ \ \  \frac{s}{t} = \frac{y}{w} \]
Adding
\[ \frac{s}{u} + \frac{s}{t} = \frac{x + y}{w} = \frac{s}{w} \]
\[ \frac{1}{t} + \frac{1}{u} = \frac{1}{w} \]

$\square$

\subsection*{Pythagorean theorem}

Let the quadrilateral be a rectangle.  The the sum of squares of opposing sides is
\[ a^2 + b^2 \]

Triangles made by opposing diagonals are congruent, so the diagonals are equal in length.  The diagonal is the hypotenuse, hence
\[ a^2 + b^2 = c^2 \]

We saw this proof previously (\hyperref[sec:PProof_Ptolemy]{\textbf{here}}).

\subsection*{Law of Cosines}

\label{sec:LOC_by_Ptolemy}

Draw $\triangle ABC$ (suppress the $B$ label) and then draw another triangle congruent with it, with a shared base, and all four points in a circle, forming a cyclic quadrilateral.  

Relying on previous work with a rectangle in a circle (\hyperref[sec:rectangle_side_on_a_circle]{\textbf{here}}), we know this construction is possible.

The points are $A, A', C, C'$.  $\beta$ marks the original $\angle B$, but will not be used.

\begin{center} \includegraphics [scale=0.15] {law_of_cosines3.png} \end{center}

We need an expression for $x$.  We have that the base of the altitude from $C$ to side $c$ is a distance from $A$ equal to $(c-x)/2$.  It follows that
\[ \frac{c-x}{2} \div b = \cos A \]
\[ c - x = 2b \cos A \]
\[ x = c - 2b \cos A \]

Now, apply Ptolemy's Theorem.  We have:
\[ a^2 = b^2 + cx \]
\[ = b^2 + c(c - 2b \cos A) \]
\[ = b^2 + c^2 - 2bc \cos A \]

$\square$

\subsection*{golden mean in the pentagon}

\begin{center} \includegraphics [scale=0.3] {Ptolemy5.png} \end{center}

Take four vertices of the regular pentagon and draw two diagonals.  From the theorem, we have
\[ b \cdot b = a \cdot a + a \cdot b \]
\[ \frac{b^2}{a^2} = 1 + \frac{b}{a} \]

Rather than use the quadratic equation, rearrange and add $1/4$ to both sides to ``complete the square":
\[ \frac{b^2}{a^2} - \frac{b}{a} + \frac{1}{2^2} = 1 + \frac{1}{2^2} \]

So
\[ (\frac{b}{a} - \frac{1}{2})^2  = \frac{5}{4} \]
\[ \frac{b}{a} - \frac{1}{2}  = \pm \ \frac{\sqrt{5}}{2} \]
\[ \frac{b}{a}  = \frac{1 \pm \sqrt{5}}{2} \]

This ratio $b/a$ is known as $\phi$, the golden mean.

\subsection*{diagonals}

Let us look at something like what we used for the proof of Ptolemy's theorem in the beginning.

\begin{center} \includegraphics [scale=0.5] {pt6.png} \end{center}
\[ nm = ac + bd \]

We move one of the points, exchanging sides $a$ and $d$.  Then, one of the diameters, $n$, changes length to $u$.
\[ mu = ab + cd \]

If, instead, we exchange sides $a$ and $b$, the old $m$ changes to $u$.  Why?

\begin{center} \includegraphics [scale=0.5] {pt7.png} \end{center}

Mark the peripheral angles with equal arcs ($abcd$:  red, blue, green, magenta).

The triangle with sides $b$ and $d$ in the middle, and magenta plus red for the vertex angle, is congruent to one in the right panel.  So their long sides are equal, both have length $u$.

Thus,
\[ nu = ad + bc \]

We get a formula for the square of the diagonal:
\[ m^2 = \frac{(mu)(nm)}{nu} = \frac{(ab + cd)(ac + bd)}{(ad + bc)}  \]

There is a similar formula for $n^2$.  These formulas are sometimes attributed to Brahmagupta.  This beautiful proof is due to Paramesvara (14th century).

\url{https://www.cut-the-knot.org/proofs/PtolemyDiagonals.shtml}

The ratio is
\[ \frac{m}{n} = \frac{ab + cd}{ad + bc} \]

which is referred to as Ptolemy's second theorem.

\end{document}