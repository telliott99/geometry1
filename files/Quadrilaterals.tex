\documentclass[11pt, oneside]{article} 
\usepackage{geometry}
\geometry{letterpaper} 
\usepackage{graphicx}
	
\usepackage{amssymb}
\usepackage{amsmath}
\usepackage{parskip}
\usepackage{color}
\usepackage{hyperref}

\graphicspath{{/Users/telliott/Dropbox/Github-math/figures/}}
% \begin{center} \includegraphics [scale=0.4] {gauss3.png} \end{center}

\title{Quadrilaterals}
\date{}

\begin{document}
\maketitle
\Large

%[my-super-duper-separator]

\subsection*{classification}

Polygons are constructed from straight sides.  If the sides are all the same length, the figure is a \emph{regular} polygon.

A polygon may have $3$ or more sides:  triangles (3), hexagons (6), and so on.  There is a famous theorem from Gauss that involves the construction of a 17-sided regular polygon.

Let us start with four-sided polygons, called quadrilaterals.  There are several types, of which the most important are:

$\circ$ \ parallelogram: opposite sides equal and parallel

$\circ$ \ rectangle:  a parallelogram with four right angles

$\circ$ \ square:  a rectangle with four sides equal 

\begin{center} \includegraphics [scale=0.2] {rect4r.png} \end{center}

We usually specify these shapes according to the most restrictive conditions they meet.  When we think about a parallelogram we mean one without right angles, and when we think about a rectangle, we mean one that is not a square.

There are a few more quadrilaterals to mention in passing (left to right in the figure below):

\begin{center} \includegraphics [scale=0.20] {rect5b.png} \end{center}

$\circ$ \ rhombus:  a square-like parallelogram with all sides equal

$\circ$ \ trapezoid: just two sides parallel

$\circ$ \ kite: both pairs of \emph{adjacent} sides equal, one pair of opposite angles equal

$\circ$ \ dart: like the kite, but one vertex concave (angle $> 180$)

\subsection*{parallelograms}

Parallelograms are four-sided polygons with

$\circ$ \ both pairs of opposing sides parallel

$\circ$ \ both pairs of opposing sides equal

$\circ$ \ both pairs of opposing angles equal

It is convenient to think of parallelograms primarily in terms of the first property.  For example, parallel sides is enough to establish the other two properties.  

In fact, each can be derived easily from the others.  It is also sufficient if one pair of sides is both equal and parallel.

\subsection*{rectangles}

Rectangles are parallelograms with  

$\circ$ all four vertices containing right angles

We now look at some fundamental theorems about polygons.

\subsection*{sum of angles theorem}

$\bullet$ \ The sum of all the internal angles in any quadrilateral  is the same as that in two triangles, or four right angles altogether.

\emph{Proof}.

\begin{center} \includegraphics [scale=0.2] {quadrilateral.png} \end{center}

Connect two opposing vertices ($A$ and $C$ above) to form two triangles.  

Using the triangle sum theorem, add the component angles at all four vertices.

\[ \angle B + \angle BAC + \angle BCA = 180 \]
\[ \angle D + \angle DAC + \angle DCA = 180 \]

But 
\[ \angle A = \angle BAC + \angle DAC \]
\[ \angle C = \angle BCA + \angle DCA \]

So by addition:
\[ \angle A + \angle B + \angle C + \angle D = 360 \]

$\square$

This theorem can be extended to polygons having more sides.  

Adding a vertex (with a net addition of one side) is the same as adding another triangle.  Correcting for the ``base case'' we have the sum of angles $S = (n - 2) \cdot 180$.

The proof of the extended theorem is famously done by induction.  We cover this elsewhere.

\subsection*{diagonal theorem}

\label{sec:diagonal_theorem}

$\bullet$ \ Any diagonal in a parallelogram produces two congruent triangles.

\begin{center} \includegraphics [scale=0.16] {pgram_diags.png} \end{center}

\emph{Proof}.

\textbf{parallelogram}:  

$\triangle EFG \cong \triangle GHE$ and $\triangle EFH \cong \triangle GHF$.

$\circ$ \ Given $EF = GH$ and $EH = FG$, the result follows by by SSS.

$\circ$ \ Given $EF \parallel GH$ and $EH \parallel FG$, the result follows by by ASA.

$\circ$ \ Given $\angle E = \angle G$ and $\angle F = \angle H$, $\angle E + \angle F$ is equal to two right angles, by the sum of angles.  Thus, $EF \parallel GH$ and $EH \parallel FG$, and the result follows.

$\square$

Since every rectangle is also a parallelogram, the theorem applies to rectangles.

\textbf{rectangle}:  $\triangle ABC \cong \triangle CDA$.

\begin{center} \includegraphics [scale=0.30] {rect_pgram.png} \end{center}

\subsection*{diagonal theorem:  equal angles corollary}

$\bullet$ \ In any parallelogram the opposing angles are equal.

This follows immediately from the main theorem.

\subsection*{diagonal theorem:  bisection corollary}

$\bullet$ \ The two diagonals in a parallelogram bisect one another.

\begin{center} \includegraphics [scale=0.18] {rect_pgram2.png} \end{center}

\emph{Proof}.

Given $EH \parallel FG$ and $EH = FG$.

$\angle GEH = \angle EGF$ by alternate interior angles.

$\angle EMH = \angle FMG$ by vertical angles.

$\triangle EMH$ and $\triangle FMG$ are equiangular by sum of angles in a triangle.

$\triangle EMH \cong \triangle FMG$ by ASA.

It follows that $EM = GM$ and $FM = HM$. 

$\square$


\subsection*{special parallelogram theorem}

\label{sec:one_pair_of_sides}

We state this result separately to emphasize it.  We will need it for the theory of similar triangles.

In $EFGH$ let $EF = GH$ and $EF \parallel GH$.  Then $EFGH$ is a parallelogram.

\emph{Proof}.

\begin{center} \includegraphics [scale=0.18] {rect_pgram2.png} \end{center}

Draw diagonal $EG$.

By alternate interior angles, $\angle FEG = \angle EGH$.  

Given $EF = GH$ and $EG$ is shared.  $\triangle EGH \cong \triangle GEF$ by SAS.

By the diagonal theorem, it follows that $EFGH$ is a parallelogram.

$\square$


\subsection*{diagonal theorem:  converse 1}

$\bullet$ \ If the two diagonals in a quadrilateral bisect each another, the figure is a parallelogram.

\begin{center} \includegraphics [scale=0.18] {rect_pgram2.png} \end{center}

\emph{Proof}.

Given $EM = GM$ and $FM = HM$.

$\angle EMH = \angle GMF$ by vertical angles.

$\triangle EMH \cong \triangle GMF$ by SAS.

It follows that $EH = FG$ and $\angle GEH = \angle EGF$.

So $EH \parallel FG$ by alternate interior angles.

The same logic gives $EF$ parallel and equal to $HG$.

$\square$

Note that the diagonal theorem and its converse together allow us to interconvert all four specifications for a parallelogram.

\subsection*{diagonal theorem:  converse 2}

$\bullet$ \ If the diagonal in a quadrilateral forms congruent triangles, the figure is a parallelogram.

\begin{center} \includegraphics [scale=0.18] {rect_pgram2.png} \end{center}

\emph{Proof}.

Given $\triangle EHG \cong \triangle GFE$

$\angle GEH = \angle EGF$ and $\angle EGH = \angle GEF$.

We have both pairs of opposing sides parallel by the converse of alternate interior angles.

$\square$


\subsection*{midline theorem}

\label{sec:midline_theorem}

$\bullet$ \ The midline of a triangle is parallel to the third side of that triangle and half its length.

\begin{center} \includegraphics [scale=0.18] {midline_thm.png} \end{center}

\emph{Proof}.

Given $AM = BM$ and $BN = CN$.  Extend $MN$ to $D$ with $MN = DN$.  Draw $BD$, $CM$ and $CD$.

$BDCM$ is a parallelogram, by the converse of the diagonal theorem.

$CD = BM = AM$ and $CD \parallel AMB$.  It follows that $MDCA$ is a parallelogram.

Hence $AC = MD$ which is twice $MN$ and $AC \parallel MND$.

$\square$

\emph{Proof}.  (Alternate)

With the same construction, $MN = DN$, and given $BN = CN$, and vertical angles, we have $\triangle BMN \cong \triangle CDN$ by SAS.

It follows that $BM = CD$ hence $CD = AM$.  

Also from the congruent triangles, $\angle BMN = \angle CDN$.  By alternate interior angles, $BMA \parallel CD$.

Thus, $MDCA$ is a parallelogram, with $AC = MD = 2 MN$ and $AC \parallel MN$.

$\square$

\subsection*{Varignon's theorem}

\label{sec:Varignon_theorem}

This famous theorem concerns any quadrilateral.  Let's start with the four points lying flat in the same plane.  

If we draw the lines connecting the midpoints of each side, the result must be a parallelogram.  Here is Acheson's figure:

\begin{center} \includegraphics [scale=0.5] {Acheson_G50.png} \end{center}

To visualize this, imagine the quadrilateral drawn as two triangles connected at the diagonal.

\begin{center} \includegraphics [scale=0.5] {Acheson_G51.png} \end{center}

This idea about the diagonal contains the germ of the answer.  In the second figure, above, by the midline theorem, $EF \parallel BD$, but also $BD \parallel GH$.  

Thus $EF \parallel GH$.  We have opposite sides parallel.  Repeat with $FG$ and $EH$ to obtain the result.

Now, if we imagine the quadrilateral folding on a hinge at $DB$, we see that the midlines $EF$ and $GH$ will remain parallel even if $C$ is no longer co-planar with $A$, $D$ and $B$.

\end{document}