\documentclass[11pt, oneside]{article} 
\usepackage{geometry}
\geometry{letterpaper} 
\usepackage{graphicx}
	
\usepackage{amssymb}
\usepackage{amsmath}
\usepackage{parskip}
\usepackage{color}
\usepackage{hyperref}

\graphicspath{{/Users/telliott/Dropbox/Github-math/figures/}}
% \begin{center} \includegraphics [scale=0.4] {gauss3.png} \end{center}

\title{Rectangles}
\date{}

\begin{document}
\maketitle
\Large

%[my-super-duper-separator]

\subsection*{minimal specification}

In considering a shape that might be a rectangle, and given only some of the properties, the rest may follow.  There are too many permutations to go through them all.

First, if we know the figure is a parallelogram, then if it also has at least one right angle, it is a rectangle.

\emph{Proof}.

By the diagonal theorem, there are two opposing angles that are right angles.

Because opposite sides are parallel, it follows that if one of two adjacent angles is a right angle, then so is the other one, by alternate interior angles.

Thus, all four angles are right angles.

$\square$

If we do not know about the sides but only about the right angles, and that there are four of them, then we have a rectangle.

\emph{Proof}.

Since the adjacent angles are right, the sides in question are parallel.

A parallelogram with at least one right angle is a rectangle.

$\square$

Even if we know only three angles, by the sum of angles in a quadrilateral, we have four.

However, two right angles are not enough, as counter-examples are easily constructed for both cases (right angles as either neighbors or opposite one another).

\begin{center} \includegraphics [scale=0.35] {rect3.png} \end{center}

In the left panel $\angle A$ and $\angle B$ are right angles, and in the middle panel, $\angle F$ and $\angle H$ are right angles.  But neither $ABCD$ nor $EFGH$ is a rectangle.

If both pairs of sides are equal and parallel, and we don't know there is a right angle, that's not enough by itself.

Here's another permutation.

Let $AD = BC$ and let $\angle A$ and $\angle B$ both be right angles.  $ABCD$ is a rectangle.

\begin{center} \includegraphics [scale=0.35] {rect8.png} \end{center}

\emph{Proof.}

Bisect $AB$ at $E$ and draw the half-diagonals.  $\triangle ADE \cong \triangle BCE$ by SAS.  

Therefore $ED = EC$ so the angles marked with a red dot are equal by the forward isosceles theorem.  

The other components of $\angle C$ and $\angle D$ are equal because of the congruent triangles, so $\angle C = \angle D$.  

By the angle sum theorem for quadrilaterals and given the angles at $A$ and $B$ are right angles:
\[ \angle C + \angle D = 180 \]

Then, both are right angles and thus, all four angles are right angles.

$\square$

\subsection*{diagonals of a rectangle}

The diagonals of a rectangle are equal.

\begin{center} \includegraphics [scale=0.15] {rect9.png} \end{center}

\emph{Proof}.

We consider two overlapping triangles.

$\triangle ABC \cong \triangle BAD$ by SAS.

So $AC = BD$.

Since the diagonals cross at their midpoints, by the diagonal bisector corollary.

All four half-diagonals are equal.

It follows that the four smaller triangles are all isosceles.

$\square$

We can also invoke symmetry.

The rectangle has mirror image symmetry by reflection in both the left-right and top-bottom dimensions.  As a result, the black-dotted angles in the figure below are equal, and so are the red ones.

\begin{center} \includegraphics [scale=0.35] {rect2.png} \end{center}

And that implies that all four segments from a vertex to the central point are equal in length, by the converse of the isosceles triangle theorem.

\subsection*{bisection of a rectangle}

\label{rectangle_bisection}

Suppose we are given that $ABCD$ is a rectangle and that $EF$ is the perpendicular bisector of one of the sides, say $AD$.  Then $AE$ is also the perpendicular bisector of the other side, $BC$.

\begin{center} \includegraphics [scale=0.35] {rect6.png} \end{center}

\emph{Proof}.

Draw the diagonals of the two small quadrilaterals, namely $AF$ and $DF$.

Then $\triangle AEF \cong \triangle DEF$ by SAS, using the right angles at $E$.  

But $\triangle AEF$ is also congruent to $\triangle ABF$ (by SAS).  

Reasoning in the same way, and using transitivity, we have four congruent right triangles.

It follows easily that the two small rectangles $ABFE$ and $DCFE$ are congruent, so $\angle BFE$ and $\angle CFE$ are right angles, with $BF = CF$.

Therefore $EF$ is the perpendicular bisector of $BC$.

$\square$

An even simpler proof:  given $EF \parallel DC$ so $EDCF$ is a parallelogram.  Thus $ED = FC = BF$.

\subsection*{area of a rectangle}

Let's say a few brief words about rectilinear area, the area of shapes like squares or rectangles with perpendicular sides.  We will then extend this to the areas of triangles and parallelograms, which are like squashed rectangles.

To find areas, we must first fix a unit length.  For now, in geometry, we will need an even number of units for each dimension.  (There is an exception, but it occurs in a case where we only care about the squared area --- see the chapter on the Pythagorean theorem).

Suppose that, in the figure below, the small squares have side lengths of 1 cm, and 6 squares stack vertically and 8 horizontally to fill the shape.

\begin{center} \includegraphics [scale=0.35] {area5.png} \end{center}
Just multiply the width by the height (in cm) to obtain $48$ cm$^2$.

But then suppose instead that the squares have side lengths of $2.54$ cm.  Define $1$ in $= 2.54$ cm.  The total area would be $48$ in$^2$.

This particular figure above (from Lockhart) shows the distributive law in action:
\[ (3 + 5) \cdot (4 + 2) \]
\[ =3 \cdot 4 + 3 \cdot 2 + 5 \cdot 4 + 5 \cdot 2 \]
\[ = 48 \]
Any combination of numbers that add up to $8$, times any combination of numbers that add up to $6$, gives the same result.

\subsection*{problem}

From Jacobs, chapter 9.  

\begin{center} \includegraphics [scale=0.4] {sciam2.png} \end{center}

Suppose each of $A$ through $I$ is a square, and the areas of squares $C$ and $D$ are 64 and 81, respectively.  

Can you find the areas of all the other squares?  

Is the entire figure a square?  What is the total area?

\subsection*{problem}

Next is a problem from the web.  Given three identical squares arranged as follows:
\begin{center} \includegraphics [scale=0.75] {gardner7.png} \end{center}

There is a simple solution, to be obtained without measuring or using trigonometry.

As in so many problems, the key is to draw an inspired diagram, one that extends the figure somehow.  Here, a major hint was provided, namely, a grid of squares.
\begin{center} \includegraphics [scale=0.5] {gardner6b.png} \end{center}

So let's draw the same angles using that grid and form a triangle.
\begin{center} \includegraphics [scale=0.2] {gardner13.png} \end{center}

We can identify some equal angles.  The two with red dots are equal by alternate interior angles.  The two rectangles, one containing the diagonal $AC$ and the other the diagonal $BC$ each consist of three unit squares so they are congruent.  That plus alternate interior angles accounts for all the black dotted equal angles.

The last thing we can learn from the diagram is that because of congruent rectangles, $AC = BC$.  So that means $\angle BAC$ is equal to $\angle ABC$.

It follows that $\angle BAC$ is one-half the total angle at $A$, namely one-half of a right angle.

So finally, by sum of angles, $\triangle ABC$ is a right triangle.

Here is a similar proof, more compactly executed

\begin{center} \includegraphics [scale=0.4] {gardner12.png} \end{center}

\url{https://mathenchant.wordpress.com/2022/07/17/twisty-numbers-for-a-screwy-universe/}

(Endnote 5).

Annotated:
\begin{center} \includegraphics [scale=0.16] {pww_gardner2.png} \end{center}

$\square$

Martin Gardner has a version of this problem for which he gives this diagram:
\begin{center} \includegraphics [scale=0.3] {gardner1.png} \end{center}

\end{document}