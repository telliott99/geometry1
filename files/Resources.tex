\documentclass[11pt, oneside]{article} 
\usepackage{geometry}
\geometry{letterpaper} 
\usepackage{graphicx}
	
\usepackage{amssymb}
\usepackage{amsmath}
\usepackage{parskip}
\usepackage{color}
\usepackage{hyperref}

\graphicspath{{/Users/telliott/Dropbox/Github-math/figures/}}
% \begin{center} \includegraphics [scale=0.4] {gauss3.png} \end{center}

\title{Introduction}
\date{}

\begin{document}
\maketitle
\Large

%[my-super-duper-separator]

I have given very few problems, and solved nearly all of those, but you will need to do a lot of problems to really learn the material.  

Start with Simmons.  He summarizes Algebra, Geometry, Analytic Geometry and Trigonometry in a little more than one hundred pages.  The book can easily be found used, and I highly recommend it.

David Acheson's books are wonderful, including \emph{The Wonder Book of Geometry}.

Another resource that I enthusiastically recommend is a website with a highly annotated version of Euclid's \emph{Elements}.

\url{https://mathcs.clarku.edu/~djoyce/elements/elements.html}

You owe it to yourself to work your way through the first book, at least.

Here is an old textbook by Hopkins that I found online:

\url{https://archive.org/details/inductiveplanege00hopkrich}

As a unique feature it contains more than a dozen entrance examinations for major colleges from the years around 1900.

Two examples from Dartmouth:

\begin{center} \includegraphics [scale=0.4] {dartmouth_1900.png} \end{center}

\begin{center} \includegraphics [scale=0.4] {dartmouth_1901.png} \end{center}

\end{document}