\documentclass[11pt, oneside]{article} 
\usepackage{geometry}
\geometry{letterpaper} 
\usepackage{graphicx}
	
\usepackage{amssymb}
\usepackage{amsmath}
\usepackage{parskip}
\usepackage{color}
\usepackage{hyperref}

\graphicspath{{/Users/telliott/Dropbox/Github-Math/figures/}}
% \begin{center} \includegraphics [scale=0.4] {gauss3.png} \end{center}

\title{Area}
\date{}

\begin{document}
\maketitle
\Large

%[my-super-duper-separator]

In the chapter on congruence, we introduced similar triangles as triangles with the same three angles, but scaled differently.  

\begin{center} \includegraphics [scale=0.08] {similar26b.png} \end{center}

Let $\triangle ABC \sim \triangle ADE$. ($\sim$ is the symbol for similar).  Then they can nestle one inside the other, using any one of the shared angles.

If we do this so that $\angle ADE = \angle ABC$, then by alternate interior angles $DE \parallel BC$.

The other property is the equal proportions of sides.  
\[ \frac{AD}{BD} = \frac{AE}{CE} \]

equal angles and parallel third side $\iff$ equal ratios of sides.

\subsection*{similar right triangles}

\label{sec:similar_right_triangles}

We will show an easy proof that for similar right triangles, all angles equal implies equal ratios of sides.  Our approach is from Acheson and is equivalent to a previous result (Euclid I.43).

\begin{quote}Draw a rectangle ABCD, and a diagonal AC.  Then pick a point E on the diagonal and draw lines through it parallel to the sides.\end{quote}

\begin{center} \includegraphics [scale=0.6] {Acheson_G42.png} \end{center}

All of the right triangles in the figure are similar.  Start with the alternate interior angles theorem, then use complementary angles in a right triangle, and finish with vertical angles.  

By changing the height of the figure, we can obtain any ratio $a/c$ that we wish, and by changing the placement of $E$ we can get any ratio $a/b$ that we wish, which amounts to the same thing.

So then, the two shaded rectangles are bisected by the diagonal $AEC$, by the diagonal theorem.  So the two light-gray triangles have equal area, and the two dark gray ones do as well.

But $\triangle ABC$ and $\triangle ADC$ also have equal area.

Therefore, we just subtract equal areas to find that the two unshaded rectangles above and below the diagonal are equal in area.  The one on top has area $bc$ and the one below has area $ad$.  We have

\[ bc = ad \]
\[ \frac{a}{c} = \frac{b}{d} \]
and also
\[ \frac{a}{b} = \frac{c}{d} \]

$\square$

Corresponding sides are in the same proportion, but also the ratio of corresponding sides is the same for similar triangles.

\subsection*{similar parts and the whole}

\label{sec:parts_and_whole}

We showed that the two smaller triangles have corresponding sides in proportion. 

But the large triangle (one-half of the entire rectangle) has the same angles, and should have the same ratios.  Here is a simple manipulation to obtain that result:

\begin{center} \includegraphics [scale=0.4] {Acheson_G42b.png} \end{center}

\[ bc = ad \]
\[ bc + ab = ad + ab \]
\[ b(a + c) = a(b + d) \]
\[ \frac{a}{b} = \frac{a + c}{b + d} \]

Given any two of these relationships we can derive the third.  This is the same math in reverse.

\[ \frac{a + c}{b + d} = \frac{c}{d} \]
\[ \frac{a + c}{c} = \frac{b + d}{d} \]
\[ \frac{a}{c} + 1 = \frac{b}{d} + 1 \]
\[ ad = bc \]

$\square$

\subsection*{hypotenuse in proportion}

It is natural to ask, what about the hypotenuse?

Consider any right triangle.  Drop the altitude to the hypotenuse.

Using complementary angles, we can show that the two new smaller triangles formed by the altitude are similar to each other and to the original large one.
\begin{center} \includegraphics [scale=0.45] {triangle3.png} \end{center}
Now, write the ratio of the long to the short side \emph{for each one} of the three triangles:
\[ \frac{h}{x} = \frac{y}{h} = \frac{b}{a} = k \]

From our previous theorem we know this equality is valid.

However the same statement can be viewed in a different way.  

$b/a$ is the ratio for the hypotenuse of the medium triangle compared to the small one.  

And $y/h$ is the ratio of the long side in the medium triangle to the long side in the small one.

They are equal, and this completes the proof!

$\square$

We can also give an algebraic proof, by looking ahead to the Pythagorean theorem.  As you likely know, for a right triangle with sides $a$ and $b$ and hypotenuse $g$:
\[ a^2 + b^2 = g^2 \]

We can use the Pythagorean theorem to prove that:
\[ \frac{a}{c} = \frac{b}{d} = \frac{g}{h} \]

\emph{All} of the sides of two similar right triangles have the same ratio.  

We must be careful, however.  A deep connection exists between similarity, area and the Pythagorean theorem.  It is important that we will have Euclid's proof of the Pythagorean theorem, and that proof depends on SAS rather than on similarity.

Equal ratios extends to the hypotenuse.

\emph{Proof}.

Start with 
\[ \frac{a}{c} = \frac{b}{d} = k \]
\[ a = kc, \ \ \ \ \ \ b = kd \]
\[ a^2 + b^2 = k^2c^2 + k^2d^2 \]
\[ g^2 = k^2h^2 \]

Since these are lengths, we can take the positive square root and obtain

\[ \frac{g}{h} = k \]
\[ = \frac{a}{c} = \frac{b}{d} \]

$\square$

Thus, AAA similarity is established for right triangles.  If either of the smaller angles matches between two right triangles, then they are not only similar but all the side lengths are in the same ratio as well.

Getting the converse, going from equal ratios to equal angles, uses a result about parallelograms.  Also, the proofs apply not just to right triangles, but to triangles of any type.  For that reason, we defer further development of similarity to a later chapter.

We proved this above for right triangles, and could extend the result to all triangles by dissection.  We just show the figure and sketch the proof.

\begin{center} \includegraphics [scale=0.25] {similar14b.png} \end{center}

$\triangle ABC$ and $\triangle CDE$ are similar because they have the same angles at all three vertices.  It follows that corresponding sides are parallel (e.g. $AB \parallel CD$, $AC \parallel DE$, and also the altitudes $AG \parallel DH$).  Equal ratios also comes easily.

We promise to prove this for all triangles later when we deal with the theory of similarity more explicitly.  Euclid VI.2 does this elegantly.

We can prove it for a special case now.

\subsection*{triangle dissection}

$\bullet$ \ Any triangle can be dissected into four congruent triangles.

\begin{center} \includegraphics [scale=0.2] {rot_triangle.png} \end{center}

\emph{Proof}.

Find the midpoints of the sides and connect them.

By the midline theorem each side of the central triangle is one-half the length of the side to which it is parallel.

\begin{center} \includegraphics [scale=0.18] {rot_triangle2.png} \end{center}

We have all congruent triangles by SSS.

Also, as a result we have 3 parallelograms, each containing two congruent triangles.

$\square$

\subsection*{dissection into right triangles}

Any triangle can be cut into two right triangles by drawing its \emph{altitude}, $h$, where $h$ is perpendicular to $c$.  

\begin{center} \includegraphics [scale=0.4] {area10.png} \end{center}

The area of the triangle with sides $a,d,h$ is $dh/2$, and that with sides $a,e,h$ is $eh/2$ so the area of the original triangle is
\[ \frac{dh}{2} + \frac{eh}{2} = \frac{(d+e)h}{2} = \frac{ch}{2} \]

This formula is correct even for an obtuse triangle like the one in the right panel, below.  The area of the two red triangles is the same:  $ch/2$.

\begin{center} \includegraphics [scale=0.4] {area_obtuse.png} \end{center}

We get that by computing the area of the large triangle with base $c + c'$ and then subtracting the area of the skinny triangle with the base $c'$: 
\[ \mathcal{A} = \frac{h(c + c') - h(c')}{2} = \frac{hc}{2} \]

Of course, for an obtuse triangle we could choose one of the other sides as the base and proceed in the usual way, but this works as well.

\end{document}