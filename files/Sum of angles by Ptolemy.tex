\documentclass[11pt, oneside]{article} 
\usepackage{geometry}
\geometry{letterpaper} 
\usepackage{graphicx}
	
\usepackage{amssymb}
\usepackage{amsmath}
\usepackage{parskip}
\usepackage{color}
\usepackage{hyperref}

\graphicspath{{/Users/telliott/Dropbox/Github-Math/figures/}}
% \begin{center} \includegraphics [scale=0.4] {gauss3.png} \end{center}

\title{Basic trigonometry}
\date{}

\begin{document}
\maketitle
\Large

%[my-super-duper-separator]

\subsection*{general sum of angles}

Previously, we saw the double angle formulas
\[ \sin 2A =  2 \sin A \cos A \]
\[ \cos 2A = \cos^2 A - \sin^2 A \]

We'd like to find a general formula for the sine of two angles added together.  $\sin A + B = \ ?$  

Of course, if $A = B$ then the formula must reduce to what we have above, so it seems there are two reasonable possibilities if we keep things simple:
\[ \sin A + B \stackrel{?}{=} \sin A \cos A + \sin B \cos B \]
\[ \sin A + B = \sin A \cos B + \sin B \cos A \]

Without explaining it at present, we will just assume it is known that $\sin 0 = 0$ and $\cos 0 = 1$.  It is easy to see that the sine and cosine approach these values for very small angles but a proper explanation will have to wait for analytic geometry.

If you start with any angle $A$, set $B = 0$, and then use the first formula it gives $\sin A + 0 = \sin A \cos A + 0 \cdot 1$ which means $\sin A = \sin A \cos A$ which makes no sense.  Indeed, we will now derive the second formula as the correct result.  

If we start with the cosine formula there seems to be only one simple candidate, namely
\[ \cos A + B = \cos A \cos B - \sin A \sin B \]

We use Ptolemy's theorem, which we proved previously using the properties of parallelograms.  In fact, a big reason for including this section now is to illustrate the power of this theorem, which is normally thought to be difficult, but since we found an easy proof we are in great shape.

Ptolemy's theorem says that if a four-sided figure, a quadrilateral, has all of its vertices on a circle, then we can form the two products of opposing sides and add them together to obtain the product of the diagonals.
\[ AB \cdot CD + AD \cdot BC = AC \cdot BD \]
\begin{center} \includegraphics [scale=0.55] {pt1.png} \end{center}

Our first proof of this is \hyperref[sec:Ptolemy]{\textbf{here}}.  And we'll revisit the topic later.

Notice that if the quadrilateral is a rectangle, then the diagonals of the rectangle are also diagonals of the circle (\hyperref[sec:rectangle_in_a_circle]{\textbf{here}}), so this theorem gives a simple proof of the Pythagorean theorem.

\subsection*{sum of angles}
\label{sec:sum_angles_Ptolemy}

Ptolemy's theorem can be used to give direct proofs of the sum (and difference) of angles formulas for both sine and cosine.  It's a fun exercise because the results come easily from inspired diagrams with slight changes between them.

It helps that we have an idea about what we want.

We need to say a bit more before we start.  Each diagram contains a diameter of the circle, scaled so its length is $1$.  

In two cases that length is also a diagonal of the quadrilateral, and in the other two it is a side.  In the first case, the product with the other diagonal of the quadrilateral is just equal to whatever value that second diagonal has.  Here is the first figure:
\begin{center} \includegraphics [scale=0.5] {pt19.png} \end{center}

There are two right angles on the circle with sides extending to the ends of the diameter.  This makes the setup a natural for sine and cosine formulas.

Second, a peripheral or inscribed angle $\theta$ is related to its chord length by $L = 2R \sin \theta$.  We saw this important result previously (\hyperref[sec:sine_secant]{\textbf{here}}).  Since we have $2R = 1$, this simplifies to $L = \sin \theta$, where in this case $\theta = s + t$.

We start with the sum of sines.  The idea is that $\sin s + t$ should be the second diagonal.

\begin{center} \includegraphics [scale=0.5] {pt19.png} \end{center}
From there the formula basically writes itself.
\[ \sin s + t = \sin s \cos t + \sin t \cos s \]

Our analysis of what was likely to be the form of the final result turns out to be correct.

\subsection*{sine of the difference}

Algebraically, the difference of sines is easily derived using the fact that cosine is an even function, $\cos (x) = \cos (-x)$ while sine is odd, so $\sin (x) = - \sin (-x)$.  

Thus, substituting $-t$ for $t$ changes the "sign" of the second term but not the first.
\[ \sin s + (- t) = \sin s \cos (- t) + \sin (- t) \cos s \]
\[ = \sin s \cos t - \sin t \cos s \]

However, we can use a diagram like the one above to do this formula too.  See:

\url{https://www.cut-the-knot.org/proofs/sine_cosine.shtml}

The trick is that, because of the minus sign on the right-hand side in the final formula, we want $\sin s - t$ to be one of the sides.  So the diagonal of the circle must be opposite, and also be a side of the quadrilateral.  

Now, $s$ is the whole angle in the red triangle.

\begin{center} \includegraphics [scale=0.5] {pt20.png} \end{center}
And again, the formula writes itself:
\[ \sin (s - t) + \sin t \cos s = \sin s \cos t \]
which rearranges to give
\[ \sin s - t = \sin s \cos t - \sin t \cos s \]

\subsection*{cosine of the sum}

For the cosine formulas, we'll need to relate them to the sine of another angle.  Recall that if 
\[ s + t + u = 90 \]
then
\[ \cos (s + t) = \sin u = \sin 90 - (s + t) \]

So for the first cosine formula we add an additional angle $u$, and use the fact that $s+t$ is complementary to $u$.

We make $\sin u = \cos s + t$ one of the sides, because we know the formula has a minus sign in it.
\begin{center} \includegraphics [scale=0.5] {pt21.png} \end{center}
\[ \sin s \sin t + \cos (s + t)  = \cos s \cos t \]
\[ \cos s + t = \cos s \cos t - \sin s \sin t \]

\subsection*{cosine of the difference}

Last is the difference formula for cosine.  Again, we can derive this formula easily by using the fact that cosine is even and sine is odd.

Knowing where we're headed, we want opposing sides to be both sine or both cosine, and we have them as sides, because they add in the final formula.

So then, somehow, the dotted diagonal must be $\cos s - t$.
\begin{center} \includegraphics [scale=0.5] {pt22b.png} \end{center}

The complementary angle to $s$ is $90 - s$.  Adding it to the adjacent angle $t$ we have that the dotted diagonal is $\sin (90 - s) + t$.  But that is 
\[ \cos 90 - \ [ \ (90 - s) + t \ ] \ = \cos s - t \]
which is just what we needed!

\[ \cos s - t = \cos s \cos t + \sin s \sin t \]

Perhaps you may object that, from the diagram it looks like $s > t$ so $t - s < 0$.  But we can also use the angle on the opposite side and write:
\[ \sin s + (90 - t) = \cos 90 - \ [ \ (90 - t) + s \ ]  \]
\[ = \cos t - s \]

Again, this works because cosine is an even function so $\cos s - t = \cos t - s$, or if you prefer you can use the fact that the sine of an angle is equal to the the sine of its supplementary angle, which is obvious from the diagram.
\[ \sin s + (90 - t) = \sin 180 - \ [ \ s + (90 - t) \ ] = \sin t + (90 - s) \]


\end{document}