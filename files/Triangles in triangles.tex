\documentclass[11pt, oneside]{article} 
\usepackage{geometry}
\geometry{letterpaper} 
\usepackage{graphicx}
	
\usepackage{amssymb}
\usepackage{amsmath}
\usepackage{parskip}
\usepackage{color}
\usepackage{hyperref}

\graphicspath{{/Users/telliott/Dropbox/Github-math/figures/}}
% \begin{center} \includegraphics [scale=0.4] {gauss3.png} \end{center}

\title{Incenter, Orthocenter, Circumcenter}
\date{}

\begin{document}
\maketitle
\Large

%[my-super-duper-separator]

\begin{center} \includegraphics [scale=0.4] {gauss_crop.png} \end{center}

In the figure above, the outer red triangle lies on its circumcircle, where the circumcenter is the white point at the center.  

The midpoints of the sides of the outer triangle are joined to form a second triangle, in blue.  We will show that the perpendicular bisectors of the sides of the outer triangle are the altitudes of the blue triangle.   The proof is due to Gauss.

Even more interesting, if the feet of the altitudes of the blue triangle are joined to make a third triangle, the angle bisectors of that triangle meet at an incenter, I.  The incenter of the dark gray third triangle is the orthocenter of the blue triangle and also the circumcenter of the original red triangle.

\subsection*{Gauss and altitude}

\label{sec:Gauss_orthocenter}

\emph{Proof}.

Recall that we can find the circumcenter of any triangle by erecting perpendicular bisectors at the midpoints of the sides.  

But those midpoints are the vertices of the blue triangle.  And in drawing the triangle on the midpoints, by the midpoint theorem the resulting triangle has sides parallel to the original.  So the perpendicular bisectors of the first triangle are perpendicular to the sides of the second, and also go through its vertices.  Thus, they are altitudes of the second triangle.

$\square$

\begin{center} \includegraphics [scale=0.4] {gauss_crop.png} \end{center}

\subsection*{orthocenter and incenter}

\label{sec:orthocenter_and_incenter}

We have proven previously that the three altitudes meet at a single point, the orthocenter.  The proofs include one from \hyperref[sec:Newton_altitude]{\textbf{Newton}}, and the previous one (from \hyperref[sec:Gauss_orthocenter]{\textbf{Gauss}}).

\begin{center} \includegraphics [scale=0.4] {altitude_proof_1.png} \end{center}

Above we have drawn the altitudes (left panel) and then also connected the points where the altitudes meet the sides at right angles.  We will prove that the dotted lines are the bisectors of the angles at the vertices of the small inset triangle. 

In other words, the incenter of the small triangle is the same point as the orthocenter of the bigger one.  

\emph{Proof}.

The key to the proof is to recognize that we can use a part of an altitude as the diameter of a circle.  Draw the circle that has for its diameter the line segment connecting the orthocenter $H$ and vertex $B$ of the large blue triangle.

In this diagram we have changed the color of the inner triangle to be red, and also, the orthocenter of the blue triangle, where the altitudes cross, is not labeled because the figure is already too busy.  It is, however, marked with a red dot.

\begin{center} \includegraphics [scale=0.4] {ortho_incenter_crop.png} \end{center}

Now consider the parts of the other two altitudes that terminate at $D$ and $F$   I claim that these two points lie on the same circle.  

The reason is that, each one individually, taken together with the first two points, forms a right triangle.  By the converse of Thales theorem, they must lie on the circle. 

The same thing can be done with $CH$ as the diameter of a different circle, through  $D$ and $E$.

Now we can use the \hyperref[sec:equal_angles_same_arc]{\textbf{theorem}} about arcs that subtend an angle on the perimeter of the circle.  

$\angle ADF = \angle EBA$ for this reason, while $\angle ACF = \angle ADE$ as well.

But then we notice that $\triangle HEC$ and $\triangle HFB$ are right triangles that share vertical angles, so they are similar.  In particular $\angle ACF = \angle EBA$.

Thus, all four angles marked with black dots are equal.  Therefore, the angle at vertex $D$ in $\triangle EDF$ is bisected by the altitude of the outer triangle $AD$.

Exactly the same logic will show that the angles at the other two vertices of $\triangle DEF$ are also bisected by the altitudes of $\triangle ABC$.

\begin{center} \includegraphics [scale=0.4] {ortho_incenter_crop.png} \end{center}

We conclude that the orthocenter $H$ of the blue triangle is the incenter $I$ of the red triangle.

$\square$

\end{document}